% $Id$ %
\chapter{Browsing and playing}
\section{\label{ref:file_browser}File Browser}
\screenshot{rockbox_interface/images/ss-file-browser}{The file browser}{}
Rockbox lets you browse your music in either of two ways. The 
\setting{File Browser} lets you navigate through the files and directories on 
your \dap, entering directories and executing the default action on each file.
To help differentiate files, each file format is displayed with an icon. 

The \setting{Database Browser}, on the other hand, allows you to navigate 
through the music on your player using categories like album, artist, genre,
etc.

You can select whether to browse using the \setting{File Browser} or the
\setting{Database Browser} by selecting either \setting{Files} or
\setting{Database} in the \setting{Main Menu}.
If you choose the \setting{File Browser}, the \setting{Show Files} setting
lets you select what types of files you wish to view. See
\reference{ref:ShowFiles} for more information on the \setting{Show Files}
setting.

\note{The \setting{File Browser} allows you to manipulate your files in ways
that are not available within the \setting{Database Browser}. Read more about
\setting{Database} in \reference{ref:database}. The remainder of this section
deals with the \setting{File Browser}.}

\opt{ondio}{
Unlike the Archos Firmware, Rockbox provides multivolume support for the
MultiMediaCard, this means the \dap{} can access both data volumes (internal
memory and the MMC), thus being able to for instance, build playlists with
files from both volumes.
In the \setting{File Browser} a new directory will appear as soon as the device
has read the content after inserting the card. This new directory's name is
generated as \fname{<MMC1>}, and will behave exactly as any other directory
on the \dap{}.
}

\opt{h10,h10_5gb}{\note{
If your \dap{} is a MTP model, the Music directory where all your music is stored
may be hidden in the \setting{File Browser}. This may be fixed by either
either changing its properties (on a computer) to not hidden, or by changing
the \setting{Show Files} setting to all.
}}

\subsection{\label{ref:controls}File Browser Controls}
\begin{table}
    \begin{btnmap}{}{}
      \ActionStdPrev{}/\ActionStdNext{}
      \opt{HAVEREMOTEKEYMAP}{& \ActionRCStdPrev{}/\ActionRCStdNext{}}
         & Goes to previous/next item in list. If you are on the first/last 
           entry, the cursor will wrap to the last/first entry.\\
      %
      \opt{IRIVER_H100_PAD,IRIVER_H300_PAD,RECORDER_PAD}
        {
          \ButtonOn+\ButtonUp{}/ \ButtonDown
          \opt{HAVEREMOTEKEYMAP}{&
            \opt{IRIVER_RC_H100_PAD}{\ButtonRCSource{}/ \ButtonRCBitrate}
          }
          & Moves one page up/down in the list.\\
        }
      \opt{IRIVER_H10_PAD}
        {
          \ButtonRew{}/ \ButtonFF
          & Moves one page up/down in the list.\\
        }
      %
      \ActionTreeParentDirectory
      \opt{HAVEREMOTEKEYMAP}{& \ActionRCTreeParentDirectory}
      & Goes to the parent directory.\\
      %
      \ActionTreeEnter
      \opt{HAVEREMOTEKEYMAP}{& \ActionRCTreeEnter}
      & Executes the default action on the selected file or enters a
        directory.\\
      %
      \ActionTreeWps 
      \opt{HAVEREMOTEKEYMAP}{& \ActionRCTreeWps}
         & If there is an audio file playing, returns to the
         \setting{While Playing Screen} (WPS) without stopping playback.\\
      %
      \nopt{player,c200}%
        {%
          \ActionTreeStop 
          \opt{HAVEREMOTEKEYMAP}{& \ActionRCTreeStop}
          & Stops audio playback.\\%
        }%
      %
      \ActionStdContext{}
      \opt{HAVEREMOTEKEYMAP}{& \ActionRCStdContext}
      & Enters the \setting{Context Menu}\\
      %
      \ActionStdMenu{}
      \opt{HAVEREMOTEKEYMAP}{& \ActionRCStdMenu}
      & Enters the \setting{Main Menu}\\
      %
      \opt{quickscreen}{
        \ActionStdQuickScreen
        \opt{HAVEREMOTEKEYMAP}{& \ActionRCStdQuickScreen}
        & Switches to the \setting{Quick Screen}
        (see \reference{ref:QuickScreen}) \\
      }
      \opt{RECORDER_PAD}{
        \ButtonFThree & Switches to the Display Quick Screen \\ 
        %
      }
      %
      \opt{SANSA_E200_PAD}{
        \ActionStdRec & Switches to the Recording screen \\
      %
      }
    \end{btnmap}
\end{table}

\opt{RECORDER_PAD}{
  The functions of the F keys are also summarised on the button bar at the
  bottom of the screen.
}

\subsection{\label{ref:Contextmenu}\label{ref:PartIISectionFM}Context Menu}
\screenshot{rockbox_interface/images/ss-context-menu}{The Context Menu}{}

The \setting{Context Menu} allows you to perform certain operations on files or 
directories.  To access the \setting{Context Menu}, position the selector over a file 
or directory and access the context menu with \ActionStdContext{}.\\

\note{The \setting{Context Menu} is a context sensitive menu.  If the 
\setting{Context Menu} is invoked on a file, it will display options available 
for files.  If the \setting{Context Menu} is invoked on a directory, 
it will display options for directories.\\}

The \setting{Context Menu} contains the following options (unless otherwise noted, 
each option pertains both to files and directories):

\begin{description}
\item [Playlist.]
  Enters the \setting{Playlist Submenu} (see \reference{ref:playlist_submenu}).
\item [Playlist Catalog.]
  Enters the \setting{Playlist Catalog Submenu} (see 
  \reference{ref:playlist_catalog}).
\item [Rename.]
  This function lets the user modify the name of a file or directory.
\item [Cut.]
  Copies the name of the currently selected file or directory to the clipboard
  and marks it to be `cut'.
\item [Copy.]
  Copies the name of the currently selected file or directory to the clipboard
  and marks it to be `copied'.
\item [Paste.]
  Only visible if a file or directory name is on the clipboard. When selected
  it will move or copy the clipboard to the current directory.
\item [Delete.]
  Deletes the currently selected file. This option applies only to files, and
  not to directories. Rockbox will ask for confirmation before deleting a file.
  Press \ActionYesNoAccept{}
  to confirm deletion or any other key to cancel.
\item [Delete Directory.]
  Deletes the currently selected directory and all of the files and subdirectories
  it may contain. Deleted directories cannot be recovered. Use this feature with
  caution!
\opt{lcd_non-mono}{
\item [Set As Backdrop.]
  Set the selected \fname{bmp} file as background image. The bitmaps need to meet the
  conditions explained in \reference{ref:LoadingBackdrops}.
}
\item [Open with.]
  Runs a viewer plugin on the file. Normally, when a file is selected in Rockbox,
  Rockbox automatically detects the file type and runs the appropriate plugin.
  The \setting{Open With} function can be used to override the default action and
  select a viewer by hand.
  For example, this function can be used to view a text file
  even if the file has a non-standard extension (i.e., the file has an extension
  of something other than \fname{.txt}). See \reference{ref:Viewersplugins}
  for more details on viewers.
\item [Create Directory.]
  Create a new directory in the current directory on the disk.
\item [Properties.]
  Shows properties such as size and the time and date of the last modification
  for the selected file. If used on a directory, the number of files and
  subdirectories will be shown, as well as the total size.
\opt{recording}{
  \item [Set As Recording Directory.]
    Save recordings in the selected directory.
}
\item [Add to Shortcuts.]
  Adds a link to the selected item in the \fname{shortcuts.link} file.
  If the file does not already exist it will be created in the root directory.
  Note that if you create a shortcut to a file, Rockbox will not open it upon
  selecting, but simply bring you to it's location in the \setting{File Browser}.
\end{description}

\subsection{\label{sec:virtual_keyboard}Virtual Keyboard}
\screenshot{rockbox_interface/images/ss-virtual-keyboard}{The virtual keyboard}{}
This is the virtual keyboard that is used when entering text in Rockbox, for 
example when renaming a file or creating a new directory.
\nopt{player}{The virtual keyboard can be easily changed by making a text file
 with the required layout. More information on how to achieve this can be found
 on the Rockbox website at \wikilink{LoadableKeyboardLayouts}.}

\opt{morse_input}{
  Also you can switch to Morse code input mode by changing the
  \setting{Use Morse Code Input} setting%
  \opt{IRIVER_H100_PAD,IRIVER_H300_PAD,IPOD_4G_PAD,IPOD_3G_PAD,IRIVER_H10_PAD%
      ,GIGABEAT_PAD,GIGABEAT_S_PAD,MROBE100_PAD,SANSA_E200_PAD}%
    { or by pressing \ActionKbdMorseInput{} in the virtual keyboard}%
  .
}

\opt{IRIVER_H100_PAD,IRIVER_H300_PAD,RECORDER_PAD,GIGABEAT_PAD,GIGABEAT_S_PAD%
    ,MROBE100_PAD,SANSA_E200_PAD,SANSA_FUZE_PAD,SANSA_C200_PAD}{
  \begin{table}
    \begin{btnmap}{}{}
      \ActionKbdLeft{} / \ActionKbdRight{} / \ActionKbdUp{} / \ActionKbdDown
      \opt{HAVEREMOTEKEYMAP}{
      & \ActionRCKbdLeft{} / \ActionRCKbdRight{} / \ActionRCKbdUp{} / \ActionRCKbdDown}
      & Moves the cursor on the virtual keyboard. \\
      %
      \ActionKbdPageFlip
      \opt{HAVEREMOTEKEYMAP}{& \ActionRCKbdPageFlip}
      & Flips to the next page of characters (if there is more than one) \\
      %
      \ActionKbdCursorLeft{} / \ActionKbdCursorRight
      \opt{HAVEREMOTEKEYMAP}{& \ActionRCKbdCursorLeft{} / \ActionRCKbdCursorRight}
      & Moves the line cursor within the text line \\
      %
      \ActionKbdSelect
      \opt{HAVEREMOTEKEYMAP}{& \ActionRCKbdSelect}
      & Inserts the selected keyboard letter at the current line cursor position \\
      %
      \ActionKbdBackSpace
      \opt{HAVEREMOTEKEYMAP}{& \ActionRCKbdBackSpace}
      & Deletes the character before the line cursor \\
      %
      \ActionKbdDone
      \opt{HAVEREMOTEKEYMAP}{& \ActionRCKbdDone}
      & Exits the virtual keyboard and saves any changes \\
      %
      \ActionKbdAbort
      \opt{HAVEREMOTEKEYMAP}{& \ActionRCKbdAbort}
      & Exits the virtual keyboard without saving any changes \\
      %
      \opt{morse_input}{
        \opt{IRIVER_H100_PAD,IRIVER_H300_PAD,GIGABEAT_PAD,GIGABEAT_S_PAD%
            ,MROBE100_PAD,SANSA_E200_PAD}{
          \ActionKbdMorseInput
          \opt{HAVEREMOTEKEYMAP}{& \ActionRCKbdMorseInput}
          & Toggles keyboard input mode and Morse code input mode. \\
        }
        %
        \ActionKbdMorseSelect
        \opt{HAVEREMOTEKEYMAP}{& \ActionRCKbdMorseSelect}
        & Tap to select a character in Morse code input mode. \\
        %
      }
    \end{btnmap}
  \end{table}
}
\opt{ONDIO_PAD,IPOD_4G_PAD,IPOD_3G_PAD,IRIVER_H10_PAD,IAUDIO_X5_PAD,IAUDIO_M3_PAD}{
  \textbf{Picker area}
  \begin{table}
    \begin{btnmap}{}{}
      \ActionKbdLeft{} / \ActionKbdRight
      \opt{HAVEREMOTEKEYMAP}{& }
      & Moves the cursor on the virtual keyboard.
        If you move out of the picker area, you get the previous/next page of
        characters (if there is more than one). \\
      %
      \ActionKbdUp{} / \ActionKbdDown
      \opt{HAVEREMOTEKEYMAP}{& }
      & Moves the cursor on the virtual keyboard.
        If you move out of the picker area you get to the line edit mode. \\
      %
      \ActionKbdSelect
      \opt{HAVEREMOTEKEYMAP}{& }
      & Inserts the selected keyboard letter at the current line cursor position \\
      %
      \ActionKbdDone
      \opt{HAVEREMOTEKEYMAP}{& }
      & Exits the virtual keyboard and saves any changes \\
      %
      \ActionKbdAbort
      \opt{HAVEREMOTEKEYMAP}{& }
      & Exits the virtual keyboard without saving any changes \\
      %
      \opt{morse_input}{
        \opt{IPOD_4G_PAD,IPOD_3G_PAD,IRIVER_H10_PAD}{
          \ActionKbdMorseInput
          \opt{HAVEREMOTEKEYMAP}{& }
          & Toggles keyboard input mode and Morse code input mode. \\
        }
        %
        \ActionKbdMorseSelect
        \opt{HAVEREMOTEKEYMAP}{& \ActionRCKbdMorseSelect}
        & Tap to select a character in Morse code input mode. \\
        %
      }
    \end{btnmap}
  \end{table}

  \textbf{Line edit mode}
  \begin{table}
    \begin{btnmap}{}{}
      \ActionKbdLeft{} / \ActionKbdRight
      \opt{HAVEREMOTEKEYMAP}{& }
      & Moves the line cursor within the text line \\
      %
      \ActionKbdSelect
      \opt{HAVEREMOTEKEYMAP}{& }
      & Deletes the character before the line cursor \\
      %
      \ActionKbdUp{} / \ActionKbdDown
      \opt{HAVEREMOTEKEYMAP}{& }
      & Returns to the picker area \\
      %
    \end{btnmap}
  \end{table}
}
\opt{player}{
  The current text line to be entered or edited is always listed on the first
  line of the display. The second line of the display can contain the character
  selection bar, as in the screenshot above.
  \begin{table}
    \begin{btnmap}{}{}
      \ButtonOn & Toggles picker- and line edit mode \\
      \ButtonLeft{} / \ButtonRight
        & Moves back and forth in the selected line (picker of input line) \\
      \ButtonPlay
        & Picks character in character bar, or acts as backspace in the text line. \\
      Long \ButtonPlay & Accept \\
      \ButtonStop & Cancel \\
      \ButtonMenu & Flips picker lines \\
    \end{btnmap}
  \end{table}
}

% $Id$ %
\section{\label{ref:database}Database}

\subsection{Introduction}
This chapter describes the Rockbox music database system. Using the information
contained in the tags (ID3v1, ID3v2%
  \opt{SWCODEC}{, Vorbis Comments, Apev2, etc.}%
) in your audio files, Rockbox builds and maintains a database of the music
files on your player and allows you to browse them by Artist, Album and Genre.

\subsection{Initializing the database}
The first time you use the database, Rockbox will scan your disk for audio files.
This can take quite a while depending on the number of files on your \dap{}.
This scan happens in the background, so you can choose to return to the
Main Menu and continue to listen to music.
If you shut down your player, the scan will continue next time you turn it on.
After the scan is finished you may be prompted to restart your \dap{} before
you can use the database.

\subsubsection{Ignoring directories during database initialization}

You may have directories on your \dap{} whose contents should not be added
to the database. Placing a file named \fname{database.ignore} in a directory
will exclude the files in that directory and all its subdirectories from
scanning their tags and adding them to the database. This will speed up the
database initialization.

If a subdirectory of an 'ignored' directory should still be scanned, place a
file named \fname{database.unignore} in it. The files in that directory and
its subdirectories will be scanned and added to the database.

\subsection{\label{ref:databasemenu}The Database Menu}

\begin{description}
  \opt{SWCODEC}{
  \item[Load To Ram.]
    The database can either be kept on disk (to save memory), or
    loaded into RAM (for fast browsing). Setting this to \setting{Yes} loads
    the database to RAM, allowing faster browsing and searching. Setting this
    option to \setting{No} keeps the database on the disk, meaning slower 
    browsing but it does not use extra RAM and saves some battery on boot up. 
    
    \note{If you browse your music frequently using the database, you should
      load to RAM, as this will reduce the overall battery consumption because
      the disk will not need to spin on each search.}
  }
  
\item[Auto Update.]
  If \setting{Auto update} is set to \setting{on}, each time the \dap{}
  boots, the database will automatically be updated.
  \opt{SWCODEC}{
    \note{The \setting{Auto Update} will only check for deleted files if the
      \setting{Directory Cache} (\setting{Settings $\rightarrow$ General
      Settings $\rightarrow$ System $\rightarrow$ Disk $\rightarrow$
      Directory Cache}) is enabled. \setting{Update now} includes that check
      whether dircache has been enabled or not.}
  }%
  \opt{MASCODEC}{\setting{Auto Update} does not detect deleted files. To remove
    deleted files from the database you need to run \setting{Update Now}.}%

\item[Initialize Now.]
  You can force Rockbox to rescan your disk for tagged files by
  using the \setting{Initialize Now} function in the \setting{Database
    Menu}.
  \warn{\setting{Initialize Now} removes all database files (removing
    runtimedb data also) and rebuilds the database from scratch.}

\item[Update Now.]
  \setting{Update now} causes the database to detect new and deleted files
  \opt{SWCODEC}{
    \note{Unlike the \setting{Auto Update} function, \setting{Update Now}
      will update the database regardless of whether the \setting{Directory Cache}
      is enabled. Thus, an update using \setting{Update now} may take a long
      time.
    }
  }
  Unlike \setting{Initialize Now}, the \setting{Update Now} function
  does not remove runtime database information.
  
\item[Gather Runtime Data.]
  When enabled, rockbox will record how often and how long a track is being played, 
  when it was last played and its rating. This information can be displayed in
  the WPS and is used in the database browser to, for example, show the most played, 
  unplayed and most recently played tracks.
  
\item[Export modifications.]
  This allows for the runtime data to be exported to the file \\
  \fname{/.rockbox/database\_changelog.txt}, which backs up the runtime data in
  ASCII format. This is needed when database structures change, because new
  code cannot read old database code. But, all modifications
  exported to ASCII format should be readable by all database versions.
  
\item[Import modifications.]
  Allows the \fname{/.rockbox/database\_changelog.txt} backup to be 
  conveniently loaded into the database. If \setting{Auto Update} is
  enabled this is performed automatically when the database is initialized.
  
\end{description}

\subsection{Using the database}
Once the database has been initialized, you can browse your music by Artist, 
Album, Genre and Song Name. To use the database, go to the \setting{Main Menu}
and select \setting{Database}.\\

\note{You may need to increase the value of the \setting{Max files in dir 
browser} setting (\setting{Settings $\rightarrow$ General Settings
$\rightarrow$ System $\rightarrow$ Limits}) in order to view long lists of
tracks in the ID3 database browser.\\

There is no option to turn off database completely. If you do not want
to use it just do not do the initial build of the database and do not load it
to RAM.}
%
\begin{table}
\begin{center}
  \begin{tabularx}{.75\textwidth}{XX}%
  \toprule%
  \textbf{Tag}   & \textbf{Type}  & \textbf{Origin} \\
  \midrule
  filename              & string    & system \\ 
  album                 & string    & id tag \\
  albumartist           & string    & id tag \\
  artist                & string    & id tag \\
  comment               & string    & id tag \\
  composer              & string    & id tag \\
  genre                 & string    & id tag \\
  grouping              & string    & id tag \\
  title                 & string    & id tag \\
  bitrate               & numeric   & id tag \\
  discnum               & numeric   & id tag \\
  year                  & numeric   & id tag \\
  tracknum              & numeric   & id tag/filename \\
  autoscore             & numeric   & runtime db \\
  lastplayed            & numeric   & runtime db \\
  playcount             & numeric   & runtime db \\
  Pm (play time - min)  & numeric   & runtime db \\
  Ps (play time - sec)  & numeric   & runtime db \\
  rating                & numeric   & runtime db \\
  commitid              & numeric   & system \\
  entryage              & numeric   & system \\
  length                & numeric   & system \\
  Lm (track len - min)  & numeric   & system \\
  Ls (track len - sec)  & numeric   & system \\
  \bottomrule
  \end{tabularx}
\end{center}
\end{table}

% $Id$ %
\section{\label{ref:WPS}While Playing Screen}
The While Playing Screen (WPS) displays various pieces of information about the
currently playing audio file.
%
The appearance of the WPS can be configured using WPS configuration files.
The items shown depend on your configuration -- all items can be turned on
or off independently. Refer to \reference{ref:wps_tags} for details on how
to change the display of the WPS.
\begin{itemize}
\item Status bar: The Status bar shows Battery level, charger status,
  volume, play mode, repeat mode, shuffle mode\opt{rtc}{ and clock}.
  In contrast to all other items, the status bar is always at the top of
  the screen.
\item (Scrolling) path and filename of the current song.
\item The ID3 track name.
\item The ID3 album name.
\item The ID3 artist name.
\item Bit rate. VBR files display average bitrate and ``(avg)''
\item Elapsed and total time.
\item A slidebar progress meter representing where in the song you are.
\item Peak meter.
\end{itemize}
%

See \reference{ref:ConfiguringtheWPS} for details of customising
your WPS (While Playing Screen).


\subsection{\label{ref:WPS_Key_Controls}WPS Key Controls}

  \begin{btnmap}
      \ActionWpsVolUp{} / \ActionWpsVolDown
      \opt{HAVEREMOTEKEYMAP}{& \ActionRCWpsVolUp{} / \ActionRCWpsVolDown}
      & Volume up/down.\\
      %
      \ActionWpsSkipPrev
       \opt{HAVEREMOTEKEYMAP}{& \ActionRCWpsSkipPrev}
      & Go to beginning of track, or if pressed while in the
        first seconds of a track, go to the previous track.\\
      %
      \ActionWpsSeekBack
      \opt{HAVEREMOTEKEYMAP}{& \ActionRCWpsSeekBack}
      & Rewind in track.\\
      %
      \ActionWpsSkipNext
      \opt{HAVEREMOTEKEYMAP}{& \ActionRCWpsSkipNext}
      & Go to the next track.\\
      %
      \ActionWpsSeekFwd
      \opt{HAVEREMOTEKEYMAP}{& \ActionRCWpsSeekFwd}
      & Fast forward in track.\\
      %
      \ActionWpsPlay
      \opt{HAVEREMOTEKEYMAP}{& \ActionRCWpsPlay}
      & Toggle play/pause.\\
      %
      \ActionWpsStop
      \opt{HAVEREMOTEKEYMAP}{& \ActionRCWpsStop}
      & Stop playback.\\
      %
      \ActionWpsBrowse
      \opt{HAVEREMOTEKEYMAP}{& \ActionRCWpsBrowse}
      & Return to the \setting{File Browser} / \setting{Database}.\\
      %
      \ActionWpsContext
      \opt{HAVEREMOTEKEYMAP}{& \ActionRCWpsContext}
      & Enter \setting{WPS Context Menu}.\\
      %
      \ActionWpsMenu
      \opt{HAVEREMOTEKEYMAP}{& \ActionRCWpsMenu}
      & Enter \setting{Main Menu}%
      .\\%
      %
      \opt{quickscreen}{%
        \ActionWpsQuickScreen
        \opt{HAVEREMOTEKEYMAP}{& \ActionRCWpsQuickScreen}
          & Switch to the \setting{Quick Screen}
          (see \reference{ref:QuickScreen}). \\}%
      %
      % software hold targets
      \nopt{hold_button}{%
          \opt{SANSA_CLIP_PAD}{\ButtonHome+\ButtonSelect}
          \opt{SANSA_FUZEPLUS_PAD}{\ButtonPower}
          & Key lock (software hold switch) on/off.\\
      }%
      % We explicitly list all the appropriate targets here and do no condition
      % on the 'pitchscreen' feature since some players have the feature but do
      % not have the button to go from the WPS to the pitch screen.
      \opt{IRIVER_H100_PAD,IRIVER_H300_PAD,IRIVER_H10_PAD,MROBE100_PAD%
          ,GIGABEAT_PAD,GIGABEAT_S_PAD,SANSA_E200_PAD,SANSA_C200_PAD,SANSA_FUZEPLUS_PAD}{%
        \ActionWpsPitchScreen
        \opt{HAVEREMOTEKEYMAP}{& \ActionRCWpsPitchScreen}
          & Show \setting{Pitch Screen} (see \reference{sec:pitchscreen}).\\%
      }%
      \opt{GIGABEAT_PAD,GIGABEAT_S_PAD,SANSA_CLIP_PAD,MROBE100_PAD,PBELL_VIBE500_PAD%
          ,SAMSUNG_YH92X_PAD,SAMSUNG_YH820_PAD,XDUOO_X3_PAD}{%
        \ActionWpsPlaylist
        \opt{HAVEREMOTEKEYMAP}{&}
          & Show current \setting{Playlist}.\\%
      }%
      \opt{IRIVER_H100_PAD,IRIVER_H300_PAD,IRIVER_H10_PAD%
          ,SANSA_E200_PAD,SANSA_C200_PAD,SANSA_FUZEPLUS_PAD}{%
        \ActionWpsIdThreeScreen
          \opt{HAVEREMOTEKEYMAP}{& \ActionRCWpsIdThreeScreen}
          & Enter \setting{ID3 Viewer}.\\%
      }%
      \opt{hotkey}{%
        \ActionWpsHotkey \opt{HAVEREMOTEKEYMAP}{& }
        & Activate the \setting{Hotkey} function (see \reference{ref:Hotkeys}).\\
      }
      \opt{ab_repeat_buttons}{%
         \ActionWpsAbSetBNextDir{} or }%
         % not all targets have the above action defined but the one below works on all
      Short \ActionWpsSkipNext{} + Long \ActionWpsSkipNext
      \opt{HAVEREMOTEKEYMAP}{
        &
          \opt{IRIVER_RC_H100_PAD}{\ActionRCWpsAbSetBNextDir{} or}
        Short \ActionRCWpsSkipNext{} + Long \ActionRCWpsSkipNext}
      & Skip to the next directory.\\
      %
      \opt{ab_repeat_buttons}{%
         \ActionWpsAbSetAPrevDir{} or }%
      Short \ActionWpsSkipPrev{} + Long \ActionWpsSkipPrev
      \opt{HAVEREMOTEKEYMAP}{
        &
          \opt{IRIVER_RC_H100_PAD}{\ActionRCWpsAbSetAPrevDir{} or}
        Short \ActionRCWpsSkipPrev{} + Long \ActionRCWpsSkipPrev}
      & Skip to the previous directory.\\
      %
      \opt{SANSA_E200_PAD,SANSA_C200_PAD,IRIVER_H100_PAD,IRIVER_H300_PAD}{
        \ActionStdRec
          \opt{HAVEREMOTEKEYMAP}{&}
          & Switch to the \setting{Recording Screen}.\\
      }%
  \end{btnmap}


\subsection{\label{ref:peak_meter}Peak Meter}
The peak meter can be displayed on the While Playing Screen and consists of
several indicators.
\opt{recording}{
  For a picture of the peak meter, please see the While
  Recording Screen in \reference{ref:while_recording_screen}.
}
\opt{ipodvideo}{
  \note{Especially the \playerman{} \playertype{}'s performance and battery runtime
   suffers when this feature is enabled. For this \dap{} it is highly recommended
   to not use peak meter.}
}

\begin{description}
\item [The bar:]
  This is the wide horizontal bar. It represents the current volume value.
\item [The peak indicator:]
  This is a little vertical line at the right end of the bar. It indicates
  the peak volume value that occurred recently.
\item [The clip indicator:]
  This is a little black block that is displayed at the very right of the
  scale when an overflow occurs. It usually does not show up during normal
  playback unless you play an audio file that is distorted heavily.
  \opt{recording}{
    If you encounter clipping while recording, your recording will sound distorted.
    You should lower the gain.
  }
  \note{Note that the clip detection is not very precise.
   Clipping might occur without being indicated.}
\item [The scale:]
  Between the indicators of the right and left channel there are little dots.
  These dots represent important volume values. In linear mode each dot is a
  10\% mark. In dBFS mode the dots represent the following values (from right
  to left): 0~dB, {}-3~dB, {}-6~dB, {}-9~dB, {}-12~dB, {}-18~dB, {}-24~dB, {}-30~dB,
  {}-40~dB, {}-50~dB, {}-60~dB.
\end{description}

\subsection{\label{sec:contextmenu}The WPS Context Menu}
Like the context menu for the \setting{File Browser}, the \setting{WPS Context Menu}
allows you quick access to some often used functions.

\subsubsection{Playlist}
The \setting{Playlist} submenu allows you to view, save, search, reshuffle,
and display the play time of the current playlist. These and other operations
are detailed in \reference{ref:working_with_playlists}. To change settings for
the \setting{Playlist Viewer} press \ActionStdContext{} while viewing the
current playlist to bring up the \setting{Playlist Viewer Menu}. In this
menu, you can find the \setting{Playlist Viewer Settings}.

\paragraph{Playlist Viewer Settings}
  \begin{description}
    \item[Show Icons.] This toggles display of the icon for the currently
    selected playlist entry and the icon for moving a playlist entry
    \item[Show Indices.] This toggles display of the line numbering for
       the playlist
    \item[Track Display.] This toggles between filename only and full path
       for playlist entries
  \end{description}


\subsubsection{Playlist catalogue}
  \begin{description}
    \item [View catalogue.] This lists all playlists that are part of the
    Playlist catalogue. You can load a new playlist directly from this list.
    \item [Add to playlist.] Adds the currently playing file to a playlist.
    Select the playlist you want the file to be added to and it will get
    appended to that playlist.
    \item [Add to new playlist.] Similar to the previous entry this will
    add the currently playing track to a playlist. You need to enter a name
    for the new playlist first.
  \end{description}

\subsubsection{Sound Settings}
This is a shortcut to the \setting{Sound Settings Menu}, where you can configure volume,
bass, treble, and other settings affecting the sound of your music.
See \reference{ref:configure_rockbox_sound} for more information.

\subsubsection{Playback Settings}
This is a shortcut to the \setting{Playback Settings Menu}, where you can configure shuffle,
repeat, party mode, skip length and other settings affecting the playback of your music.

\subsubsection{Rating}
The menu entry is only shown if \setting{Gather Runtime Information} is
enabled. It allows the assignment of a personal rating value (0 -- 10)
to a track which can be displayed in the WPS and used in the Database
browser. The value wraps at 10.

\subsubsection{Bookmarks}
This allows you to create a bookmark in the currently-playing track.

\subsubsection{\label{ref:trackinfoviewer}Show Track Info}
\screenshot{rockbox_interface/images/ss-id3-viewer}{The track info viewer}{}
This screen is accessible from the WPS screen, and provides a detailed view of
all the identity information about the current track. This info is known as
meta data and is stored in audio file formats to keep information on artist,
album etc. To access this screen, %
\opt{IRIVER_H100_PAD,IRIVER_H300_PAD,IRIVER_H10_PAD,%
      SANSA_C200_PAD,SANSA_E200_PAD,SANSA_FUZE_PAD,SANSA_FUZEPLUS_PAD}{
  press \ActionWpsIdThreeScreen. }%
\opt{IPOD_4G_PAD,IPOD_3G_PAD,IAUDIO_X5_PAD,IAUDIO_M3_PAD,%
      GIGABEAT_PAD,GIGABEAT_S_PAD,MROBE100_PAD,SANSA_CLIP_PAD,PBELL_VIBE500_PAD,%
      MPIO_HD200_PAD,MPIO_HD300_PAD,SAMSUNG_YH92X_PAD,SAMSUNG_YH820_PAD,XDUOO_X3_PAD}%
      {press \ActionWpsContext{} to access the
      \setting{WPS Context Menu} and select \setting{Show Track Info}. }

\subsubsection{Open With...}
This \setting{Open With} function is the same as the \setting{Open With}
function in the file browser's \setting{Context Menu}.

\subsubsection{Delete}
Delete the currently playing file. The file will be deleted but the playback
of the file will not stop immediately. Instead, the part of the file that
has already been buffered (i.e. read into the \daps\ memory) will be played.
This may even be the whole track.

\opt{pitchscreen}{
  \subsubsection{\label{sec:pitchscreen}Pitch}

  The \setting{Pitch Screen} allows you to change the rate of playback
  (i.e. the playback speed and at the same time the pitch) of your
  \dap.  The rate value can be adjusted
  between 50\% and 200\%. 50\% means half the normal playback speed and a
  pitch that is an octave lower than the normal pitch. 200\% means double
  playback speed and a pitch that is an octave higher than the normal pitch.

  The rate can be changed in two modes: procentual and semitone.
  Initially, procentual mode is active.

    If you've enabled the \setting{Timestretch} option in
    \setting{Sound Settings} and have since rebooted, you can also use
    timestretch mode. This allows you to change the playback speed
    without affecting the pitch, and vice versa.

    In timestretch mode there are separate displays for pitch and
    speed, and each can be altered independently.  Due to the
    limitations of the algorithm, speed is limited to be between 35\%
    and 250\% of the current pitch value.  Pitch must maintain the
    same ratio as well as remain between 50\% and 200\%.

  The value of the rate, pitch and speed
  is not persistent, i.e. after the \dap\ is turned on it will
  always be set to 100\%.  However, the rate, pitch and speed
  information will be stored in any bookmarks you may create
  (see \reference{ref:Bookmarkconfigactual}) and will be restored upon
  playing back those bookmarks.

  \begin{btnmap}
    \ActionPsToggleMode
    \opt{HAVEREMOTEKEYMAP}{& \ActionRCPsToggleMode}
    & Toggle pitch changing mode (cycle through all available modes).\\
    %
    \ActionPsIncSmall{} / \ActionPsDecSmall
    \opt{HAVEREMOTEKEYMAP}{& \ActionRCPsIncSmall{} / \ActionRCPsDecSmall}
    & Increase~/ Decrease pitch by 0.1\% (in procentual mode) or 0.1
      semitone (in semitone mode).\\
    %
    \nopt{PBELL_VIBE500_PAD}{ % there is no long scroll up or down because of slide
    \ActionPsIncBig{} / \ActionPsDecBig
    \opt{HAVEREMOTEKEYMAP}{& \ActionRCPsIncBig{} / \ActionRCPsDecBig}
    & Increase~/ Decrease pitch by 1\% (in procentual mode) or a semitone
      (in semitone mode).\\
    }
    %
    \ActionPsNudgeLeft{} / \ActionPsNudgeRight
    \opt{HAVEREMOTEKEYMAP}{& \ActionRCPsNudgeLeft{} / \ActionRCPsNudgeRight}
    & Temporarily change pitch by 2\% (beatmatch), or modify speed (in timestretch mode).\\
    %
    \ActionPsReset
    \opt{HAVEREMOTEKEYMAP}{& \ActionRCPsReset}
    & Reset pitch and speed to 100\%. \\
    %
    \ActionPsExit
    \opt{HAVEREMOTEKEYMAP}{& \ActionRCPsExit}
    & Leave the \setting{Pitch Screen}. \\
    %
  \end{btnmap}

}


%Include playlist section
\chapter{Getting started}
\section{Welcome}
This is the manual for Rockbox.  Rockbox is a replacement firmware for the
Jukebox Studio, Recorder and Ondio players made by Archos, the H120/140
players from iRiver and the Apple iPod Nano etc.  It is a complete rewrite of
the software used to make the DAP play and record music, and contains many
features and enhancements not available in the original firmware supplied by
the manufacturer.  Among the things that Rockbox has to offer are the
following:

\begin{itemize}
\item Faster loading than the \playername\ firmware
\item Uninterrupted playing of MP3 files {--} skipping is very rare
\item More control over how your music is played
\item Built in viewers for several common file types
\item Sophisticated plugin system that allows the Jukebox to run games,
a calendar, a clock, and many other applications.
\item Totally removable. (Removal of Rockbox before returning the
Jukebox for repair under warranty is advised.)
\item Optional voice user interface for complete control without looking
at the screen
\end{itemize}
Rockbox is a complete from scratch rewrite of the \playername\ software and
uses no fragments of the original firmware.  Not only is it free to
use, it's also released under the GNU public license,
which means that it will always remain free to both use and to change.

\opt{Ondio}{Although Rockbox also runs on the Archos Ondio series of
flash based MP3 players,  this is a recent development, which is not covered
fully in this manual.  Most of this manual will, however, apply equally to
Rockbox on the Ondio Jukeboxes.  For more details on the Ondio port, please
see the web page:
\url{http://www.rockbox.org/twiki/bin/view/Main/ArchosOndio}.}

\section{Getting more help}

This manual is intended to be a comprehensive introduction to the Rockbox
software.  There is, however, more help available.  The Rockbox website at
\url{http://www.rockbox.org/}contains very extensive documentation and guides
written by members of the Rockbox community and this should be your first port
of call when looking for further help.

\opt{player,recorder,recorderv2fm,ondio}{\section{Before installation}

Before you install Rockbox, you will need to know what model of Archos Jukebox
you own.  Rockbox comes in different versions depending on the model of your
Jukebox.  There are six different versions of the software.  The table below
will help you to identify which version of the software you need.

The model name is printed on the case.  The hard drive size is listed on the
serial number sticker on the back of the unit.

\begin{center}
  \begin{tabularx}{\textwidth}{llXl}\toprule
    \label{ref:Jukeboxtypetable}
    \textbf{Picture} & \textbf{Disk size} & \textbf{Model Name} & \textbf{Version Name} \\\midrule
    \begin{minipage}{2.2cm}
      \includegraphics[width=2cm]{getting_started/images/archos-studio-small.png}
    \end{minipage} 
    & 5GB, 6GB, 10GB, 20GB & 
                             \begin{minipage}{8cm}
                             Jukebox 5000 \newline
                             Jukebox 6000 \newline
                             Jukebox Studio 10 \newline
                             Jukebox Studio 20
                             \end{minipage}
                               & player \\\midrule
    \begin{minipage}{2.2cm}
      \includegraphics[width=2cm]{getting_started/images/archos-recorder-small.png}
    \end{minipage}
    & 6GB, 10GB, 15GB, 20GB & \begin{minipage}{8cm}
                              Jukebox Recorder 6 \newline
                              Jukebox Recorder 10 \newline
                              Jukebox Recorder 15 \newline
                              Jukebox Recorder 20 
                              \end{minipage}
                               & recorder\\\midrule
    \begin{minipage}{2.2cm}
      \includegraphics[width=2cm]{getting_started/images/archos-recorderv2-small.png}
    \end{minipage}
                     & 20GB & Jukebox Recorder v2 & recorderv2\\\midrule
    \begin{minipage}{2.2cm}
      \includegraphics[width=2cm]{getting_started/images/archos-recorderfm-small.png}
    \end{minipage}
                     & 20GB & Jukebox Recorder FM & fmrecorder \\\midrule
    \begin{minipage}{2.2cm}
      \includegraphics[width=2cm]{getting_started/images/archos-ondiosp-small.png}
    \end{minipage}
                     & 128MB (flash) & Ondio 128 SP & ondiosp \\\midrule
    \begin{minipage}{2.2cm}
      \includegraphics[width=2cm]{getting_started/images/archos-ondiofm-small.png}
    \end{minipage}
                     & 128MB (flash) & Ondio 128 FM & ondiofm \\\bottomrule
  \end{tabularx}
\end{center}
\note{Rockbox does not run on the Archos Jukebox Multimedia or any
Archos MP3 player products other than those mentioned here.}

}


\section{Downloading Rockbox}

The latest release of the Rockbox software will always be available from
\url{http://www.rockbox.org/download/}.
 Windows users may wish to download the self{}-extracting Windows
installer, which works for all Jukebox models, but those wishing to
install manually or using a different operating system should choose
the .zip archive containing the firmware for their model of the
Jukebox.

\section{Installing Rockbox}

Using the Windows self installing executable to install Rockbox is the easiest
method of installing the software on your Jukebox.  Simply follow the
on{}-screen instructions and select the appropriate drive letter and Jukebox
model when prompted.  You can use ``Add / Remove Programs'' to uninstall the
software at a later date.

For non{}-Windows users and those wishing to install manually from the archive
the procedure is still fairly simple.  Connect your \playername\ to the
computer via USB as described in the manual that came with your \playername. On
Windows, the \playername\ drive will appear as a drive letter in your
``My Computer'' folder. Take the file that you downloaded above, and unpack
its contents to your \playername\ drive. You can do this using a program such
as \url{http://www.info-zip.org/} or \url{http://www.winzip.org/}. 

You will need to unpack all of the files in the archive onto your hard
disk. If this has been done correctly, you will have a file called
\opt{h1xx,h300}{\fname{rockbox.iriver}}
\opt{ipodcolor,ipodnano}{\fname{rockbox.ipod?}}
\opt{ondio}{\fname{rockbox.ondio?}}
\opt{player}{\fname{archos.mod}}
\opt{recorder,recorderv2fm}{\fname{ajbrec.ajz}}
in the main folder of your \playername\ drive, and also a folder called
/\fname{.rockbox}, which contains a number of system files used by the
software.

\section{Enabling Speech Support (optional)}

If you wish to use speech support you will also need a language file,
available from
\url{http://www.rockbox.org/twiki/bin/view/Main/VoiceFiles/}.
 For the English language, the file is called \fname{english.voice}.
When it has been downloaded, unpack this file and copy it into the
\fname{lang} folder which is inside the /\fname{.rockbox} folder on
your Jukebox. Voice menus are turned on by default.  See page
\pageref{ref:Voiceconfiguration} for details on voice settings.


\section{Running Rockbox}

Remove your Jukebox from the computer's USB port.
Unplug any connected power supply and turn the unit off. When you next
turn the unit on, the Jukebox firmware will start to load, and then it
will load Rockbox for you. When you see the Rockbox splash screen,
Rockbox is loaded and ready for use. 

\section{Uninstalling Rockbox}

If you would like to go back to using the original \playername\ software, then
connect the \playername\ to your computer, and delete the
\opt{h1xx,h300}{\fname{rockbox.iriver}}
\opt{ipodcolor,ipodnano}{\fname{rockbox.ipod?}}
\opt{ondio}{\fname{rockbox.ondio?}}
\opt{player}{\fname{archos.mod}}
\opt{recorder,recorderv2fm}{\fname{ajbrec.ajz}} file.
If you wish to clean up your disk, you may also wish to delete the
\fname{.rockbox} folder and its contents. Turn the \playername\ off and on and
the normal \playername\ software will load.

