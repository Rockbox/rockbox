% $Id$ %
\section{\label{ref:WPS}While Playing Screen}
The While Playing Screen (WPS) displays various pieces of information about the
currently playing audio file.
%
\opt{HAVE_LCD_BITMAP}{%
  The apperance of the WPS can be configured using wps configuration files.
  The items shown depend on your configuration -- all item can be turned on
  or off independently. Refer to \pageref{ref:wps_tags} for details on how
  to change the display of the WPS.
  \begin{itemize}
  \item Status bar: The Status bar shows Battery level, charger status, 
    volume, play mode, repeat mode, shuffle mode\opt{CONFIG_RTC}{ and clock}.
    In contrast to all other item the status bar is always at the top of
    the screen.
  \item (Scrolling) path and filename of the current song.
  \item The ID3 track name.
  \item The ID3 album name.
  \item The ID3 artist name.
  \item Bit rate. VBR files display average bitrate and ``(avg)''
  \item Elapsed and total time.
  \item A slidebar progress meter representing where in the song you are.
  \item Peak meter.
  \end{itemize}
}
\opt{recorder,recorderv2fm,ondio}{
  \note{
  \begin{itemize}
  \item The number of lines shown depends on the size of the font used.
  \item The peak meter is only visible if you turn off the status bar or if
    using a small font that gives 8 or more display lines.
  \end{itemize}
  }
}
%
\opt{player}{
  \note{
  \begin{itemize}
  \item Playlist index/Playlist size: Artist {}- Title.
  \item Current{}-time Progress{}-indicator Left.
  \end{itemize}
  }
}

See page \pageref{ref:ConfiguringtheWPS} for details of customising
your WPS (While Playing Screen).


\subsection{\label{ref:WPS_Key_Controls}WPS Key Controls}

\begin{table}
  \begin{btnmap}{}{}
      \opt{IRIVER_H100_PAD,IRIVER_H300_PAD,RECORDER_PAD,ONDIO_PAD}{\ButtonUp/\ButtonDown}
      \opt{IAUDIO_X5_PAD}{Please add correct keys}
      \opt{IPOD_4G_PAD}{Please add correct keys}
      \opt{PLAYER_PAD}{\ButtonMenu+\ButtonRight/\ButtonLeft} 
      & Volume up/down \\
      %
      \opt{IRIVER_H100_PAD,IRIVER_H300_PAD,IAUDIO_X5_PAD,ONDIO_PAD,RECORDER_PAD,PLAYER_PAD}{\ButtonLeft}
      \opt{IPOD_4G_PAD}{Please add correct keys} 
      & Go to beginning of track, or if pressed while in the first seconds of a track,
        go to previous track. \\
      %
      \opt{IRIVER_H100_PAD,IRIVER_H300_PAD,IAUDIO_X5_PAD,ONDIO_PAD,RECORDER_PAD,PLAYER_PAD}{Hold \ButtonLeft}
      \opt{IPOD_4G_PAD}{Please add correct keys} 
      & Rewind in track \\
      %
      \opt{IRIVER_H100_PAD,IRIVER_H300_PAD,IAUDIO_X5_PAD,ONDIO_PAD,RECORDER_PAD,PLAYER_PAD}{\ButtonRight}
      \opt{IPOD_4G_PAD}{Please add correct keys} 
      & Go to next track. \\
      %
      \opt{IRIVER_H100_PAD,IRIVER_H300_PAD,IAUDIO_X5_PAD,ONDIO_PAD,RECORDER_PAD,PLAYER_PAD}{Hold \ButtonRight}
      \opt{IPOD_4G_PAD}{Please add correct keys} 
      & Fast forward in track. \\
      %
      \opt{IRIVER_H100_PAD,IRIVER_H300_PAD}{\ButtonOn}
      \opt{IPOD_4G_PAD,IAUDIO_X5_PAD,RECORDER_PAD,PLAYER_PAD}{\ButtonPlay}
      \opt{ONDIO_PAD}{\ButtonOff} 
      & Toggle play/pause \\
      %
      \opt{IRIVER_H100_PAD,IRIVER_H300_PAD}{\ButtonSelect}
      \opt{ONDIO_PAD}{\ButtonMenu}
      \opt{RECORDER_PAD,PLAYER_PAD}{\ButtonOn}
      \opt{IAUDIO_X5_PAD}{Please add correct keys}
      \opt{IPOD_4G_PAD}{Please add correct keys} 
      & Go to file browser \\
      %
      \opt{IRIVER_H100_PAD,IRIVER_H300_PAD}{Hold \ButtonSelect}
      \opt{ONDIO_PAD}{Hold \ButtonMenu}
      \opt{RECORDER_PAD,PLAYER_PAD}{Hold \ButtonOn}
      \opt{IAUDIO_X5_PAD}{Please add correct keys}
      \opt{IPOD_4G_PAD}{Please add correct keys} 
      & Enter WPS context menu \\
      %
      \opt{RECORDER_PAD,IRIVER_H100_PAD,IRIVER_H300_PAD}{Hold \ButtonOn}
      \opt{IAUDIO_X5_PAD}{Please add correct keys}
      \opt{IPOD_4G_PAD}{Please add correct keys}
      \opt{PLAYER_PAD}{n/a}
      \opt{ONDIO_PAD}{In WPS context menu}
      & Show pitch setting screen \\
      %
      \opt{RECORDER_PAD,IRIVER_H100_PAD,IRIVER_H300_PAD}{\ButtonOff}
      \opt{ONDIO_PAD}{Hold \ButtonOff}
      \opt{IAUDIO_X5_PAD}{Please add correct keys}
      \opt{IPOD_4G_PAD}{Please add correct keys}
      \opt{PLAYER_PAD}{\ButtonStop}
      & Stop playback \\
      %
      \opt{IRIVER_H100_PAD,IRIVER_H300_PAD}{\ButtonSelect}
      \opt{RECORDER_PAD}{\ButtonFOne}
      \opt{PLAYER_PAD,ONDIO_PAD}{\ButtonMenu}
      \opt{IAUDIO_X5_PAD}{Please add correct keys}
      \opt{IPOD_4G_PAD}{Please add correct keys} 
      & Go to Main menu \\
      %
      %These actions need definitions for the other targets
      \opt{RECORDER_PAD}{
        \ButtonFTwo & Toggles Play/browse quick menu \\
        \ButtonFThree & Toggles Display quick menu \\
        \ButtonFOne+\ButtonDown & Key lock on/off \\
        \ButtonFOne+\ButtonPlay & Mute on/off \\
        \ButtonFOne+\ButtonOn & Enter ID3 viewer \\
      }
      \opt{PLAYER_PAD}{
        \ButtonMenu+\ButtonStop & Key lock on/off \\
        \ButtonMenu+\ButtonPlay & Mute on/off \\
        \ButtonMenu+\ButtonOn & Enter ID3 viewer \\
      }
    \end{btnmap}
\end{table}


\opt{HAVE_LCD_BITMAP}{
  \subsection{\label{ref:peak_meter}Peak Meter}
  The peak meter can be displayed on the While Playing Screen and consists of
  several indicators.  For a picture of the peak meter, please see the  While
  Recording Screen on page \pageref{ref:Whilerecordingscreen}.
  
  \begin{description}
  \item [The bar:]
    This is the wide horizontal bar. It represents the current volume value.
  \item [The peak indicator:]
    This is a little vertical line at the rightend of the bar. It indicates the
    peak volume value that occurred recently.
  \item [The clip indicator:]
    This is a little black block that is displayed at the very right of the
    scale when an overflow occurs. It usually doesn't show up during normal
    playback unless you play an audio file that is distorted heavily. If you
    encounter clipping while recording your recording will sound distorted. You
    should lower the gain. Note that the clip detection is not very precise.
    Clipping might occur without being indicated.
  \item [The scale:]
    Between the indicators of the right and left channel there are little dots.
    These dots represent important volume values. In linear mode each dot is a
    10\% mark. In dbfs mode the dots represent the following values (from right
    to left): 0db, {}-3db, {}-6db, {}-9db, {}-12db, {}-18db, {}-24db, {}-30db,
    {}-40db, {}-50db, {}-60db.
  \end{description}
}
\subsection{\label{sec:quickscreens}Quickscreens}

\subsubsection{\label{sec:contextmenu}Context Menu}
Like the context menu for the file browser this context menu allows you
to quickly access some often used functions.
\opt{SWCODEC}{These include the ID3 Viewer and the pitch screen.}

\opt{SWCODEC}{
  \subsubsection{\label{sec:pitchscreen}Pitch Screen}
  The Pitch Screen allows you to quickly change the pitch of your \dap. The
  pitch value can be adjusted between 50\% and 200\%.
  \begin{table}
  \begin{btnmap}{}{}
      \opt{h1xx,h300,RECORDER_PAD,ONDIO_PAD,PLAYER_PAD}{\ButtonUp/\ButtonDown}
      \opt{IPOD_4G_PAD,x5}{\fixme{tbd}}
      & Increase / Decrease Pitch by 0.1\% \\
      %
      \opt{h1xx,h300,RECORDER_PAD,ONDIO_PAD,PLAYER_PAD}{\ButtonLeft/\ButtonRight}
      \opt{IPOD_4G_PAD,x5}{\fixme{tbd}}
      & Increase / Decrease Pitch by 2.0\% \\
      %
      \opt{RECORDER_PAD,ONDIO_PAD,PLAYER_PAD}{\fixme{\ButtonOn+\ButtonStop}}
      \opt{h1xx,h300}{\ButtonOn+\ButtonOff}
      \opt{IPOD_4G_PAD,x5}{\fixme{tbd}}
      & Reset Pitch to 100\% \\
      %
      \opt{RECORDER_PAD,ONDIO_PAD,PLAYER_PAD}{\ButtonStop}
      \opt{h1xx,h300}{\ButtonOff}
      \opt{IPOD_4G_PAD,x5}{\fixme{tbd}}
      & Leave Pitch screen \\
      %
  \end{btnmap}
  \end{table}
}

\subsubsection{\label{ref:ID3viewer}ID3 Viewer}
\screenshot{rockbox_interface/images/ss-id3-viewer}{The ID3 viewer}{}
This screen is accessible from the WPS screen, and provides a detailed view of
all the identity information about the current track. This info is known as
meta data and is stored in audio file formats to keep information on artist,
album etc. To access this screen, \opt{h1xx,h300}{hold \ButtonSelect\ to 
access the WPS context menu, and select the id3-viewer from there.}
\opt{RECORDER_PAD}{press \ButtonFOne+\ButtonOn}
\opt{PLAYER_PAD}{press \ButtonMenu+\ButtonOn.}
\opt{IAUDIO_X5_PAD,IPOD_4G_PAD}{\fixme{Please correct this information}}
\opt{ONDIO_PAD}{hold \ButtonMenu to access the WPS context menu, and 
  select the id3-viewer from there.}
\opt{RECORDER_PAD,PLAYER_PAD,ONDIO_PAD}{Use \ButtonLeft\ and \ButtonRight\
  to move through the information.}

%********************QUICKSCREENS***********************************************
\opt{RECORDER_PAD}{
  \section{\label{ref:QuickScreenMenus}Quick Screen Menus}
  \screenshot{rockbox_interface/images/ss-quick-screen-112x64x1.png}{The quick screen}{}
  \screenshot{rockbox_interface/images/ss-quick-screen2-112x64x1.png}{The quick screen}{}
  Rockbox handles function buttons in a different way to the Archos software.
  \ButtonFOne\ is always bound to the menu function, while \ButtonFTwo\ and
  \ButtonFThree\ enable two quick menus.
  
  \ButtonFTwo\ displays some browse and play settings which are likely to be
  changed frequently. This settings are Shuffle mode, Repeat mode and the Show
  files options
  
  Shuffle mode plays each track in the currently playing list in a random order
  rather than in the order shown in the browser.

  Repeat mode repeats either a single track (One) or the entire playlist (All).

  Show files determines what type files can be seen in the browser.  This can be
  just MP3 files and directories (Music), Playlists, MP3 files and directories
  (Playlists), any files that Rockbox supports (Supported) or all files on the
  disk (All).

  See page \pageref{ref:PlaybackOptions} for more information about these
  settings.

  \begin{table}
      \begin{btnmap}{}{}
        \ButtonLeft & Controls Shuffle mode setting \\
        \ButtonRight & Controls Repeat mode setting \\
        \ButtonDown & Controls Show file setting \\
      \end{btnmap}
  \end{table}
  
  \ButtonFThree\ controls frequently used display options.
  
  Scroll bar turns the display of the Scroll bar on the left of the screen on
  or off.
  
  Status bar turns the status display at the top of the screen on or off. Upside
  down inverts the screen so that the top of the display appears nearest to the
  buttons.  This is sometimes useful when storing the \dap\ in a pocket.  Key
  assignments swap over with the display orientation where it is logical for
  them to do so.

  See page \pageref{ref:Displayoptions} for more information about these
  settings.
  
  \begin{table}
    \begin{btnmap}{}{}
      \ButtonLeft & Controls scroll bar display \\
      \ButtonRight & Controls status bar display \\
      \ButtonDown & Controls upside down screen setting \\
    \end{btnmap}
  \end{table}
}
