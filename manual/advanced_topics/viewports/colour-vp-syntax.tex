\subsubsection{Viewport Declaration Syntax}

{\config{\%V}}{\textbar}x{\textbar}y{\textbar}[width]{\textbar}[height]{\textbar}[font]{\textbar}[fgcolour]{\textbar}[bgcolour]{\textbar}%

    \begin{itemize}
      \item `fgcolour' and `bgcolour' are 6-digit RGB888 colours, e.g. FF00FF.
      \item `font' is a number: 0 is the built-in system font, 1 is the
      user-selected font.
      \item Only the coordinates \emph{have} to be specified. Leaving the other
      definitions blank will set them to their default values.
      \note{The correct number of `{\textbar}'s (vertical bars) with hyphens in
      blank fields are still needed in any case.}
    \end{itemize}

\begin{example}
    %V|12|20|-|-|1|-|-|
    %sThis viewport is displayed permanently. It starts 12px from the left and
    %s20px from the top of the screen, and fills the rest of the screen from
    %sthat point. The lines will scroll if this text does not fit in the viewport.
    %sThe user font is used, as are the default foreground/background colours.
\end{example}
\begin{rbtabular}{.75\textwidth}{XX}{\textbf{Viewport definition} & \textbf{Default value}}{}{}
  width/height & remaining part of screen \\
  font & user defined \\
  foreground/background colours & defined by theme \\
\end{rbtabular}

