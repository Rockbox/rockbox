\subsection{Battery Benchmark}
The Battery Benchmark Plugin enables you to test your battery's performance,
while making normal use of your \dap. Once loaded it will run in the
background (TSR plugin), reading various info about your battery while you use
it. Once you finish your session you can find the benchmark output data in a
file on your \dap\ \fname{/battery\_bench.txt}.
Please submit your results to the Rockbox wiki.
\url{http://www.rockbox.org/twiki/bin/view/Main/BatteryRuntime}

\subsubsection{How it works}
After you load the plug-in the operation of your \dap\ continues as normal.
You can do whatever you could do before loading the plugin except loading
another plugin. If you happen to load a plugin while benchmarking, a splash
screen will inform you about the termination of the benchmark.
While you operate it will log various battery related information every time
the disk is activated by external causes, (buffer refill, open folder,
USB mode) or an hour passes without updating the log file.\\
The plugin will continue to log info until:

\begin{itemize}
\item Another plugin is loaded.
\item The \dap\ is shut down.
\item The battery is empty.
\end{itemize}
Between disk activity (or an hour), it will log info in memory (every
measurement is captured when the voltage changes). If there are too many
measurements older entries will be deleted and the log file will inform the
user about the interval where entries were lost. Benchmarks can be resumed if
you accidentally load a plugin, or turn off your \dap, as long as the log
file \fname{/battery\_bench.txt} is not deleted. 

\subsubsection{Information explained}
On the top of the file you will see various info on how to use the plugin.
\begin{description}
\item[Time] It is the total time of operation of the \dap. It is not the time
that you started the plug-in. If you have your player on for 5 minutes and then
startt the plugin, it will start measuring from 5 minutes.
\item[Seconds] As time, it shows time passed in seconds. Nothing special, it is
there because it is free and maybe someone might want to make graphs with
seconds.
\item[Level] The percent level of the battery estimated by Rockbox. This is an
estimation and not an accurate result. Using the real percentage (current
battery voltage / top battery voltage) * 100) we can calculate the difference
between the estimation. Goal of this column is to make the estimation algorithm
of Rockbox more accurate.
\item[Time Left] It shows the estimated (by Rockbox) remaining time until
shutdown. Again, as Level, this column can be used to see differences between
real time left and estimated time left. This could help make time left more
accurate.
\item[Voltage] The current, battery voltage, the moment the measurement was
captured. Measurements are captured when this number changes while benchmarking.
This column can be used to give quite interesting graphs in a spreadsheet
program. (Excel, Calc, e.t.c)
\item[M/DA] (Measurements per Disk Activity) The number of measurements stored
temporarily in memory, before written on the log file. This can give you an
idea on how many voltage changes are between disk activity (or one hour).
\item[C] Stands for Charger. An "A" in that column shows if there was the power
adapter attached to the unit, at the time of the measurement.
\item[S] The "S" column shows the state of the device (Charging, or not). The
"C" indicated that the unit was charging when the measurement was captured.
\item[U] USB powered. Only for targets that support this. A "U" will indicate
if the unit was using the USB port for powering. 
\end{description}

\subsubsection{Making graphs}
While you can tell how long your battery lasted, with a single look at the last
line of the battery log (\fname{/battery\_bench.txt}), the most useful purpose of
Battery Benchmark is to make graphs using a spreadsheet program like Excel or
Calc. The battery log (\fname{/battery\_bench.txt}) is in CSV format (comma separated)
so you can quite easily import it to a spreadsheet program.
