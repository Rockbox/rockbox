\subsection{Clock}
{\centering\itshape
  [Warning: Image ignored] % Unhandled or unsupported graphics:
%\includegraphics[width=3.528cm,height=2.016cm]{images/rockbox-manual-img63.png}
 \newline
Clock
\par}

This is a fully featured analogue and digital clock program.  

\subsubsection{Key configuration}

\begin{center}\begin{tabular}{|p{2.411cm}|p{6.012cm}|}
\hline
{\centering\bfseries\itshape
KEY
\par}
&
{\centering\bfseries\itshape
ACTION
\par}
\\\hline
{\centering
F1
\par}
&
Help
\\\hline
{\centering
F2
\par}
&
Start / Stop stopwatch
\\\hline
{\centering
F2 (Hold)
\par}
&
Reset stopwatch
\\\hline
{\centering
F3
\par}
&
Options
\\\hline
{\centering
Play
\par}
&
Select clock mode
\\\hline
{\centering
UP
\par}
&
Enable idle power off
\\\hline
{\centering
DOWN
\par}
&
Disable idle power off
\\\hline
{\centering
RIGHT
\par}
&
Enable backlight
\\\hline
{\centering
LEFT
\par}
&
Disable backlight
\\\hline
{\centering
OFF
\par}
&
Save settings to disk and exit
\\\hline
\end{tabular}\end{center}

\subsubsection{Backlight configuration}
If RIGHT or LEFT is not pressed during clock operation (with the
exception of at the Help/Options/Mode Selector/Credit screens) then the
backlight timeout will remain your Rockbox default setting (example, 15
seconds). If RIGHT or LEFT is pressed, Clock will set the backlight to
ON or OFF, respectively. When Clock is exited, your default Rockbox
setting for Backlight will be restored. 

\subsubsection{Saving Settings}
Settings are saved to disk when Clock is exited. They are saved to
\textbf{/.rockbox/rocks/.clock\_settings''}. To reset your settings
back to the defaults, simply navigate to this file using Rockbox,
highlight it, and press the ON+PLAY keys to get the Delete option. This way you can feel free to experiment with the settings {}- and you could even load
separate settings, say, one for your desk at home and one for in the car {}- by keeping two files in your \textbf{/.rockbox/rocks} folder such as
``h.clock\_settings'' and ``c.clock\_settings''. Simply remove the
``h'' for your home settings to go into effect, or add the ``h'' back and take off the ``c'' for your car settings.

In the future, loading different settings will probably be made easier
through a built{}-in settings file loader in Clock. 


