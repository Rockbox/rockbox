\subsection{Split Editor}
When recording an mp3 file, it is common practice to start the recording
a little bit early and stop it a little bit late to ensure all the
desired sound is recorded. This results in recordings that contain
extra snippets of sound and the beginning and end. Unfortunately these
snippets can not be deleted easily because they are stored in the same
file as the desired recording. The purpose of the split editor is to
split a mp3 file (the input file) at a point in time (split point). Two
new files can be generated from the input file. The first file contains
the part before the split point and the second file contains the part
after the split point. Once this process has been successful the
original file can be deleted or kept as a backup. 

The whole process of splitting a mp3 file consists of three steps: 

\begin{enumerate}
\item defining the split point 
\item generating the result files. 
\item if desired delete the input file (with the browser, not the split
editor) 
\end{enumerate}

\subsubsection{How to use the Split Editor}

\begin{itemize}
\item \textbf{Pause near the split point}
When the device plays the song just hit the PAUSE button, when playback
has roughly reached the split point. This need not be very precise as
the split point can be fine tuned later.
\item \textbf{Open the split editor}

Open the plugin.  A screen similar to the one below will appear. 

{\centering\itshape
  [Warning: Image ignored] % Unhandled or unsupported graphics:
%\includegraphics[width=3.701cm,height=2.11cm]{images/rockbox-manual-img67.gif}
 \newline
The Split Editor
\par}

{\centering\upshape
Here is an explanation of the areas marked in red on the screenshot.
\par}

\begin{enumerate}
\item The waveform \newline
\newline
The waveform displays the volume of the song over time. It will appear
as the song plays and help to visually identify the point in time where
the split is desired
\item The split point indicator\newline
\newline
The split point indicator is a vertical line with a small triangle at
the top end. It is the most important control element of the split
editor. It can be moved with the LEFT and RIGHT buttons. Later, when
you have fine tuned the split point, the song will be split at this
position.
\item The split time\newline
\newline
At the top of the window a time value is displayed. This is the point in
time within the song at which the split point indicator is positioned. 
\item The locator\newline
\newline
Another vertical bar represents the position locator. It moves along as
the song plays. In contrast to the split point indicator it has no
triangles at the ends. 
\item The time bar\newline
\newline
The time bar displays the current position within the song relative to
the whole song. The entire length of the time bar represents the song
length. The length of the solid part of the time bar represents the position and length
of the displayed part of the song.
\item The scale mode\newline
\newline
Directly above the F3 button the scale mode is displayed. The waveform
can be scaled either logarithmically or linearly. In logarithmic scale
mode the letters ``dB'' are displayed, in linear mode ``\%''. Use F3 to
switch between these modes. Linear mode usually gives better optical
hints with commercially recorded music. For quiet recordings,
especially of human speech, the logarithmic scale often is preferable.
\item The loop mode \newline
\newline
Directly above the F2 button the loop mode icon is displayed. There are
4 different loop modes. Pressing F2 changes to the next loop mode. 

\begin{itemize}
\item   [Warning: Image ignored] % Unhandled or unsupported graphics:
%\includegraphics[width=0.794cm,height=0.476cm]{images/rockbox-manual-img68.gif}
  Playback loops around the split point indicator. This mode is best
used when searching and zooming for the desired point at which to split
the recording. 
\item   [Warning: Image ignored] % Unhandled or unsupported graphics:
%\includegraphics[width=0.794cm,height=0.476cm]{images/rockbox-manual-img69.gif}
  Playback loops from the split point indicator to the end of the
visible area. This mode is best used when fine tuning the split
indicator position at the beginning of a recording. 
\item   [Warning: Image ignored] % Unhandled or unsupported graphics:
%\includegraphics[width=0.794cm,height=0.476cm]{images/rockbox-manual-img70.gif}
  Playback loops from the beginning of the
visible area to the split point. This mode is best used when fine
tuning the split indicator position at the end of a recording.
\item   [Warning: Image ignored] % Unhandled or unsupported graphics:
%\includegraphics[width=0.688cm,height=0.476cm]{images/rockbox-manual-img71.gif}
  Playback doesn't loop, the borders of the visible
area as well as the split point indicator are ignored. This mode is
best used when playing the song outside of the borders of the displayed
region. 
\end{itemize}

\item Perform the split \newline
\newline
The icon directly above the F1 button indicates its function to execute
the split. When split positioning is complete open the save dialogue with F1.
\end{enumerate}

{\bfseries
Controls in the split editor }
\end{itemize}

\begin{tabular}[c]{|p{2.975cm}|p{3.047cm}|p{6.649cm}|}
\hline
{\centering\bfseries\itshape
Recorder 
\par}
&
{\centering\bfseries\itshape
Ondio 
\par}
&
{\centering\bfseries\itshape
Function 
\par}
\\\hline
{\centering
Off 
\par}
&
{\centering
On/Off 
\par}
&
Quit plugin 
\\\hline
{\centering
Left/Right 
\par}
&
{\centering
Left/Right 
\par}
&
Move the split point indicator 
\\\hline
{\centering
Up/Down 
\par}
&
{\centering
Up/Down 
\par}
&
Zoom in / out 
\\\hline
{\centering
Play 
\par}
&
{\centering
Mode 
\par}
&
Play from the split position 
\\\hline
{\centering
F1 
\par}
&
{\centering
Mode+Left 
\par}
&
Enter the save dialogue
\\\hline
{\centering
F2 
\par}
&
{\centering
Mode+Up 
\par}
&
Toggle loop modes 
\\\hline
{\centering
F3 
\par}
&
{\centering
Mode+Right 
\par}
&
Toggle logarithmic / linear scaling 
\\\hline
{\centering
On+Left 
\par}
&
{\centering
~ 
\par}
&
Play half speed 
\\\hline
{\centering
On+Right 
\par}
&
{\centering
~ 
\par}
&
Play 150\% speed 
\\\hline
{\centering
On+Play 
\par}
&
{\centering
~ 
\par}
&
Play normal speed 
\\\hline
\end{tabular}

\subsubsection{Save the files}
In the save dialogue it is possible to specify which of the files you
want to save and their names.  When finished, select
``Save'' and the files will be written to
disk. Note that files can not be overwritten, so filenames that
don't exist yet must be chosen. If unsure whether the
file already exists simply try to save it. If another file with this
name exists the dialogue will return and you can choose another
filename

{\centering\itshape
  [Warning: Image ignored] % Unhandled or unsupported graphics:
%\includegraphics[width=3.701cm,height=2.11cm]{images/rockbox-manual-img72.gif}
 \newline
Save dialogue
\par}

Controls in the save dialogue
\begin{tabular}[c]{|p{2.62cm}|p{2.266cm}|p{3.965cm}|}
\hline
{\centering\bfseries\itshape
RECORDER 
\par}
&
{\centering\bfseries\itshape
ONDIO 
\par}
&
{\centering\bfseries\itshape
FUNCTION 
\par}
\\\hline
{\centering
UP/DOWN 
\par}
&
{\centering
UP/DOWN 
\par}
&
Select item 
\\\hline
{\centering
PLAY 
\par}
&
{\centering
RIGHT 
\par}
&
Toggle / edit item 
\\\hline
\end{tabular}

\subsubsection{Scale}
The values in the waveform are scaled according to the settings of the
peak meter. These can be altered in the menu
\textbf{General Settings {}-{\textgreater} Display{}-{\textgreater} Peak Meter}. If extreme minimum /
maximum values are set the waveform might be cut off.  A minimum
setting of {}-60 dB and a maximum setting of 0 dB are recommended.
These settings should be capable of producing useful waveforms for very
soft sounds in logarithmic mode (dB). When the editor is used on loud
sounds (such as commercial rock or pop music) switching to the linear
scale may prove more effective since the logarithmic scale compresses
loud noises and makes it more difficult to identify characteristic
shapes. Note that it is always possible to toggle the scale with F3. 


