% $Id$ %
\appendix

% $Id$ %
\chapter{File formats}
\section{\label{ref:Supportedfileformats}Supported file formats}
\begin{table}
\begin{center}
\begin{tabularx}{\textwidth}{llXX}\toprule
\textbf{Icon} & \textbf{File Type} & \textbf{Extension} 
  & \textbf{Action when selected} \\\midrule
\includegraphics[width=0.37cm]{appendix/images/icon-directory.png} 
  & Directory & \emph{none} & The browser enters that directory \\
\includegraphics[width=0.37cm]{appendix/images/icon-audio-file.png} 
  & Audio file & 
  \opt{MASCODEC}{\fname{.mp2, .mp3}}\opt{swcodec}{\emph{various (see
    \reference{ref:Supportedaudioformats}})}
  & Rockbox takes you to the WPS and starts playing the file \\
  \includegraphics[width=0.37cm]{appendix/images/icon-cuesheet.png} 
  & Cuesheet & \fname{.cue} & View a cuesheet file \\
\opt{masf}{
  \includegraphics[width=0.37cm]{appendix/images/icon-wav-file.png} 
    & Wave Audio File & \fname{.wav} & Play a WAV file \\%
}
\includegraphics[width=0.37cm]{appendix/images/icon-playlist.png}
  & Playlist & \fname{.m3u, .m3u8} & Rockbox loads the playlist and starts playing 
    the first file \\
\includegraphics[width=0.37cm]{appendix/images/icon-rolo.png} 
  & Rockbox firmware & 
    \opt{player}{\fname{.mod}}\opt{recorder,recorderv2fm,ondiofm,ondiosp}{\fname{.ajz}}%
    \opt{h1xx,h300}{\fname{.iriver}}\opt{ipod}{\fname{.ipod}}\opt{iaudio}{\fname{.iaudio}}%
    \opt{h10,h10_5gb,sansa}{\fname{.mi4}}\opt{gigabeat}{\fname{.gigabeat}}
  & ROLO will load the new firmware \\
\includegraphics[width=0.37cm]{appendix/images/icon-wps.png} 
  & While Playing Screen & \fname{.wps} & The new WPS display configuration will be loaded \\
\includegraphics[width=0.37cm]{appendix/images/icon-lang.png} 
  & Language File & \fname{.lng} & Loads a language file \\
\includegraphics[width=0.37cm]{appendix/images/icon-text.png} 
  & Text File & \fname{.txt} & This will display the text file using Rockbox text browser plugin\\
\includegraphics[width=0.37cm]{appendix/images/icon-config.png} 
  & Configuration File & \fname{.cfg} & The settings file will be loaded\\
\includegraphics[width=0.37cm]{appendix/images/icon-font.png} 
  & Font & \fname{.fnt} & This font will replace the current one\\
\includegraphics[width=0.37cm]{appendix/images/icon-rock.png} 
  & Plugin & \fname{.rock} & Starts a Rockbox plugin\\
\opt{archos}{
  \includegraphics[width=0.37cm]{appendix/images/icon-ucl.png} 
    & Flash Image & \fname{.ucl} & This Rockbox image will be flashed into the ROM \\
  }
\includegraphics[width=0.37cm]{appendix/images/icon-chip8.png} 
  & Chip8 game & \fname{.ch8} & Play a Chip8 game \\
\includegraphics[width=0.37cm]{appendix/images/icon-image-file.png} 
  & Image & \fname{.jpg} & View a JPEG image \\
\opt{MASCODEC}{\opt{lcd_bitmap}{
  \includegraphics[width=0.37cm]{appendix/images/icon-movie-file.png} 
    & Rockbox Video & \fname{.rvf} & View a movie (Rockbox format)\\}
}
\bottomrule
\end{tabularx}
\end{center}
\end{table}

\opt{swcodec}{
  \section{\label{ref:Supportedaudioformats}Supported audio formats}
  \begin{table}
  \begin{center}
  \begin{tabularx}{\textwidth}{lXX}\toprule
  \textbf{Format} & \textbf{Extension} & \textbf{Notes} \\\midrule
    \emph{Lossy codecs} \\
    \midrule
    MPEG audio & \fname{.mp1, .mpa, .mp2, .mp3} & \\
    OGG/Vorbis & \fname{.ogg, .oga} & Some old ``floor 0'' files may crash Rockbox. \\
    Musepack & \fname{.mpc} & \\
    Advanced Audio Coding & \fname{.m4a, .m4b, .mp4} & \\
    Windows Media Audio & \fname{.wma, .wmv, .asf} & \\
    ATSC A/52 & \fname{.a52, .ac3} & Supports downmixing for playback of 5.1 streams in stereo. \\
    ADX & \fname{.adx} & \\
    Speex & \fname{.spx} & \\
    \\
    \midrule
    \emph{Lossless codecs} \\
    \midrule
    Waveform audio format & \fname{.wav} & \\
    Audio Interchange File Format & \fname{.aif, .aiff} & \\
    Free Lossless Audio & \fname{.flac} & \\
    Apple Lossless & \fname{.m4a, .mp4} & \\
    Wavpack & \fname{.wv} & \\
    Shorten & \fname{.shn} & Seeking not supported.\\
    \opt{h1xx,h300,x5,m5,m3}{
      Monkey's Audio & \fname{.ape, .mac} & -c1000 and -c2000 files decode fast enough to be useful. \\
    }
    \opt{gigabeat}{
      Monkey's Audio & \fname{.ape, .mac} & -c1000 to -c3000 files decode fast enough to be useful. \\
    }
    \nopt{h1xx,h300,x5,m5,m3,gigabeat}{%
      Monkey's Audio & \fname{.ape, .mac} & Only -c1000 files decode fast enough to be useful. \\
    }
    \\
    \midrule
    \emph{Other codecs} \\
    \midrule
    Sound Interface Device & \fname{.sid} & \\
    MOD & \fname{.mod} & \\
    NES Sound Format & \fname{.nsf, .nsfe} & \\
    SPC700 & \fname{.spc} & \\
    Atari SAP & \fname{.sap} & \\
  \bottomrule
  \end{tabularx}
  \end{center}
  \end{table}
}


\chapter{\label{ref:album_art}Album Art}
Rockbox allows you to put the album art, or another image related to the music
on your \dap{} to display it in the PictureFlow plugin\opt{albumart}{ or in the
theme}. For this feature to work, there are a few requirements.

\section{Limitations}

\opt{albumart}{%
   Rockbox supports embedded album art only for some specific formats, see
  \reference{ref:featureset_for_generic_metadata_tags} for full details. It additionally
  supports loading images located on the \disk{}. PictureFlow is currently unable to
  use embedded album art.
}%
\nopt{albumart}{%
   Rockbox currently only supports loading images located on the
   \disk{} for use in PictureFlow.
}%
The image files must be in either BMP or JPEG format\opt{albumart}{, while embedded
album art is currently limited to JPEG.  Embedded JPEG images must not be
unsynchronized}. Rockbox does not support RLE-compressed BMP files, nor does it
support progressive and multi-scan JPEG files.
JPEG files must consist of a single scan with interleaved components, 
as progessive and multi-scan images require much more memory to decode.

\section{Where to put album art}

The pictures can be named a number of different ways, and placed to a number of
different locations. You can have pictures specific to the file or the album
or use a generic picture. You can place the picture in the same directory
as the file, in the parent directory or in a fixed directory named
\fname{/.rockbox/albumart/}. The order Rockbox uses when looking for a picture
is as follows (a list in braces means that those file extensions are tried in
that order):

\begin{enumerate}
\item  embedded (JPEG images in ID3v2 or MP4 tags only)
\item  \fname{./filename.\{jpeg,jpg,bmp\}}
\item  \fname{./albumtitle.\{jpeg,jpg,bmp\}}
\item  \fname{./cover.\{jpeg,jpg,bmp\}}
\item  \fname{./folder.jpg}
\item  \fname{/.rockbox/albumart/albumartist-albumtitle.\{jpeg,jpg,bmp\}}
\item  \fname{../albumtitle.\{jpeg,jpg,bmp\}}
\item  \fname{../cover.\{jpeg,jpg,bmp\}}
\end{enumerate}

The following characters will be replaced with an underscore (\_) when looking
for albumtitle.bmp or albumartist-albumtitle.bmp: \textbackslash{} / : <
> ? * |. Doublequotes will be replaced by single quotes.
If no album artist is set, artist will be used instead. See \wikilink{AlbumArt}
in the wiki for programs that will help you automate the process of putting
album art on your \dap{}.


% $Id$ %
\chapter{\label{ref:wps_tags}WPS Tags}
\section{Status Bar}
\begin{tagmap}{}{}
\%we & Status Bar Enabled\\
\%wd & Status Bar Disabled\\
\end{tagmap}
These tags override the player setting for the display of the status bar.
They must be noted on their own line.

\section{ID3 Info}
  \begin{tagmap}{}{}
    \%ia & ID3 Artist\\
    \%ic & ID3 Composer\\
    \%id & ID3 Album Name\\
    \%ig & ID3 Genre Name\\
    \%in & ID3 Track Number\\
    \%it & ID3 Track Title\\
    \%iv & ID3 Version (1.0, 1.1, 2.2, 2.3, 2.4 or empty if no id3 tag)\\
    \%iy & ID3 Year\\
  \end{tagmap}
Remember that this information is not always available, so use the 
conditionals to show alternate information in preference to assuming.

\section{Power Related Information}
  \begin{tagmap}{}{}
    \%bl & Show numeric battery level in percent.\\
         & Can also be used in a conditional: 
           \%?bl{\textless}0{\textbar}1{\textbar}2{\textbar}3{\textbar}4{\textgreater}\\
    \%bv & Show the battery level in volts\\
    \%bt & Show estimated battery time left\\
    \%bp & ``p'' if the charger is connected \\
         & (only on targets that can charge batteries)\\
    \%bc & ``c'' if the unit is currently charging the battery\\
         & (only on targets that have software charge control or monitoring)\\
    \%bs & Sleep timer. Shows the remaining time if the sleeptimer is set\\
  \end{tagmap}

\section{File Info}
  \begin{tagmap}{}{}
    \%fb & File Bitrate (in kbps)\\
    \%fc & File Codec (e.g. ``MP3'' or ``FLAC'')\\
         & This tag can also be used in a conditional tag,\\
         & \%?fc{\textless}mp1{\textbar}mp2{\textbar}mp3{\textbar}wav%
           {\textbar}vorbis{\textbar}flac{\textbar}mpc{\textbar}a52%
           {\textbar}wavpack{\textbar}unknown{\textgreater}\\ %
         & The codec order is as follows: MP1, MP2, MP3, WAV, Ogg Vorbis (OGG),%
           FLAC, MPC, AC3, WavPack (WV), ALAC, AAC, Shorten (SHN), AIFF\\
    \%ff & File Frequency (in Hz)\\
    \%fm & File Name\\
    \%fn & File Name (without extension)\\
    \%fp & File Path\\
    \%fs & File Size (In Kilobytes)\\
    \%fv & ``(avg)'' if variable bit rate or ``'' if constant bit rate\\
    \%d1 & First directory from end of file path.\\
    \%d2 & Second directory from end of file path.\\
    \%d3 & Third directory from end of file path.\\
  \end{tagmap}
Example for the the \%dN commands: If the path is 
``/Rock/Kent/Isola/11 - 747.mp3'', \%d1 is ``Isola'', \%d2 is ``Kent'' \dots
You get the picture.

\section{Playlist/Song Info}
  \begin{tagmap}{}{}
    \%pb & Progress Bar\\
    \opt{player}{
          & This will display a 1 character ``cup'' %
            that empties as the time progresses.}
    \opt{recorder,recorderv2fm,h1xx,h300,ipodcolor,ipodnano}{
         & This will replace the entire line with a progress bar. \\
         & You can set the height, position and width of the progressbar %
           (in pixels): \%pb{\textbar}height{\textbar}leftpos%
           {\textbar}rightpos{\textbar}} \\
    \opt{player}{
    \%pf & Full-line progress bar \& time display\\
    }
    \%pc & Current Time In Song\\
    \%pe & Total Number of Playlist Entries\\
    \%pm & Peak Meter (Recorder only) The entire line is used as volume peak meter.\\
    \%pn & Playlist Name (Without path or extension)\\
    \%pp & Playlist Position\\
    \%pr & Remaining Time In Song\\
    \%ps & Shuffle. Shows 's' if shuffle mode is enabled.\\
    \%pt & Total Track Time\\
    \%pv & Current volume. Can also be used in a conditional: \\
         & \%?pv{\textless}0{\textbar}1{\textbar}2{\textbar}3%
           {\textbar}4{\textbar}5{\textbar}6{\textbar}7{\textbar}8%
           {\textbar}9{\textbar}10{\textgreater}\\
  \end{tagmap}

\section{Runtime Database}
  \begin{tagmap}{}{}
    \%rp & Song playcount\\
    \%rr & Song rating (0-10). This tag can also be used in a conditional tag, %
           \%?rr{\textless}0{\textbar}1{\textbar}2{\textbar}3{\textbar}%
           4{\textbar}5{\textbar}6{\textbar}7{\textbar}8{\textbar}9{\textbar}%
           10{\textgreater}\\
  \end{tagmap}

\opt{h1xx,h300}{
\section{Hold Switches}
  \begin{tagmap}{}{}
    \%mh & ``h'' if the main unit hold switch is on\\
    \%mr & ``r'' if the remote hold switch is on\\
  \end{tagmap}
}

\section{Virtual LED}
  \begin{tagmap}{}{}
    \%lh & ``h'' if there is hard disk activity\\
  \end{tagmap}

\section{Repeat Mode}
  \begin{tagmap}{}{}
    \%mm & Repeat mode, 0-4, in the order: Off, All, One, Shuffle
           \opt{player,recorder,recorderv2fm}{, A-B}\\
  \end{tagmap}
Example: \%?mm{\textless}Off{\textbar}All{\textbar}One{\textbar}Shuffle%
{\textbar}A-B{\textgreater}

\section{Playback Mode Tags}
  \begin{tagmap}{}{}
    \%mp & Play status, 0-4, in the order: Stop, Play, Pause, 
           Fast forward, Rewind\\
  \end{tagmap}
Example: \%?mp{\textless}Stop{\textbar}Play{\textbar}Pause{\textbar}%
Ffwd{\textbar}Rew{\textgreater}

\section{Images}
  \begin{tagmap}{}{}
    \opt{h300,x5,ipodcolor,ipodvideo}{
    \%X{\textbar}filename.bmp{\textbar} 
        & Load and set a backdrop image for the WPS. %
          This image must be exactly the same size as your LCD.\\
    }
    \%P{\textbar}filename.bmp{\textbar} 
        & Load a Progress bar image for the WPS. Use \%pb tag to show the 
          progress bar\\
    \%x{\textbar}n{\textbar}filename{\textbar}x{\textbar}y{\textbar} 
        & Load and display an image\\
        & n: image ID (a-z and A-Z)\\
        & filename: filename relative to \fname{/.rockbox/} and including .bmp\\
        & x: x coordinate\\
        & y: y coordinate.\\
    \%xl{\textbar}n{\textbar}filename{\textbar}x{\textbar}y{\textbar} 
        & Preload an image for later display\\
        & n: image ID (a-z and A-Z)\\
        & filename: filename relative to \fname{/.rockbox/} and including .bmp\\
        & x: x coordinate\\
        & y: y coordinate.\\
    \%xdn & Display a preloaded image\\
        & n: image ID (a-z and A-Z)\\
  \end{tagmap}

Example: image \fname{/.rockbox/bg.bmp} with ID ``a'' at 37, 109 would be:\\
\%x{\textbar}a{\textbar}bg.bmp{\textbar}37{\textbar}109{\textbar}

\note{
  \begin{itemize}
  \item The images must be in a rockbox compatible format (1 bit per pixel BMP)
  \item The image tag must be on its own line
  \item The ID is case sensitive, giving 52 different ID's
  \item The size of the LCD screen for each player varies. See table below 
        for appropriate sizes of each device. The x and y coordinates must 
        repect each of the players' limits.
  \end{itemize}
}

\section{Alignment}
  \begin{tagmap}{}{}
    \%al & Text is left aligned\\
    \%ac & Text is center aligned\\
    \%ar & Text is right aligned\\
  \end{tagmap}
All alignment tags may be present in one line, but they need to be in the 
order left -- center -- right. If the aligned texts overlap, they are merged.

\section{Conditional Tags}

\begin{tagmap}{}{}
\%?xx{\textless}true{\textbar}false{\textgreater}
    & If / Else: Evaluate for true or false case \\
\%?xx{\textless}alt1{\textbar}alt2{\textbar}alt3{\textbar}\dots{\textbar}else{\textgreater}
    & Enumerations: Evaluate for first / second / third / \dots / last condition \\
\end{tagmap}

\section{Other Tags}
\begin{tagmap}{}{}
  \%\%          & Display a `\%'\\
  \%{\textless} & Display a `{\textless}'\\
  \%{\textbar}  & Display a `{\textbar}'\\
  \%{\textgreater} & Display a `{\textgreater}'\\
  \%;           & Display a `;'\\
  \%s           & Indicate that the line should scroll. Can occur anywhere in 
                  a line (given that the text is displayed; see conditionals 
                  above). You can specify up to 10 scrolling lines. Scrolling
                  lines can not contain dynamic content such as timers, 
                  peak meters or progress bars.\\
\end{tagmap}



% $Id$ %
\chapter{\label{ref:config_file_options}Config file options}
\begin{center}
% define a local version of endhead, as using the output distinction adds
% an unwanted newline. endhead breaks with htlatex so we need to remove it
% for the html output.
\ifpdfoutput{\newcommand{\localendhead}{\endhead}}%
    {\newcommand{\localendhead}{}}
  \begin{longtable}{@{}>{\raggedright}p{.35\textwidth}@{}>{\raggedright}p{.4\textwidth}@{}p{.25\textwidth}@{}}
    \toprule
    \textbf{Setting} & \textbf{Allowed Values} & \textbf{Unit}\\
    \midrule\localendhead % endhead breaks with htlatex
    volume      & \opt{player}{-78 to +18}%
                  \opt{recorder,recorderv2fm,ondiosp,ondiofm}{-100 -to +12}%
                  \opt{h1xx,h300}{-84 to 0}%
                  \opt{ipodnano}{-72 to +6}%
                  \opt{ipodvideo}{-57 to +6}%
                  \opt{x5}{-73 to +6}
                  \opt{e200}{-74 to +6}
                  \opt{ipodcolor}{-\fixme{??} to +\fixme{??}}%
                                        & dB\\
    \nopt{x5}{%
      bass      & \opt{MASCODEC}{-15 to +15}%
                  \opt{h1xx,h300}{0 to +24}%
                  \opt{ipod}{-6 to +9}%
                  \opt{e200}{-24 to +24}%
                                        & dB\\
      treble    & \opt{MASCODEC}{-15 to +15}%
                  \opt{h1xx,h300}{0 to +6}%
                  \opt{ipod}{-6 to +9}%
                  \opt{e200}{-24 to +24}%
                                        & dB\\
    }%
    balance         & -100 to +100      & \%\\
    channels        & stereo, mono, custom, mono left, mono right, karaoke
                                        & N/A\\
    shuffle         & on, off               & N/A\\
    repeat          & off, all, one, shuffle, ab
                                        & N/A\\
    play selected   & on, off           & N/A\\
    resume          & on, off           & N/A\\
    scan min step   & 1, 2, 3, 4, 5, 6, 8, 10, 15, 20, 25, 30, 45, 60
                                        & seconds\\
    scan accel      & 0 to 15           & seconds\\
    antiskip        & 0 to 7            & seconds\\
    volume fade     & on, off           & N/A\\
    id3 tag priority & v2-v1, v1-v2     & N/A\\
    sort case       & on, off           & N/A\\
    show files  & all, supported, music, playlists
                                        & N/A\\
    follow playlist & on, off           & N/A\\
    playlist viewer icons
                    & on, off           & N/A\\
    playlist viewer indices
                    & on, off           & N/A\\
    playlist viewer track display
                    & track name,full path
                                        & N/A\\
    recursive directory insert
                    & on, off           & N/A\\
    scroll speed    & 1 to 25           & Hz\\
    scroll delay    & 0 to 250          & 1/10s\\
    scroll step     & 1 to 112          & pixels\\
    bidir limit     & 0 to 200          & \% screen\\
    contrast        & 0 to 63           & N/A\\
    backlight timeout
                    & off, on, 1, 2, 3, 4, 5, 6, 7, 8, 9, 10, 15, 20, 25, 30,
                      45, 60, 90        & seconds\\
    backlight timeout plugged
                    & off, on, 1, 2, 3, 4, 5, 6, 7, 8, 9, 10, 15, 20, 25, 30,
                      45, 60, 90        & seconds\\
    disk spindown   & 3 to 254          & seconds\\
    battery capacity
                    & 1500 - 3200      & mAh\\
    idle poweroff   & off, 1, 2, 3, 4, 5, 6, 7, 8, 9, 10, 15, 30, 45, 60
                                        & minutes\\
    lang            & /path/filename.lng & N/A\\
    wps             & /path/filename.wps & N/A\\
    autocreate bookmarks
                    & off, on           & N/A\\
    autoload bookmarks
                    & off, on           & N/A\\
    use most-recent-bookmarks
                    & off, on           & N/A\\
    talk dir        & off, number, spell, enter, hover
                                        & N/A\\
    talk file       & off, number, spell& N/A\\
    talk menu       & off, on           & N/A\\
    tagcache\_autoupdate
                    & on, off           & N/A\\
    warn when erasing dynamic playlist
                    & on, off           & N/A\\
    cuesheet support
                    & on, off           & N/A\\
%
    \opt{swcodec}{
      replaygain    & on, off           & N/A\\
      replaygain type
                    & track, album, track shuffle
                                        & N/A\\
      replaygain noclip
                    & on, off           & N/A\\
      replaygain preamp
                    & -120 to 120       & 0.1dB\\
%
      crossfade     & off, shuffle, track skip, always
                                        & N/A\\
      crossfade fade in delay
                    & 0 to 7            & seconds\\
      crossfade fade out delay
                    & 0 to 7            & seconds\\
      crossfade fade in duration
                    & 0 to 15           & seconds\\
      crossfade fade out duration
                    & 0 to 15           & seconds\\
      crossfade fade out mode
                    & crossfade, mix    & N/A\\
%
      crossfeed     & on, off           & N/A\\
      crossfeed direct gain
                    & 0 to 60           & 0.1dB\\
      crossfeed cross gain
                    & 30 to 120         & 0.1dB\\
      crossfeed hf attenuation
                    & 60 to 240         & 0.1dB\\
      crossfeed hf cutoff
                    & 500 to 2000       & Hz\\
%
      eq enabled    & on, off           & N/A\\
      eq precut     & 0 to 240          & 0.1dB\\
      eq band 0 cutoff & 0 to 32768     & Hz\\
      eq band 1 cutoff & 0 to 32768     & Hz\\
      eq band 2 cutoff & 0 to 32768     & Hz\\
      eq band 3 cutoff & 0 to 32768     & Hz\\
      eq band 4 cutoff & 0 to 32768     & Hz\\
      eq band 0 q   & 0 to 64           & N/A\\
      eq band 1 q   & 0 to 64           & N/A\\
      eq band 2 q   & 0 to 64           & N/A\\
      eq band 3 q   & 0 to 64           & N/A\\
      eq band 4 q   & 0 to 64           & N/A\\
      eq band 0 gain & -240 to 240      & 0.1dB\\
      eq band 1 gain & -240 to 240      & 0.1dB\\
      eq band 2 gain & -240 to 240      & 0.1dB\\
      eq band 3 gain & -240 to 240      & 0.1dB\\
      eq band 4 gain & -240 to 240      & 0.1dB\\
%
      beep          & off, weak, moderate, strong
                                        & N/A\\
      dircache      & on, off           & N/A\\
      tagcache\_ram & on, off           & N/A\\
    }%

    \opt{recorder,recorderv2fm}{
      loudness      & 0 to 17           & N/A\\
      superbass     & on, off           & N/A\\
      auto volume   & off, 20ms, 2s, 4s, 8s
                                        & seconds\\
      mdb enable    & on,off            & N/A\\
      mdb strength  & 0 to 127          & dB\\
      mdb harmonics & 0 to 100          & \%\\
      mdb center    & 20 to 300         & Hz\\
      mdb shape     & 50 to 300         & Hz\\
    }%

    \opt{lcd_bitmap}{
      peak meter release
                    & 1 to 126          & ?\\
      peak meter hold
                    & off, 200ms, 300ms, 500ms, 1, 2, 3, 4, 5, 6, 7, 8, 9, 10,
                      15, 20, 30, 1min  & N/A \\
      peak meter clip hold
                    & on, 1, 2, 3, 4, 5, 6, 7, 8, 9, 10, 15, 20, 25, 30, 45,
                      60, 90, 2min, 3min, 5min, 10min, 20min, 45min, 90min
                                        & N/A \\
      peak meter busy & on, off         & N/A\\
      peak meter dbfs & on, off         & on:~dbfs, off:~linear\\
      peak meter min  & 0 to 89 (dB) or 0 to 100 (\%)
                                        & dB or \%\\
      peak meter max  & 0 to 89 /(dB) or 0 to 100 (\%)
                                        & dB or \%\\
      statusbar     & on, off           & N/A\\
      scrollbar     & on, off           & N/A\\
      volume display
                    & graphic, numeric  & N/A\\
      battery display
                    & graphic, numeric  & N/A\\
      font          & /path/filename.fnt & N/A\\
      invert        & on, off           & N/A\\
    }%

    \opt{swcodec}{% This doesn't depend on swcodec but using a \nopt here
                  % causes ondiosp not to build for mysterious reasons.
      backdrop      & /path/filename.bmp    & N/A\\
    }%

    \opt{lcd_color}{
      foreground color & 000000 to FFFFFF   & RRGGBB\\
      background color & 000000 to FFFFFF   & RRGGBB\\
      line selector start color & 000000 to FFFFFF  & RRGGBB\\
      line selector end color   & 000000 to FFFFFF  & RRGGBB\\
    }

    \opt{HAVE_REMOTE_LCD}{
      rwps      & /path/filename.rwps   & N/A\\
      remote contrast
                & 5 to 63               & N/A\\
      remote invert
                & on, off               & N/A\\
      remote flip display
                & on, off               & N/A\\
      remote backlight timeout
                & off, on, 1, 2, 3, 4, 5, 6, 7, 8, 9, 10, 15, 20, 25,
                  30, 45, 60, 90        & seconds\\
      remote backlight timeout plugged
                & off, on, 1, 2, 3, 4, 5, 6, 7, 8, 9, 10, 15, 20, 25,
                  30, 45, 60, 90        & seconds\\
      remote caption backlight
                & on, off               & N/A\\
      remote scroll speed
                & 0 to 15               & N/A\\
      remote scroll step
                & 1 to 160              & N/A\\
      remote scroll delay
                & 0 to 250              & N/A\\ 
      remote bidir limit
                & 0 to 200              & N/A\\
      backlight filters first remote keypress
                & on, off               & N/A\\
      \opt{h1xx,h300}{
        remote reduce ticking
                & on, off               & N/A\\
      }%
    }
    \opt{rtc}{
      time format & 12hour, 24hour      & N/A\\
    }%
    \opt{recording}{
     rec quality & 0 to 7               & 0: small size, 7: high quality\\
     rec frequency
                & 48, 44, 32, 24, 22, 16 & kHz\\
     rec source & mic, line, spdif      & N/A\\
     rec channels & mono, stereo        & N/A\\
     rec mic gain & 0 to 15             & N/A\\
     rec left gain & 0 to 15            & N/A\\
     rec right gain
                & 0 to 15               & N/A\\
     editable recordings
                & off,on                & N/A\\
     rec timesplit
                & off, 0:05, 0:10, 0:15, 0:30, 1:00, 2:00, 4:00, 6:00,
                  8:00, 16:00, 24:00    & h:mm\\
     pre-recording time
                & off, 1 to 30          & seconds\\
     rec directory
                & /recordings,current   & N/A\\
    }%
    \opt{radio}{
      force fm mono
                & off, on               & N/A\\
    }%
    \opt{player}{
      jump scroll
                & 0 to 5                & N/A\\
      jump scroll delay
                & 0 to 250              & 0.01s\\
    }%

    \bottomrule
  \end{longtable}
\end{center}


\chapter{Menu Overview}
\fixme{include an overview of the menu structure here}


\chapter{User feedback}\label{sec:feedback}
\section{Bug reports}
If you experience inappropriate performance from any supported feature,
please file a bug report on our web page. Do not report missing
features as bugs, instead file them as feature ideas (see below).

For open bug reports refer to
\url{http://www.rockbox.org/tracker/index.php?type=2}

\subsection{Rules for submitting new bug reports}

\begin{enumerate}
\item  Check that the bug has not already been reported
\item  Always include the following information in your bug report:

\begin{itemize}
\item  Which exact \dap{} you have.
\item  Which exact Rockbox version you are using
(Menu $\rightarrow$ System $\rightarrow$ Rockbox Info $\rightarrow$ Version)
\item  A step{}-by{}-step description of what you did and what happened
\item  Whether the problem is repeatable or a one{}-time occurrence
\item  All relevant data regarding the problem, such as playlists, MP3
files etc. (IMPORTANT!)
\end{itemize}
\end{enumerate}

\section{Feature ideas}
To suggest an idea for a feature or to read those made by others, see
\url{http://forums.rockbox.org/index.php?board=49.0}.  Please keep in
mind that this forum is for the discussion of feature ideas -- they are not
 requests and there is no guarantee they will be acted upon.

\subsection{Rules for submitting a new feature idea}

\begin{enumerate}
\item Check that the feature has not already been suggested. 
  Duplicates are really boring!
\item Check that the feature has not already been implemented. 
  Download the latest current/daily build and/or search the mail list archive.
\item Check that the feature is possible to implement (see \reference{ref:NODO}).
\end{enumerate}

\subsection{\label{ref:NODO}Features we will not implement}
This is a list of Feature Requests we get repeatedly that we simply
cannot do. View it as the opposite of a TODO!

\begin{itemize}
\nopt{iriverh300,iaudiox5}{
\item Interfacing with other USB devices (like cameras) or 2 player games over USB.\\
  The USB system demands that there is a master that talks to a slave. The
  \dap{} can only serve as a slave, as most other USB devices such as
  cameras can. Thus, without a master no communication between the slaves
  can take place. If that is not enough, we have no way of actually
  controlling the communication performed over USB since the USB circuit
  in the \dap{} is strictly made for disk{}-access and does not allow us
  to play with it the way we'd need for any good communication to work.
}
\item Support other file systems than FAT32 (like NTFS or ext2 etc.).\\
  No.  (Except perhaps for ExFAT)
  Most \dap{}s can only start off FAT32 partitions, so adding support
  for more file systems will just take away valuable ram for
  unnecessary features. You can partition your \dap{} fine, just make sure
  the first one is FAT32 and then make the other ones whatever file system
  you want. Just do not expect Rockbox to understand them.
\item Add scandisk{}-like features.\\
  It would be a very slow operation that would drain the batteries and
  take a lot of useful ram for something that is much better and faster
  done when connected to a host computer.
\item Alphabetical list skipping.\\
  Skipping around the lists by jumping letters (i.e skip all C's and go
  straight to the first D). This isn't feasible with the current list
  implementation, if you really want this you can get similar effects using
  the database (see \reference{ref:database}).
\item Add support for non standard tag formats.\\
APE tags in MP3 files has been rejected a few times already. Its not something we want.
\item Implementing the ability to playback DRM files.\\
  Firstly, this would be extremely difficult to implement legally -- Rockbox
  is not legal entity as such, and therefore is unable to enter into license
  agreements with providers of DRM technology.
  Secondly, Rockbox is open source, which would mean that any DRM technology we
  incorporated into our codebase would suddenly become visible to the whole world,
  completely defeating its purpose. Remember, DRM achieves part of its security
  through obscurity, and publishing the keys necessary to decrypt DRM'd 
  media would essentially render it useless.
\end{itemize}

\chapter{Credits}
People that have contributed to the project, one way or another. Friends!
%
\begin{multicols}{2}
\noindent\caps{\small{\input{CREDITS.tex}}}
\end{multicols}

\chapter{Licenses}

\section{GNU Free Documentation License}
\input{appendix/fdl.tex}
\newpage
\section{The GNU General Public License}
\input{appendix/gpl-2.0.tex}
