\subsection{Installing the bootloader}
  Installing the bootloader is the trickiest part of the installation.
  The process is different depending on your operating system, but before
  starting, connect the \dap\ to the computer using either an USB \fixme{or
  Firewire?} cable. Next, create a folder on the computer's harddrive and
  download the following file to that folder:
\opt{ipodvideo}{\url{http://www.rockbox.org/twiki/pub/Main/IpodInstallation/bootloader-video.bin}}
\opt{ipodnano}{\url{http://www.rockbox.org/twiki/pub/Main/IpodInstallation/bootloader-nano.bin}}
\opt{ipodmini}{\url{http://www.rockbox.org/twiki/pub/Main/IpodInstallation/bootloader-mini1g.bin} or \url{http://www.rockbox.org/twiki/pub/Main/IpodInstallation/bootloader-mini1g.bin} depending on which generation your \dap\ is.\fixme{Describe how to identify 1/2G}}
\opt{ipodcolor}{\url{http://www.rockbox.org/twiki/pub/Main/IpodInstallation/bootloader-color.bin}}
\opt{ipod4g}{\url{http://www.rockbox.org/twiki/pub/Main/IpodInstallation/bootloader-4g.bin}}

  When that is done, proceed to the section below that matches the operating
  system on the computer.

  \note{These instructions all require you to have administrator rights
  on your computer, regardless of the operating system.}
  
  \note{Rockbox only works on FAT32 partitions (called Windows formatted by
    Apple). So if your \dap\ is Mac formatted (HFS+), you should first convert
    it to FAT32. Information on how to do this can be found on the Rockbox
    website. \fixme{Include these instructions?}}

\subsubsection{Windows users}
\begin{enumerate}
  \item Download the following two programs and save them in the folder just
    created. These programs will be used in the next steps:
    \begin{itemize}
      \item \url{http://www.rockbox.org/twiki/pub/Main/IpodInstallation/ipodpatcher.exe}
      \item \url{http://www.rockbox.org/twiki/pub/Main/IpodInstallation/ipod_fw.exe}
    \end{itemize}
    \fixme{Is this needed?}
    These two programs are both licensed under the GPL, just like Rockbox
    itself. The source can be downloaded from the Rockbox website.
  \item Locate the \dap\ by opening a commandprompt, change directory to the
    folder you created and run the following commands:
    \begin{verbatim}
    ipodpatcher 0
    ipodpatcher 1
    ipodpatcher 2
    ipodpatcher 3
    etc.
    \end{verbatim}
    Keep increasing the number until the \dap\ is located. \fixme{What is the
    output?} Remember the number that corresponds to your \dap\ -- in the 
    following steps, N should be replaced with the number just found.
  \item Now, extract the firmware partition currently on the \dap\  with the
    following command:
    \begin{verbatim}
    ipodpatcher -r N bootpartition.bin
    \end{verbatim}
    \note{You should keep a safe backup of this \fname{bootpartition.bin} file
      for use if you ever wish to either upgrade the Rockbox bootloader or
      uninstall Rockbox from your iPod
    }
  \item Extract the Apple firmware from the partition image image just created:
    \begin{verbatim}
    ipodpatcher -r N bootpartition.bin
    \end{verbatim}
\opt{ipodvideo}{
  \item Similarly, extract the Broadcom firmware: \fixme{Formatting broken. Let's cause a break.}
}
\optv{ipodvideo}{\verb|ipod_fw -o apple_sw_5g_rcsc.bin -e 1 bootpartition.bin|}

    \begin{verbatim}
    ipod_fw -g GEN -o rockboot.bin -i apple_os.bin bootloader-GEN.bin
    \end{verbatim}
  \item
    Install the Rockbox-enabled firmware:
    \begin{verbatim}
    ipodpatcher -w N rockboot.bin
    \end{verbatim}
\end{enumerate}

Now, proceed to \fixme{insert ref?} Installing the firmware.

\subsubsection{Mac OS X users}
\begin{enumerate}
  \item Download the following two programs and save them in the folder just
    created. These programs will be used in the next steps:
    \begin{itemize}
      \item \url{http://www.rockbox.org/twiki/pub/Main/IpodInstallationFromMacOSX/diskdump}
      \item \url{http://www.rockbox.org/twiki/pub/Main/IpodInstallationFromMacOSX/ipod_fw}
    \end{itemize}
    Start a Terminal and type navigate into the folder you created. Before
    you can continue, you need to ensure that Mac OS knows that the
    \begin{verbatim}
    	ipod_fw
    \end{verbatim}
    and diskdump files you downloaded are executable programs. To do this,
    type the following command:
    \begin{verbatim}
    chmod +x ipod_fw diskdump
    \end{verbatim}
    \fixme{Is this needed?}
    These two programs are both licensed under the GPL, just like Rockbox
    itself. The source can be downloaded from the Rockbox website.
  \item Locate the \dap\ by running the following command:
    \begin{verbatim}
    mount
    \end{verbatim}
    The output will look something like this: \fixme{Add full example}
    \begin{verbatim}
    /dev/disk1s2 on /Volumes/DAVE_S IPOD 1 (local, nodev, nosuid)
    \end{verbatim}
    In this example, the \dap\ is located at /dev/disk1s2 Remember the 
    location of your \dap\  -- in the following steps, /dev/disk1s2 should be
    replaced with the location just found.
  \item Before continuing, the \dap\ must be ``unmounted'', which is
    done with the following command:
    \begin{verbatim}
    diskutil unmount /dev/disk1s2
    \end{verbatim}
  \item Now, extract the Apple firmware currently on the \dap\ with the
    following command:
    \note{The last part of the location is left out.}
    \begin{verbatim}
    ./diskdump -r /dev/disk1 bootpartition.bin
    \end{verbatim}
    \note{You should keep a safe backup of this \fname{bootpartition.bin} file                                                                                                for use if you ever wish to either upgrade the Rockbox bootloader or
      uninstall Rockbox from your iPod
    }
  \item Extract the Apple firmware from this partition image:
    \begin{verbatim}
    ./ipod_fw -o apple_os.bin -e 0 bootpartition.bin
    \end{verbatim}
\opt{ipodvideo}{
  \item Similarly, extract the Broadcom firmware: \fixme{Formatting broken. Let's cause a break.}
}
\optv{ipodvideo}{\verb|./ipod_fw -o apple_sw_5g_rcsc.bin -e 1 bootpartition.bin|}

  \item Merge the Rockbox bootloader you downloaded previously with the Apple
    firmware \fixme{Substitute GEN with your \daps\ generation: color,video,nano,4g,mini1g,mini2g}:
    \begin{verbatim}
    ./ipod_fw -g GEN -o rockbox.bin -i apple_os.bin bootloader-GEN.bin
    \end{verbatim}
  \item
    Install the Rockbox-enabled firmware:
    \note{The last part of the location is left out.}
    \begin{verbatim}
    ./diskdump -w /dev/disk1 rockboot.bin
    \end{verbatim}
\end{enumerate}

Now, proceed to \fixme{insert ref?} Installing the firmware.

\subsubsection{Linux users}
\begin{enumerate}
  \item Download the following and save it in the folder just
    created:
    \begin{itemize}
      \item \url{http://www.rockbox.org/viewcvs.cgi/*checkout*/tools/ipod_fw.c}
    \end{itemize}
    Now compile it to an executable by opening a command prompt and changing
    to the folder created previously. Thn run the following command:
    \begin{verbatim}
    gcc -o ipod_fw ipod_fw.c
    \end{verbatim}
    If you get the message that the command gcc is not found, you need to
    install gcc. How to do this depends on your Linux distribution, and
    you should consult its documentation for help on this.
    \fixme{Is this needed?}
    This program is licensed under the GPL, just like Rockbox itself. The
    source can be downloaded from the Rockbox website.
  \item Locate your Ipod by running the command \verb|dmesg|. In the output
    something like the following should be seen:
\begin{verbatim}
usb 4-1: new high speed USB device using ehci_hcd and address 7
scsi4 : SCSI emulation for USB Mass Storage devices
usb-storage: device found at 7
usb-storage: waiting for device to settle before scanning
  Vendor: Apple     Model: iPod              Rev: 1.62
  Type:   Direct-Access                      ANSI SCSI revision: 00
SCSI device sdb: 58605120 512-byte hdwr sectors (30006 MB)
\end{verbatim}
    In this example, the \dap\ is located on /dev/sdb. In the following,
    /dev/sdb should be replaced with the location just found.
  \item Run \verb|fdisk -l /dev/sdb| and press \verb|p|. Verify that the
    output is similar to the one below:
    \begin{verbatim}
## Some text removed
Device Boot      Start         End      Blocks   Id  System
/dev/sdb1               1          10       80293+   0  Empty
/dev/sdb2              11        3648    29222235    b  W95 FAT32
    \end{verbatim}
    Exit fdisk by pressing \verb|q|
  \item Back up the partition table using the following command:
    \note{The last part of the location is left out.}
    \begin{verbatim}
    dd if=/dev/sdb of=mbr.bin count=1
    \end{verbatim}

  \item Now, extract the firmware partition currently on the \dap\ with the
    following command:
    \begin{verbatim}
    dd if=/dev/sdb1 of=bootpartition.bin
    \end{verbatim}
    \note{You should keep a safe backup of this \fname{bootpartition.bin} file                                                                                                for use if you ever wish to either upgrade the Rockbox bootloader or
      uninstall Rockbox from your iPod
    }
  \item Extract the Apple firmware from this partition image:
    \begin{verbatim}
    ./ipod_fw -o apple_os.bin -e 0 bootpartition.bin
    \end{verbatim}
\opt{ipodvideo}{
  \item Similarly, extract the Broadcom firmware: \fixme{Formatting broken. Let's cause a break.}
}
\optv{ipodvideo}{\verb|./ipod_fw -o apple_sw_5g_rcsc.bin -e 1 bootpartition.bin|}

  \item Merge the Rockbox bootloader you downloaded previously with the Apple
    firmware \fixme{Substitute GEN with your \daps\  generation: color,video,nano,4g,mini1g,mini2g}:
    \begin{verbatim}
    ./ipod_fw -g GEN -o rockbox.bin -i apple_os.bin bootloader-GEN.bin
    \end{verbatim}
  \item
    Install the Rockbox-enabled firmware:
    \begin{verbatim}
    dd if=rockboot.bin of=/dev/sdb1
    \end{verbatim}
\end{enumerate}

Now, proceed to \fixme{insert ref?} Installing the firmware.
