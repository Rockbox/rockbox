% $Id$ %
\chapter{Getting started}
\section{Welcome}
This is the manual for Rockbox. Rockbox is an open source firmware replacement
for a growing number of MP3 players. Rockbox aims to be considerably more
functional and efficient than your device's stock firmware while remaining easy
to use and customizable. Rockbox is written by users, for users. Not only is it
free to use, it's also released under the GNU public license, which means that
it will always remain free to both use and to change.

Rockbox has been in development since 2001, and recieves new features, tweaks
and fixes each day to provide you with the best possible experience on your MP3
player. A major goal of Rockbox is to be simple and easy to use, yet remain very
customizable and configurable. We believe that you should never need to go
through a series of menus for an action you perform frequently. We also believe
that you should be able to configure almost anything about Rockbox you could
want, pertaining to functionality. Another top priority of Rockbox is audio
playback quality -- Rockbox, for most models, includes a wider range of sound
settings than that device's original firmware. A lot of work has been put into
making Rockbox sound the best it can, and improvements are constantly being made.
All models have access to a large number of plugins, including many games,
applications, and graphical ``demos''. You can load different configurations
quickly for different purposes (e.g. a large font for in your car, different
sound settings for at home). Rockbox features a very wide range of languages, and
all supported models also have the ability to talk to you -- menus can be voiced
and filenames spelled out or spoken.

\section{Getting more help}
This manual is intended to be a comprehensive introduction to the Rockbox
software. There is, however, more help available.  The Rockbox website at
\url{http://www.rockbox.org/} contains very extensive documentation and guides
written by members of the Rockbox community and this should be your first port
of call when looking for further help.

\opt{h1xx,h300}{% $Id$ %

  Installing the boot loader is the trickiest part of the installation.
  The Rockbox boot loader allows users to boot into either the Rockbox 
  firmware or the \playerman{} firmware. For legal reasons, we cannot distribute 
  the boot loader. Instead, we have developed a program that will patch the 
  Iriver firmware with the Rockbox boot loader. These instructions will explain 
  how to download and patch the Iriver firmware with the Rockbox boot loader 
  and install it on your jukebox.


\begin{enumerate}
  \item Download a supported version of the Iriver firmware for your 
  \playername{} from the Iriver website or from 
  \wikilink{ManualRockboxInstall}.
  Supported Iriver firmware versions currently include 
  \opt{IRIVER_H100_PAD}{1.63US, 1.63EU, 1.63K, 1.65US, 1.65EU, 1.65K, 1.66US, 
    1.66EU and 1.66K.  Note that the H140 uses the same firmware as the H120;
    H120 and H140 owners should use the	firmware called \fname{ihp\_120.hex}.
    Likewise, the iHP110 and iHP115 use the same firmware, called 
    \fname{ihp\_100.hex}.   Be sure to use the correct firmware file for 
    your player.}
  \opt{IRIVER_H300_PAD}{1.28K, 1.28EU, 1.28J, 1.29K, 1.29J and 1.30EU.
    \note{The US \playername{} firmware is not supported and cannot be
    patched to be used with the boot loader. If you wish to install Rockbox
    on a US \playername, you must first install a non-US version of the
    original firmware and then install one of the supported versions patched
    with the Rockbox bootloader. 
    \warn{Installing non-US firmware on a US \playername{} will
    permanently remove DRM support from the player.}}
  }%
  If the file that you downloaded is a \fname{.zip} file, use an unzip 
  utility like mentioned in the prerequisites section to extract
  the \fname{.hex} from the \fname{.zip} file
  to your desktop. Likewise, if the file that you downloaded is an 
  \fname{.exe} file, double-click on the \fname{.exe} file to extract 
  the \fname{.hex} file to your desktop.
  When running Linux you should be able extracting \fname{.exe}
  files using \fname{unzip}.
  %
  \item Download the firmware patcher \fname{fwpatcher.exe} from 
  \url{http://download.rockbox.org/bootloader/iriver/} and save it to your desktop.
    \warn{The firmware patcher contains Unicode support, which is not supported by 
    all versions of Windows. If you have difficulty with the firmware patcher, try 
    downloading the alternate firmware patcher \fname{fwpatchernu.exe}, which is 
    built without Unicode support.}
  %
  \item Go to your desktop and double-click on whichever version of the firmware 
  patcher you downloaded in the prior step.
  %
  \item In the firmware patcher dialog box, click on the \setting{Browse}
  button and navigate
  to the \fname{.hex} file that you previously downloaded to your desktop.
  %
  \item Click \setting{Patch}. The firmware patcher will patch the
    original firmware to include the Rockbox boot loader. The \fname{.hex}
    file on your desktop is now a modified version of the original
    \fname{.hex} file.
  %
  \item Turn on your \playerman{} and connect it to your computer via USB.
  %
  \item Copy or move the modified \fname{.hex} file directly to the root of
    your \daps{} drive. Do not put it inside a directory on your \dap.
  %
  \item Disconnect the jukebox from USB. (Be sure to use Windows' ``safely remove
  hardware'' option.)
  \warn{Before proceeding further, make sure that your player has a full charge
    or that it is connected to the power adaptor. Interrupting the next step
    due to a power failure most likely will brick your \dap{}.}
  %
  \item Update your \daps{} firmware with the patched boot loader. To do this, turn
    the jukebox on. Press and hold the
    \opt{IRIVER_H100_PAD,IRIVER_H300_PAD}{\ButtonSelect{} button }%
    to enter the main menu, and navigate to \setting{General $\rightarrow$ Firmware 
    Upgrade}. Select \setting{Yes} when asked to confirm if you want to upgrade the 
    firmware. The \playerman{} will display a message indicating that the
    firmware update 
    is in progress. Do \emph{not} interrupt this process. When the
    firmware update is complete the player will turn itself off. (The update
    firmware process usually takes a minute or so.)
    
    You have now installed the Rockbox boot loader.
\end{enumerate}

\note{If you install the Rockbox boot loader but do not install the
  Rockbox firmware the Rockbox boot loader will load the Iriver firmware when the
  jukebox is turned on.
  To load the \playerman{} firmware press and hold \ButtonRec{} before
  powering up the \dap{} until the \playerman{} logo appears.
  }

\note{The boot loader has a built-in ``boot loader USB mode''. This function
  switches to USB mode when the \dap{} is connected to a computer upon
  power-up. This way you can access the \daps{} hard disk without the need
  to boot any firmware (which is also useful when your hard disk is 
  damaged). The screen will simply display the text ``boot loader USB mode''.
  After you disconnect the \dap{} from USB the boot loader will
  continue booting Rockbox. As in boot loader USB mode the firmware
  itself hasn't been loaded this is also a simple way of updating Rockbox.
  After the disconnect the boot loader will load the updated version of
  Rockbox.
}
}

\section{Downloading Rockbox}
The latest release of the Rockbox software will always be available from
\url{http://www.rockbox.org/download/}.
\opt{MASCODEC}{
  Windows users may wish to download the self-extracting Windows installer,
  which works for all Jukebox models, but those wishing to install manually or
  using a different operating system should choose the .zip archive containing
  the firmware for their model of the Jukebox.
}

\section{Installing Rockbox}\label{sec:installing_rockbox}
\opt{MASCODEC}{
  \subsection{Using the windows installer}
  Using the Windows self installing executable to install Rockbox is the easiest
  method of installing the software on your Jukebox.  Simply follow the
  on-screen instructions and select the appropriate drive letter and Jukebox
  model when prompted.  You can use ``Add / Remove Programs'' to uninstall the
  software at a later date.

  \subsection{Manual installation}
  For non{}-Windows users and those wishing to install manually from the archive
  the procedure is still fairly simple.
}
\opt{SWCODEC}{
  \subsection{Installing the bootloader}
  Installing the bootloader is the trickiest part of the installation.
  \fixme{add note for other OS as windows. Point to the appropriate wiki page}
  Due to legal reasons the Rockbox project can not provide ready-to-use flash
  images. Instead, you need to get an original firmware from your \dap s
  manufacturer homepage and modify it to include the rockbox bootloader.
  \fixme{include further bootloader install instructions: iriver, ipod, iaudio}
  \fixme{add something about the bootloader, esp. the bootloader usb mode}
  \subsection{Installing the firmware}
  After installing the bootloader the installation becomes fairly easy.
}
  Connect your \playername\ to the
computer via USB as described in the manual that came with your \playername. On
Windows, the \playername\ drive will appear as a drive letter in your
``My Computer'' folder. Take the file that you downloaded above, and unpack
its contents to your \playername\ drive. You can do this using a program such
as \url{http://www.info-zip.org/} or \url{http://www.winzip.org/}.

You will need to unpack all of the files in the archive onto your hard disk. If
 this has been done correctly, you will have a file called 
\fname{\firmwarefilename} in the main folder of your \playername\ drive, and
also a folder called /\fname{.rockbox}, which contains a number of system files
used by the software.
\note{Please note that the firmware folder starts with a leading dot. You may
experience problems when trying to create such folders when using Windows.
Directly unzipping to your \dap's drive works flawlessly, only Windows'
explorer is limited handling with such files.}

\section{Enabling Speech Support (optional)}\label{sec:enabling_speech_support}
If you wish to use speech support you will also need a language file, available
from \url{http://www.rockbox.org/twiki/bin/view/Main/VoiceFiles/}.  For the
English language, the file is called \fname{english.voice}. When it has been
downloaded, unpack this file and copy it into the \fname{lang} folder which is
inside the /\fname{.rockbox} folder on your Jukebox. Voice menus are turned on
by default. See page \pageref{ref:Voiceconfiguration} for details on voice
settings.


\section{Running Rockbox}
Remove your Jukebox from the computer's USB port. Unplug any connected power
supply and turn the unit off. When you next turn the unit on, the Jukebox
firmware will start to load, and then it will load Rockbox for you. When you see
the Rockbox splash screen, Rockbox is loaded and ready for use.

\section{Uninstalling Rockbox}
If you would like to go back to using the original \playername\ software, then
connect the \playername\ to your computer, and delete the
\fname{\firmwarefilename} file. If you wish to clean up your disk, you may also
wish to delete the \fname{.rockbox} folder and its contents. Turn the
\playername\ off and on and the normal \playername\ software will load.

