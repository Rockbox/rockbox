% $Id$ %
\chapter{Getting started}
\section{Welcome}
This is the manual for Rockbox. Rockbox is an open source firmware replacement
for a growing number of MP3 players. Rockbox aims to be considerably more
functional and efficient than your device's stock firmware while remaining easy
to use and customizable. Rockbox is written by users, for users. Not only is it
free to use, it's also released under the GNU public license, which means that
it will always remain free to both use and to change.

Rockbox has been in development since 2001, and recieves new features, tweaks
and fixes each day to provide you with the best possible experience on your MP3
player. A major goal of Rockbox is to be simple and easy to use, yet remain very
customizable and configurable. We believe that you should never need to go
through a series of menus for an action you perform frequently. We also believe
that you should be able to configure almost anything about Rockbox you could
want, pertaining to functionality. Another top priority of Rockbox is audio
playback quality -- Rockbox, for most models, includes a wider range of sound
settings than that device's original firmware. A lot of work has been put into
making Rockbox sound the best it can, and improvements are constantly being made.
All models have access to a large number of plugins, including many games,
applications, and graphical ``demos''. You can load different configurations
quickly for different purposes (e.g. a large font for in your car, different
sound settings for at home). Rockbox features a very wide range of languages, and
all supported models also have the ability to talk to you -- menus can be voiced
and filenames spelled out or spoken.

\section{Getting more help}
This manual is intended to be a comprehensive introduction to the Rockbox
software. There is, however, more help available.  The Rockbox website at
\url{http://www.rockbox.org/} contains very extensive documentation and guides
written by members of the Rockbox community and this should be your first port
of call when looking for further help.

\section{Installing Rockbox}\label{sec:installing_rockbox}
\opt{MASCODEC}{
  \subsection{Using the windows installer}
  Using the Windows self installing executable to install Rockbox is the easiest
  method of installing the software on your Jukebox.  Simply follow the
  on-screen instructions and select the appropriate drive letter and Jukebox
  model when prompted.  You can use ``Add / Remove Programs'' to uninstall the
  software at a later date.

  \subsection{Manual installation}
  For non{}-Windows users and those wishing to install manually from the archive
  the procedure is still fairly simple.
}
\opt{SWCODEC}{
\subsection{Introduction}
There are two separate components of Rockbox that need to be installed in order
to run Rockbox.
\begin{enumerate}
\item The Rockbox bootloader. This is the component of Rockbox that is installed
  to the flash memory of your \playerman. The bootloader is the program that tells
  your \dap\ how to boot and load other components of Rockbox.
\item The Rockbox firmware. Unlike the \playerman\ firmware, which runs entirely
  from flash memory, most of the Rockbox code is contained in the build that
  resides on your jukebox's hard drive. This makes it easy to update Rockbox. The
  build contain a file named \firmwarefilename\ and a directory called
  \fname{.rockbox} which are located in the root directory of your  hard drive.
\end{enumerate}
\opt{h1xx,h300}{% $Id$ %

  Installing the boot loader is the trickiest part of the installation.
  The Rockbox boot loader allows users to boot into either the Rockbox 
  firmware or the \playerman{} firmware. For legal reasons, we cannot distribute 
  the boot loader. Instead, we have developed a program that will patch the 
  Iriver firmware with the Rockbox boot loader. These instructions will explain 
  how to download and patch the Iriver firmware with the Rockbox boot loader 
  and install it on your jukebox.


\begin{enumerate}
  \item Download a supported version of the Iriver firmware for your 
  \playername{} from the Iriver website or from 
  \wikilink{ManualRockboxInstall}.
  Supported Iriver firmware versions currently include 
  \opt{IRIVER_H100_PAD}{1.63US, 1.63EU, 1.63K, 1.65US, 1.65EU, 1.65K, 1.66US, 
    1.66EU and 1.66K.  Note that the H140 uses the same firmware as the H120;
    H120 and H140 owners should use the	firmware called \fname{ihp\_120.hex}.
    Likewise, the iHP110 and iHP115 use the same firmware, called 
    \fname{ihp\_100.hex}.   Be sure to use the correct firmware file for 
    your player.}
  \opt{IRIVER_H300_PAD}{1.28K, 1.28EU, 1.28J, 1.29K, 1.29J and 1.30EU.
    \note{The US \playername{} firmware is not supported and cannot be
    patched to be used with the boot loader. If you wish to install Rockbox
    on a US \playername, you must first install a non-US version of the
    original firmware and then install one of the supported versions patched
    with the Rockbox bootloader. 
    \warn{Installing non-US firmware on a US \playername{} will
    permanently remove DRM support from the player.}}
  }%
  If the file that you downloaded is a \fname{.zip} file, use an unzip 
  utility like mentioned in the prerequisites section to extract
  the \fname{.hex} from the \fname{.zip} file
  to your desktop. Likewise, if the file that you downloaded is an 
  \fname{.exe} file, double-click on the \fname{.exe} file to extract 
  the \fname{.hex} file to your desktop.
  When running Linux you should be able extracting \fname{.exe}
  files using \fname{unzip}.
  %
  \item Download the firmware patcher \fname{fwpatcher.exe} from 
  \url{http://download.rockbox.org/bootloader/iriver/} and save it to your desktop.
    \warn{The firmware patcher contains Unicode support, which is not supported by 
    all versions of Windows. If you have difficulty with the firmware patcher, try 
    downloading the alternate firmware patcher \fname{fwpatchernu.exe}, which is 
    built without Unicode support.}
  %
  \item Go to your desktop and double-click on whichever version of the firmware 
  patcher you downloaded in the prior step.
  %
  \item In the firmware patcher dialog box, click on the \setting{Browse}
  button and navigate
  to the \fname{.hex} file that you previously downloaded to your desktop.
  %
  \item Click \setting{Patch}. The firmware patcher will patch the
    original firmware to include the Rockbox boot loader. The \fname{.hex}
    file on your desktop is now a modified version of the original
    \fname{.hex} file.
  %
  \item Turn on your \playerman{} and connect it to your computer via USB.
  %
  \item Copy or move the modified \fname{.hex} file directly to the root of
    your \daps{} drive. Do not put it inside a directory on your \dap.
  %
  \item Disconnect the jukebox from USB. (Be sure to use Windows' ``safely remove
  hardware'' option.)
  \warn{Before proceeding further, make sure that your player has a full charge
    or that it is connected to the power adaptor. Interrupting the next step
    due to a power failure most likely will brick your \dap{}.}
  %
  \item Update your \daps{} firmware with the patched boot loader. To do this, turn
    the jukebox on. Press and hold the
    \opt{IRIVER_H100_PAD,IRIVER_H300_PAD}{\ButtonSelect{} button }%
    to enter the main menu, and navigate to \setting{General $\rightarrow$ Firmware 
    Upgrade}. Select \setting{Yes} when asked to confirm if you want to upgrade the 
    firmware. The \playerman{} will display a message indicating that the
    firmware update 
    is in progress. Do \emph{not} interrupt this process. When the
    firmware update is complete the player will turn itself off. (The update
    firmware process usually takes a minute or so.)
    
    You have now installed the Rockbox boot loader.
\end{enumerate}

\note{If you install the Rockbox boot loader but do not install the
  Rockbox firmware the Rockbox boot loader will load the Iriver firmware when the
  jukebox is turned on.
  To load the \playerman{} firmware press and hold \ButtonRec{} before
  powering up the \dap{} until the \playerman{} logo appears.
  }

\note{The boot loader has a built-in ``boot loader USB mode''. This function
  switches to USB mode when the \dap{} is connected to a computer upon
  power-up. This way you can access the \daps{} hard disk without the need
  to boot any firmware (which is also useful when your hard disk is 
  damaged). The screen will simply display the text ``boot loader USB mode''.
  After you disconnect the \dap{} from USB the boot loader will
  continue booting Rockbox. As in boot loader USB mode the firmware
  itself hasn't been loaded this is also a simple way of updating Rockbox.
  After the disconnect the boot loader will load the updated version of
  Rockbox.
}
}
\opt{ipod4g,ipodcolor,ipodnano,ipodmini,ipodvideo}
	{% $Id$ %

\opt{ipodvideo}{\newcommand{\bootloaderfile}{bootloader-ipodvideo.ipod}}%
\opt{ipodmini}{\newcommand{\bootloaderfile}{bootloader-ipodmini.ipod}}%
\opt{ipodnano}{\newcommand{\bootloaderfile}{bootloader-ipodnano.ipod}}%
\opt{ipodcolor}{\newcommand{\bootloaderfile}{bootloader-ipodcolor.ipod}}%
\opt{ipod4g}{\newcommand{\bootloaderfile}{bootloader-ipod4g.ipod}}%
\opt{ipod3g}{\newcommand{\bootloaderfile}{bootloader-ipod3g.ipod}}%
%
\opt{ipodnano}{\warn{If your Nano has a stainless steel back and plastic front 
it is a 1st generation and is compatible with Rockbox.  If, on the other hand, 
your Nano has a one-piece aluminum body it is a 2nd generation Nano and there 
is currently no Rockbox port available.  Do not attempt to install the 
bootloader on a 2nd generation Nano}}

In order to make your Ipod load and execute the Rockbox firmware you have just 
installed, you will need to install the Rockbox bootloader. Unless bugs are 
found in the bootloader code, or significant new feature are added, you will 
only have to perform this step once.

The following instructions refer to the ``installation folder.''  For Windows 
users, the ``installation folder'' is a folder in the root (top-level) of the C: 
drive called \fname{\textbackslash{}rockbox} (you will obviously need to create 
this folder yourself).  For Mac OS X and Linux users, the ``installation 
folder'' is assumed to be the Desktop folder.  Note that the bootloader 
installation files should be saved onto your computer's hard disk, \emph{not} on 
your Ipod. 

\begin{enumerate} 

  \item First, download the \fname{ipodpatcher} tool to your installation 
  folder.  You can download the \fname{ipodpatcher} tool for your operating 
  system at \download{bootloader/ipod/ipodpatcher/}.
  
  \item Next, download the following file to the installation folder: 

    \download{bootloader/ipod/\bootloaderfile}
    \opt{ipodmini}{%
        or \download{bootloader/ipod/bootloader-ipodmini2g.ipod}
        depending on which generation your \dap{} is.
        The following page describes the differences between the two
        generations of the \dap{}: 
        \url{http://docs.info.apple.com/article.html?artnum=300850}.
    }

  \item Next, open a command prompt (Windows) or terminal window (Mac OSX and Linux).
  
    Windows users will perform this and the following steps from the Windows 
    command prompt.  To start a command prompt, click \fname{start}, and then 
    click \fname{Run...}.  Type ``cmd'' and press \fname{Enter}.  Navigate 
    to the installation directory by typing the following command:

    \begin{code} 
        cd \textbackslash{}rockbox
      \end{code}
      
    Mac OS X and Linux/Unix users will perform these steps from the Terminal. 
    Start a new terminal window and navigate to the Desktop folder (type cd 
    Desktop into the terminal and press enter). You then need to ensure that the 
    ipodpatcher program is ``executable'' by typing the command chmod +x 
    ipodpatcher and then pressing \fname{Enter}.
  
  \item Connect your Ipod to your computer.

    If you haven't already done so, you should now plug your Ipod into your 
    computer (via either the USB or Firewire cable).

    \fixme{Notes about closing itunes, enabling the ``show ipod as disk'' option 
    in ipod, anything else?}

  \item Find your Ipod with ipodpatcher (Windows users only)

    Windows users:  Type the following command to search for Ipods attached to 
    your computer: 
      \begin{code} 
        ipodpatcher --scan 
      \end{code}
    
    When ipodpatcher finds your Ipod, remember the number it displays after the 
    words ``disk device''- this will  be the number you use to access your Ipod 
    in the following steps.  So, for example, if ipodpatcher displays ``disk 
    device 1'' you will use the number 1 in the commands described below.

    \note{Windows users require administrator rights for running ipodpatcher. 
    Either re-login as administrator, or open a command prompt running under an 
    administrator account by using one of the "Run as" features of Windows XP.}

  \item Find your Ipod (Mac OS X users only)

    Attach your Ipod to your Mac (using either USB or Firewire) and wait for 
    iTunes to open. When iTunes opens, close it down.  In your Terminal window, 
    type the command mount and press enter. This will list all the disks (and 
    other devices) that are "mounted" on your computer. The last  drive in the 
    list should be your Ipod. For example: 
    \begin{code}
       /dev/disk1s2 on /Volumes/DAVE_S IPOD 1 (local, nodev, nosuid) 
    \end{code}

    In order to install the ipod bootloader, you need to ``unmount'' this disk 
    using the following command: 
      \begin{code} 
        diskutil unmount /dev/disk1s2 
      \end{code}
      
    replacing ``/dev/disk1s2'' with the device name Mac OS has assigned to your
    Ipod. This may take a few seconds, after which Mac OS will say ``Volume 
    /dev/disk1s2 unmounted.'' ``/dev/disk1s2'' refers to the second partition on 
    /dev/disk1 - remember   ``/dev/disk1'' for the next step.

    It's possible that itunes will try to be ``helpful'' and remount your Ipod 
    after you modify it with ipodpatcher. If this happens, you need to unmount 
    it again using the above command. 
  
  \item Create a backup of your Ipod's firmware partition

    Type the following command, replacing ``N'' with the number (for 
    Windows users) or the device name (Mac OS X and Unix users) assigned to 
    your Ipod that you identified in the previous step: 
      \begin{code} 
        ipodpatcher N -r bootpartition.bin (Windows) 
      \end{code}
      or
      \begin{code}
        ./ipodpatcher N -r bootpartition.bin (Mac OS X/Unix)
      \end{code}
  
    This should create a file in the current folder called 
    \fname{bootpartition.bin} (approximately 40MB for the iPod 3G, 4G and 
    Color/Photo, 80MB for the Nano 1st gen and 30GB Video, and 112MB for the 
    60GB Video) containing a copy of the ``firmware partition'' from your Ipod.

    If it ever becomes necessary (for example, if your Ipod refuses to start), 
    you can restore this backup to your Ipod using the command ipodpatcher N -w 
    bootpartition.bin (Windows) or ./ipodpatcher N -w bootpartition.bin (Mac OS 
    X/Unix).   

    \opt{ipodmini}{
      \note{Ipod Mini 2g users need to replace ``1g'' with ``2g'' in the
      following commands.} 
    }
 
  \item Install the bootloader.
    Windows users should now type:
    \begin{code}
      ipodpatcher N -a \bootloaderfile
    \end{code}
    %

    and Mac OS X/Unix users should type:

    \begin{code}
      ./ipodpatcher N -a \bootloaderfile
    \end{code}

  Replace N with the number (Windows users) or device name (Mac OS X/Unix 
  users) you've been using to access your Ipod.  
  
  You can now disconnect your Ipod from your computer in the normal way. This 
  should cause your Ipod to reboot and start Rockbox.
  
  \note{If your Ipod displays the message ``Error: -1,'' you have either 
  neglected to install a Rockbox build as described in the preceding section, 
  or you have extracted the contents of the \fname{.zip} file to some 
  directory other than the the root directory of your Ipod.  To fix this 
  error, following the directions in the preceding section for downloading and 
  installing a Daily Build.}
  
\end{enumerate} 
}
\opt{x5}{The \playername{} has a built-in bootloader which performs the
firmware update and can also access the hard drive via USB.  The
Rockbox bootloader can therefore be very minimalistic, as it does not require
 it's own USB mode.  This makes it less dangerous to install the Rockbox bootloader
 as you can always restore it using the \playerman{} bootloader.

\note{The Rockbox bootloader overwrites the original firmware, making it
   impossible to dual-boot.}

\subsubsection{Installation}
\begin{itemize}
\item Download the Rockbox bootloader binary from 
\url{http://download.rockbox.org/bootloader/iaudio/}.
  \opt{x5}{Use the \fname{x5v\_fw.bin} file if your \dap{} is a X5V. If it is a X5
    or X5L, use the \fname{x5\_fw.bin} file.}
  \opt{m5}{Use the \fname{m5\_fw.bin} file.}
\item Copy it to the \fname{FIRMWARE} directory on your \dap{}.
\item Turn the \dap{} off, remove the USB cable and insert the charger. The
Rockbox bootloader will automatically be flashed.
\end{itemize}
}
   
  \subsection{Installing the firmware}
  After installing the bootloader the installation becomes fairly easy.
}

Connect your \playername\ to the computer via USB as described in the 
manual that came with your \playername. On Windows, the \playername\ drive 
will appear as a drive letter in your ``My Computer'' folder. Take the file 
that you downloaded above, and unpack its contents to your \playername\ drive.
You can do this using a program such as \url{http://www.info-zip.org/} or 
\url{http://www.winzip.org/}.

You will need to unpack all of the files in the archive onto your hard disk. If 
this has been done correctly, you will have a file called 
\fname{\firmwarefilename} in the main folder of your \playername\ drive, and
also a folder called /\fname{.rockbox}, which contains a number of system files
used by the software.
\note{Please note that the firmware folder starts with a leading dot. You may
experience problems when trying to create such folders when using Windows.
Directly unzipping to your \dap's drive works flawlessly; it is only Windows'
Explorer that is limited in handling such files.}

\section{Enabling Speech Support (optional)}\label{sec:enabling_speech_support}
If you wish to use speech support you will also need a language file, available
from \url{http://www.rockbox.org/twiki/bin/view/Main/VoiceFiles/}.  For the
English language, the file is called \fname{english.voice}. When it has been
downloaded, unpack this file and copy it into the \fname{lang} folder which is
inside the /\fname{.rockbox} folder on your Jukebox. Voice menus are turned on
by default. See page \pageref{ref:Voiceconfiguration} for details on voice
settings.

\section{Running Rockbox}
Remove your Jukebox from the computer's USB port. Unplug any connected power
supply and turn the unit off. When you next turn the unit on, the Jukebox
firmware will start to load, and then it will load Rockbox for you. When you see
the Rockbox splash screen, Rockbox is loaded and ready for use.

\section{Uninstalling Rockbox}
If you would like to go back to using the original \playername\ software, then
connect the \playername\ to your computer, and delete the
\fname{\firmwarefilename} file. If you wish to clean up your disk, you may also
wish to delete the \fname{.rockbox} folder and its contents. Turn the
\playername\ off and on and the normal \playername\ software will load.

\section{Updating Rockbox}
The latest release of the Rockbox software will always be available from
\url{http://www.rockbox.org/download/}.
\opt{MASCODEC}{
  Windows users may wish to download the self-extracting Windows installer,
  which works for all Jukebox models, but those wishing to install manually or
  using a different operating system should choose the .zip archive containing
  the firmware for their model of the Jukebox.
}