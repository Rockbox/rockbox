% $Id$ %
\chapter{Installation}\label{sec:installation}

\opt{ipodvideo}{ 
  \note{Rockbox presently runs only on the original Ipod Video 30GB and 60GB, 
  and on the newer 30GB Ipod Video (sometimes referred to as the ``5.5G''). 
  Rockbox does \emph{not} run on the 80GB Ipod Video) (sometimes known as the 
  ``5.5G'').  For information on identifying which Ipod you own, see this page 
  on Apple's web site:  \url{http://www.info.apple.com/kbnum/n61688}
  }
}
  
\opt{ipodnano}{ 
  \note{Rockbox presently runs only on the original Ipod Nano. Rockbox does 
  \emph{not} run on the newer, second generation Ipod Nano (the all alumminum 
  verion).  For information on identifying which Ipod you own, see this page on 
  Apple's web site:  \url{http://www.info.apple.com/kbnum/n61688}
  }
}

\section{Prerequisites}\label{sec:prerequisites}
\index{Installation!Prerequisites}
Before installing Rockbox you should make sure you meet the prerequisites.
Also you may need some tools for installation. In most cases these will be
already available on your computer but if not you need to get some additional
software.

\begin{description}

\item[ZIP utility.]\index{zip}
  Rockbox is distributed as an archive using the
  \fname{.zip} format. Thus you need a tool to handle that compressed
  format. Usually your computer should have a tool installed that can
  handle the \fname{.zip} file format. Windows XP has built-in support for
  \fname{.zip} files and presents them to you as folders unless you have
  installed a third party program that handles compressed files. For
  other operating systems this may vary. If the \fname{.zip} file format
  is not recognised on your computer you can find a program to handle them
  at \url{http://www.info-zip.org/} or \url{http://sevenzip.sf.net/} which
  can be downloaded and used free of charge.

\item[USB connection.]  To transfer Rockbox to your \dap{} you need to
  connect it to your computer. To proceed you need to know where to access the
  \dap{}. On Windows this means you need to figure out the drive letter
  associated with the device. On Linux you need to know the mount point of
  your \dap{}.

  \opt{ipod}{
    Your \dap{} should enter disk mode automatically when connected to a 
    computer via USB. If your computer does not recognise your \dap{}, you may 
    need to enter the disk mode manually. Disconnect your \dap{} from the 
    computer. Reset the \dap{} by pressing and holding the \ButtonMenu{} and 
    \ButtonSelect{} buttons simultaneously. As soon as the \dap{} resets, press 
    and hold the \ButtonMenu{} and \ButtonPlay{} buttons simultaneously. Your 
    \dap{} should enter disk mode, and you can try reconnecting to the computer. 
  }  
         
  \opt{ipod3g,ipod4g,ipodcolor,ipodmini}{
    \note{\index{Firewire}Firewire detection is not supported in Rockbox at 
    the moment. Please use USB only.} 
  }
    
\item[Text editor.]  As you will see in the following chapters, Rockbox is
  highly configurable. In addition to saving configurations within Rockbox,
  Rockbox also allows you to create customised configuration files. If you
  would like to edit custom configuration files on your computer, you will
  need a text editor like Windows' ``Wordpad''.

\end{description}

\opt{ipod}{
  \note{In addition to the requirements described above, Rockbox only works on 
  Ipods formatted with the FAT32 filesystem (i.e., Ipods initialised by iTunes 
  for Windows). It does not work with the HFS+ filesystem (i.e. Ipods 
  initialised by iTunes for the Mac). More information and instructions for 
  converting an Ipod to FAT32 can be found on the 
  \url{http://www.rockbox.org/twiki/bin/view/Main/IpodConversionToFAT32} wiki 
  page on the Rockbox web site.  Note that after conversion, you can still use 
  a FAT32 Ipod on a Mac. 
  } 
}

\section{Installing Rockbox}\label{sec:installing_rockbox}
\index{Installation}
\opt{MASCODEC}{
  \subsection{Using the windows installer}
  Using the Windows self installing executable to install Rockbox is the
  easiest method of installing the software on your \dap{}. Simply follow the
  on-screen instructions and select the appropriate drive letter and
  \dap{}-model when prompted. You can use ``Add / Remove Programs'' to
  uninstall the software at a later date.

  \subsection{Manual installation}
  For non{}-Windows users and those wishing to install manually from the
  archive the procedure is still fairly simple.
}

\opt{SWCODEC}{
  \subsection{Introduction}

  \opt{HAVE_RB_BL_ON_DISK}{There are three separate components of Rockbox,
  two of which need to be installed in order to run Rockbox.}

  \opt{HAVE_RB_BL_IN_FLASH}{There are two separate components of Rockbox
  that need to be installed in order to run Rockbox.}

  \begin{description}
  \opt{HAVE_RB_BL_ON_DISK}{
  \item[The \playerman{} boot loader.]
    The \playerman{} boot loader is the program that tells your \dap{} how to boot
    and load the remaining firmware from disk. It is also responsible for the
    disk mode on your \dap{}.

    This boot loader is stored in special flash memory in your \playerman{}.
    It is already installed on your \dap{}, so it is never necessary to modify
    this in order to install Rockbox.}

  \item[The Rockbox boot loader.] \index{Boot loader}
    \opt{HAVE_RB_BL_ON_DISK}{The Rockbox boot loader is loaded from disk by
    the \playerman{} boot loader. It is responsible for loading the Rockbox
    firmware and for providing the dual boot function. It directly replaces the
    \playerman{} firmware on the \daps{} disk.}
    \opt{HAVE_RB_BL_IN_FLASH}{
    The boot loader is the program that tells your
    \dap{} how to boot and load other components of Rockbox. This is the
    component of Rockbox that is installed to the flash memory of your
    \playerman.}

  \item[The Rockbox firmware.]
    \opt{HAVE_RB_BL_IN_FLASH}{Unlike the \playerman{} firmware, which runs
    entirely from flash memory, }
    \opt{HAVE_RB_BL_ON_DISK}{Similar to the \playerman{} firmware, }
    most of the Rockbox code is contained in a
    ``build'' that resides on your \daps{} hard drive. This makes it easy to
    update Rockbox. The build consists of a file named \firmwarefilename{} and a
    directory called \fname{.rockbox}, both of which are located in the root
    directory of your hard drive.

  \end{description}

  \subsection{Installing the firmware}
}

There are three different types of firmware binaries from Rockbox website:
\label{Version}
current version, daily build and CVS build. You need to decide which one
you want to install and get the version for your \dap{}.

\begin{description}

\item[Current Version.] The current version is the latest stable release, free
  of known critical bugs.  The current stable release of Rockbox, version 2.5,
  is available at \url{http://www.rockbox.org/download/}.
  \opt{SWCODEC}{
    \note{The current stable release is available only for Archos jukeboxes.
      There has not yet been a stable release for the \playername{}.  Until
      there is a stable release for \playername{}, use a daily build or CVS
      build.
    }
  }

\item[Daily Build.] The daily build is a development version of Rockbox. It
  contains features and patches developed since last stable version.  It
  may also contain bugs! This daily build is generated automatically every day
  and can be found at \url{http://www.rockbox.org/daily.shtml}.

\item[CVS Build (formerly, ``Bleeding Edge Build.'')] CVS stands for
  ``Concurrent Versions System.'' CVS is the system that Rockbox
  developers use to keep track of changes to the Rockbox source code. CVS
  builds are made automatically every time there is a change to the
  Rockbox source. These builds are for people who want to test the code
  that developers just checked in.

\end{description}

\nopt{player}{
  \note{\index{Installation!Fonts}
    Rockbox has a fonts package that is available at
    \url{http://www.rockbox.org/daily.shtml}. While the daily builds and CVS
    builds change frequently, the fonts package rarely changes.  Thus, the
    fonts package is not included in the daily builds and CVS builds. (The
    stable release, on the other hand, does not change, so fonts are
    included with the stable release.) When installing Rockbox for the
    first time, you should install the fonts package.
  }
}

Because daily builds and CVS builds are development versions which change
frequently, they may behave differently than described in this manual, or
they may introduce new (and maybe annoying) bugs. If you do not want to get
undefined behaviour from your \dap{} you should really stick to the current
stable release, if there is one for your \dap{}. If you want to help the
project development, you can try development builds and help by reporting
bugs. Just be aware that these are development builds that are  highly
functional, but not perfect!

After downloading the Rockbox package connect your \dap{} to the computer via
USB as described in the manual that came with your \dap{}. Take the file that
you downloaded above, and extract its contents to your \daps{} drive.

Use the ``Extract all'' command of your unzip program to extract the files in
the \fname{.zip} file onto your \dap{}. Note that the entire contents of the
\fname{.zip} file should be extracted directly to the root of your \daps{}
drive. Do not try to create a separate directory or folder on your \dap{} for
the Rockbox files! The \fname{.zip} file already contains the internal
directory structure that Rockbox needs.

\note{
  If the contents of the \fname{.zip} file are extracted correctly, you will
  have a file called \fname{\firmwarefilename} in the main folder of your
  \daps{} drive, and also a folder called \fname{/.rockbox}, which contains a
  number of other folders and system files needed by Rockbox. If you receive a
  ``-1'' error when you start Rockbox, you have not extracted the contents of
  the \fname{.zip} file to the proper location.
	}

\opt{SWCODEC}{
  \subsection{Installing the boot loader}
  \opt{h1xx,h300}{% $Id$ %

  Installing the boot loader is the trickiest part of the installation.
  The Rockbox boot loader allows users to boot into either the Rockbox 
  firmware or the \playerman{} firmware. For legal reasons, we cannot distribute 
  the boot loader. Instead, we have developed a program that will patch the 
  Iriver firmware with the Rockbox boot loader. These instructions will explain 
  how to download and patch the Iriver firmware with the Rockbox boot loader 
  and install it on your jukebox.


\begin{enumerate}
  \item Download a supported version of the Iriver firmware for your 
  \playername{} from the Iriver website or from 
  \wikilink{ManualRockboxInstall}.
  Supported Iriver firmware versions currently include 
  \opt{IRIVER_H100_PAD}{1.63US, 1.63EU, 1.63K, 1.65US, 1.65EU, 1.65K, 1.66US, 
    1.66EU and 1.66K.  Note that the H140 uses the same firmware as the H120;
    H120 and H140 owners should use the	firmware called \fname{ihp\_120.hex}.
    Likewise, the iHP110 and iHP115 use the same firmware, called 
    \fname{ihp\_100.hex}.   Be sure to use the correct firmware file for 
    your player.}
  \opt{IRIVER_H300_PAD}{1.28K, 1.28EU, 1.28J, 1.29K, 1.29J and 1.30EU.
    \note{The US \playername{} firmware is not supported and cannot be
    patched to be used with the boot loader. If you wish to install Rockbox
    on a US \playername, you must first install a non-US version of the
    original firmware and then install one of the supported versions patched
    with the Rockbox bootloader. 
    \warn{Installing non-US firmware on a US \playername{} will
    permanently remove DRM support from the player.}}
  }%
  If the file that you downloaded is a \fname{.zip} file, use an unzip 
  utility like mentioned in the prerequisites section to extract
  the \fname{.hex} from the \fname{.zip} file
  to your desktop. Likewise, if the file that you downloaded is an 
  \fname{.exe} file, double-click on the \fname{.exe} file to extract 
  the \fname{.hex} file to your desktop.
  When running Linux you should be able extracting \fname{.exe}
  files using \fname{unzip}.
  %
  \item Download the firmware patcher \fname{fwpatcher.exe} from 
  \url{http://download.rockbox.org/bootloader/iriver/} and save it to your desktop.
    \warn{The firmware patcher contains Unicode support, which is not supported by 
    all versions of Windows. If you have difficulty with the firmware patcher, try 
    downloading the alternate firmware patcher \fname{fwpatchernu.exe}, which is 
    built without Unicode support.}
  %
  \item Go to your desktop and double-click on whichever version of the firmware 
  patcher you downloaded in the prior step.
  %
  \item In the firmware patcher dialog box, click on the \setting{Browse}
  button and navigate
  to the \fname{.hex} file that you previously downloaded to your desktop.
  %
  \item Click \setting{Patch}. The firmware patcher will patch the
    original firmware to include the Rockbox boot loader. The \fname{.hex}
    file on your desktop is now a modified version of the original
    \fname{.hex} file.
  %
  \item Turn on your \playerman{} and connect it to your computer via USB.
  %
  \item Copy or move the modified \fname{.hex} file directly to the root of
    your \daps{} drive. Do not put it inside a directory on your \dap.
  %
  \item Disconnect the jukebox from USB. (Be sure to use Windows' ``safely remove
  hardware'' option.)
  \warn{Before proceeding further, make sure that your player has a full charge
    or that it is connected to the power adaptor. Interrupting the next step
    due to a power failure most likely will brick your \dap{}.}
  %
  \item Update your \daps{} firmware with the patched boot loader. To do this, turn
    the jukebox on. Press and hold the
    \opt{IRIVER_H100_PAD,IRIVER_H300_PAD}{\ButtonSelect{} button }%
    to enter the main menu, and navigate to \setting{General $\rightarrow$ Firmware 
    Upgrade}. Select \setting{Yes} when asked to confirm if you want to upgrade the 
    firmware. The \playerman{} will display a message indicating that the
    firmware update 
    is in progress. Do \emph{not} interrupt this process. When the
    firmware update is complete the player will turn itself off. (The update
    firmware process usually takes a minute or so.)
    
    You have now installed the Rockbox boot loader.
\end{enumerate}

\note{If you install the Rockbox boot loader but do not install the
  Rockbox firmware the Rockbox boot loader will load the Iriver firmware when the
  jukebox is turned on.
  To load the \playerman{} firmware press and hold \ButtonRec{} before
  powering up the \dap{} until the \playerman{} logo appears.
  }

\note{The boot loader has a built-in ``boot loader USB mode''. This function
  switches to USB mode when the \dap{} is connected to a computer upon
  power-up. This way you can access the \daps{} hard disk without the need
  to boot any firmware (which is also useful when your hard disk is 
  damaged). The screen will simply display the text ``boot loader USB mode''.
  After you disconnect the \dap{} from USB the boot loader will
  continue booting Rockbox. As in boot loader USB mode the firmware
  itself hasn't been loaded this is also a simple way of updating Rockbox.
  After the disconnect the boot loader will load the updated version of
  Rockbox.
}
}
  \opt{ipod}{% $Id$ %

\opt{ipodvideo}{\newcommand{\bootloaderfile}{bootloader-ipodvideo.ipod}}%
\opt{ipodmini}{\newcommand{\bootloaderfile}{bootloader-ipodmini.ipod}}%
\opt{ipodnano}{\newcommand{\bootloaderfile}{bootloader-ipodnano.ipod}}%
\opt{ipodcolor}{\newcommand{\bootloaderfile}{bootloader-ipodcolor.ipod}}%
\opt{ipod4g}{\newcommand{\bootloaderfile}{bootloader-ipod4g.ipod}}%
\opt{ipod3g}{\newcommand{\bootloaderfile}{bootloader-ipod3g.ipod}}%
%
\opt{ipodnano}{\warn{If your Nano has a stainless steel back and plastic front 
it is a 1st generation and is compatible with Rockbox.  If, on the other hand, 
your Nano has a one-piece aluminum body it is a 2nd generation Nano and there 
is currently no Rockbox port available.  Do not attempt to install the 
bootloader on a 2nd generation Nano}}

In order to make your Ipod load and execute the Rockbox firmware you have just 
installed, you will need to install the Rockbox bootloader. Unless bugs are 
found in the bootloader code, or significant new feature are added, you will 
only have to perform this step once.

The following instructions refer to the ``installation folder.''  For Windows 
users, the ``installation folder'' is a folder in the root (top-level) of the C: 
drive called \fname{\textbackslash{}rockbox} (you will obviously need to create 
this folder yourself).  For Mac OS X and Linux users, the ``installation 
folder'' is assumed to be the Desktop folder.  Note that the bootloader 
installation files should be saved onto your computer's hard disk, \emph{not} on 
your Ipod. 

\begin{enumerate} 

  \item First, download the \fname{ipodpatcher} tool to your installation 
  folder.  You can download the \fname{ipodpatcher} tool for your operating 
  system at \download{bootloader/ipod/ipodpatcher/}.
  
  \item Next, download the following file to the installation folder: 

    \download{bootloader/ipod/\bootloaderfile}
    \opt{ipodmini}{%
        or \download{bootloader/ipod/bootloader-ipodmini2g.ipod}
        depending on which generation your \dap{} is.
        The following page describes the differences between the two
        generations of the \dap{}: 
        \url{http://docs.info.apple.com/article.html?artnum=300850}.
    }

  \item Next, open a command prompt (Windows) or terminal window (Mac OSX and Linux).
  
    Windows users will perform this and the following steps from the Windows 
    command prompt.  To start a command prompt, click \fname{start}, and then 
    click \fname{Run...}.  Type ``cmd'' and press \fname{Enter}.  Navigate 
    to the installation directory by typing the following command:

    \begin{code} 
        cd \textbackslash{}rockbox
      \end{code}
      
    Mac OS X and Linux/Unix users will perform these steps from the Terminal. 
    Start a new terminal window and navigate to the Desktop folder (type cd 
    Desktop into the terminal and press enter). You then need to ensure that the 
    ipodpatcher program is ``executable'' by typing the command chmod +x 
    ipodpatcher and then pressing \fname{Enter}.
  
  \item Connect your Ipod to your computer.

    If you haven't already done so, you should now plug your Ipod into your 
    computer (via either the USB or Firewire cable).

    \fixme{Notes about closing itunes, enabling the ``show ipod as disk'' option 
    in ipod, anything else?}

  \item Find your Ipod with ipodpatcher (Windows users only)

    Windows users:  Type the following command to search for Ipods attached to 
    your computer: 
      \begin{code} 
        ipodpatcher --scan 
      \end{code}
    
    When ipodpatcher finds your Ipod, remember the number it displays after the 
    words ``disk device''- this will  be the number you use to access your Ipod 
    in the following steps.  So, for example, if ipodpatcher displays ``disk 
    device 1'' you will use the number 1 in the commands described below.

    \note{Windows users require administrator rights for running ipodpatcher. 
    Either re-login as administrator, or open a command prompt running under an 
    administrator account by using one of the "Run as" features of Windows XP.}

  \item Find your Ipod (Mac OS X users only)

    Attach your Ipod to your Mac (using either USB or Firewire) and wait for 
    iTunes to open. When iTunes opens, close it down.  In your Terminal window, 
    type the command mount and press enter. This will list all the disks (and 
    other devices) that are "mounted" on your computer. The last  drive in the 
    list should be your Ipod. For example: 
    \begin{code}
       /dev/disk1s2 on /Volumes/DAVE_S IPOD 1 (local, nodev, nosuid) 
    \end{code}

    In order to install the ipod bootloader, you need to ``unmount'' this disk 
    using the following command: 
      \begin{code} 
        diskutil unmount /dev/disk1s2 
      \end{code}
      
    replacing ``/dev/disk1s2'' with the device name Mac OS has assigned to your
    Ipod. This may take a few seconds, after which Mac OS will say ``Volume 
    /dev/disk1s2 unmounted.'' ``/dev/disk1s2'' refers to the second partition on 
    /dev/disk1 - remember   ``/dev/disk1'' for the next step.

    It's possible that itunes will try to be ``helpful'' and remount your Ipod 
    after you modify it with ipodpatcher. If this happens, you need to unmount 
    it again using the above command. 
  
  \item Create a backup of your Ipod's firmware partition

    Type the following command, replacing ``N'' with the number (for 
    Windows users) or the device name (Mac OS X and Unix users) assigned to 
    your Ipod that you identified in the previous step: 
      \begin{code} 
        ipodpatcher N -r bootpartition.bin (Windows) 
      \end{code}
      or
      \begin{code}
        ./ipodpatcher N -r bootpartition.bin (Mac OS X/Unix)
      \end{code}
  
    This should create a file in the current folder called 
    \fname{bootpartition.bin} (approximately 40MB for the iPod 3G, 4G and 
    Color/Photo, 80MB for the Nano 1st gen and 30GB Video, and 112MB for the 
    60GB Video) containing a copy of the ``firmware partition'' from your Ipod.

    If it ever becomes necessary (for example, if your Ipod refuses to start), 
    you can restore this backup to your Ipod using the command ipodpatcher N -w 
    bootpartition.bin (Windows) or ./ipodpatcher N -w bootpartition.bin (Mac OS 
    X/Unix).   

    \opt{ipodmini}{
      \note{Ipod Mini 2g users need to replace ``1g'' with ``2g'' in the
      following commands.} 
    }
 
  \item Install the bootloader.
    Windows users should now type:
    \begin{code}
      ipodpatcher N -a \bootloaderfile
    \end{code}
    %

    and Mac OS X/Unix users should type:

    \begin{code}
      ./ipodpatcher N -a \bootloaderfile
    \end{code}

  Replace N with the number (Windows users) or device name (Mac OS X/Unix 
  users) you've been using to access your Ipod.  
  
  You can now disconnect your Ipod from your computer in the normal way. This 
  should cause your Ipod to reboot and start Rockbox.
  
  \note{If your Ipod displays the message ``Error: -1,'' you have either 
  neglected to install a Rockbox build as described in the preceding section, 
  or you have extracted the contents of the \fname{.zip} file to some 
  directory other than the the root directory of your Ipod.  To fix this 
  error, following the directions in the preceding section for downloading and 
  installing a Daily Build.}
  
\end{enumerate} 
}
  \opt{x5}{The \playername{} has a built-in bootloader which performs the
firmware update and can also access the hard drive via USB.  The
Rockbox bootloader can therefore be very minimalistic, as it does not require
 it's own USB mode.  This makes it less dangerous to install the Rockbox bootloader
 as you can always restore it using the \playerman{} bootloader.

\note{The Rockbox bootloader overwrites the original firmware, making it
   impossible to dual-boot.}

\subsubsection{Installation}
\begin{itemize}
\item Download the Rockbox bootloader binary from 
\url{http://download.rockbox.org/bootloader/iaudio/}.
  \opt{x5}{Use the \fname{x5v\_fw.bin} file if your \dap{} is a X5V. If it is a X5
    or X5L, use the \fname{x5\_fw.bin} file.}
  \opt{m5}{Use the \fname{m5\_fw.bin} file.}
\item Copy it to the \fname{FIRMWARE} directory on your \dap{}.
\item Turn the \dap{} off, remove the USB cable and insert the charger. The
Rockbox bootloader will automatically be flashed.
\end{itemize}
}
  \opt{h10,h10_5gb}{\subsubsection{Installation}
\begin{enumerate}
  \item Download 
  \opt{iriverh10}{\url{http://download.rockbox.org/bootloader/iriver/H10_20GC.mi4}}
  \opt{iriverh10_5gb}{
    \begin{itemize}
      \item \url{http://download.rockbox.org/bootloader/iriver/H10.mi4} if your \dap{} is UMS or
      \item \url{http://download.rockbox.org/bootloader/iriver/H10_5GB-MTP/H10.mi4} if it is MTP.
    \end{itemize}}
  \item Connect your \archosplayertype{} to the computer using UMS mode and the UMS trick\opt{iriverh10_5gb}{ if necessary}.
  \item Rename the \opt{iriverh10}{\fname{H10\_20GC.mi4}}\opt{iriverh10_5gb}{\fname{H10.mi4}} file to \fname{OF.mi4} in the \fname{System} directory on your \archosplayertype{}.
    \opt{iriverh10_5gb}{\note{If you have a Pure model \archosplayertype{} (which does not have a FM radio) it is possible that this file will be called \fname{H10EMP.mi4} instead. If so, rename the \fname{H10.mi4} you downloaded in step 1 to \fname{H10EMP.mi4}.}}
    \note{You should keep a safe backup of this file for use if you ever wish to switch back to the \archosplayerman{} firmware.}
    \note{If you cannot see the \fname{System} directory, you will need to make sure your operating system is configured to show hidden files and directories.}

  \item Copy the \opt{iriverh10}{\fname{H10\_20GC.mi4}}\opt{iriverh10_5gb}{\fname{H10.mi4} (or \fname{H10EMP.mi4} if you have a \archosplayertype{} Pure)} file you downloaded to the System directory on your \dap{}.
\end{enumerate}
}
  \opt{gigabeat}{% $Id$

\begin{itemize}
\item Download the Rockbox bootloader from
  \url{http://download.rockbox.org/bootloader/gigabeat/}
\item Starting at the root directory of your player browse into the directory
  \fname{GBSYSTEM} and from that into the subdirectory \fname{FWIMG}.
  These directories are hidden. Make sure that you have configured your browser
  to show hidden files or you may be unable to see \fname{FWIMG}.
\item In that directory you'll find a file called \fname{FWIMG01.DAT}. This too
  may be hidden. Rename the file to \fname{FWIMG01.DAT.ORIG}. Make sure you
  spelled that name  correctly as it is needed for booting the \playerman{} firmware.
  \warn{If you do not complete this step then you will be unable to uninstall Rockbox
  without a copy of the original firmware from the original install CD.}
\item Now copy the file \fname{FWIMG01.DAT} you downloaded to that directory.
  Make sure the spelling is correct.
\end{itemize}}
}

\section{Enabling Speech Support (optional)}\label{sec:enabling_speech_support}
\index{Speech}\index{Installation!Optional Steps}
If you wish to use speech support you will also need a language file, available
from \wikilink{VoiceFiles}. For the English language, the file is called
\fname{english.voice}. When it has been downloaded, unpack this file and copy
it into the \fname{lang} folder which is inside the \fname{/.rockbox} folder on
your \dap{}. Voice menus are turned on by default. See
\reference{ref:Voiceconfiguration} for details on voice settings.

\section{Running Rockbox}
Remove your \dap{} from the computer's USB port.%
\nopt{ipod}{Unplug any connected power supply and turn the unit off. When
you next turn the unit on, Rockbox should load.}%
\opt{ipod}{Rebooting the Ipod by holding
  \opt{IPOD_4G_PAD}{\ButtonMenu{}+\ButtonSelect{}}%
  \opt{IPOD_3G_PAD}{\ButtonMenu{}+\ButtonPlay{}}
  for a couple of seconds until the \dap{} reboots. Now Rockbox should load.
}%
When you see the Rockbox splash screen, Rockbox is loaded and ready for
use.

\opt{ipod}{
  \note{
    Rockbox starts in the \setting{File Browser}. If you have loaded music onto
    your player using Itunes, you will not be able to see your music because
    Itunes changes your files' names and hides them in directories in the
    \fname{Ipod\_Control} folder. You can view files placed on your \dap{} by
    Itunes by initialising and using Rockbox's database. See
    \reference{ref:database} for more information.
  }
}

\section{Updating Rockbox} Updating Rockbox is easy. Download a Rockbox build.
(The latest release of the Rockbox software will always be available from
\url{http://www.rockbox.org/download/}). Unzip the build to the root directory
of your \dap{} like you did in the installation step before. If your unzip
program asks you whether to overwrite files, choose the ``Yes to all'' option.
The new build will be installed over your current build.

\note{
  Settings are stored on an otherwise-unused sector of your hard disk, not in
  any of the files contained in the Rockbox build. Therefore, generally
  speaking, installing a new build does \emph{not} reset Rockbox to its default
  settings. Be aware, however, that from time to time, a change is made to the
  Rockbox source code that \emph{does} cause settings to be reset to their
  defaults when a Rockbox build is updated. Thus it is recommended to save your
  settings using the \setting{Manage Settings} $\rightarrow$
  \setting{Write .cfg file} function before updating your Rockbox build so that
  you can easily restore the settings if necessary. For additional information
  on how to save, load, and reset Rockbox's settings, see
  \reference{ref:SystemOptions}.
}

\section{Uninstalling Rockbox}\index{Installation!uninstall}

If you would like to go back to using the original \playerman{} software, then
connect the \playerman{} to your computer, and delete the
\fname{\firmwarefilename} file.

\opt{h10,h10_5gb}{
  Next, put the \opt{h10}{\fname{H10\_20GC.mi4}}\opt{h10_5gb}{\fname{H10.mi4}}
  file backed up in the installation phase back into the \fname{System}
  directory on your \playertype{}, replacing the file that is there already. As
  in the installation, it may be necessary to first put your device into UMS
  mode.
}

\optv{ipod}{
  Next, open a command window (Windows) or a terminal window (Mac or Linux).
  Navigate to the folder you created when you downloaded the
  \fname{ipodpatcher} program you used to install the Rockbox boot loader.
  Type the following command:

  \begin{code}
    ipodpatcher -w \emph{N} bootpartition.bin
  \end{code}

  Remember that \emph{N} is the number that you found when you installed
  Rockbox on your \playerman{}.
}

If you wish to clean up your disk, you may also wish to delete the
\fname{.rockbox} folder and its contents. Turn the \playerman{} off.

\opt{h300}{Press and hold the \ButtonRec{} button.}

Turn the \dap{} back on and the original \playerman{} software will load.

\opt{h1xx}{
  \note{
    There's no need to remove the installed boot loader. If you want to remove
    it, simply flash an unpatched \playerman{} firmware. Be aware that doing so
    will also remove the boot loader USB mode. As that mode can come in quite
    handy (especially when having disk errors) it is recommended to keep the
    boot loader. It also gives you the possibility of trying Rockbox anytime
    later by simply installing the distribution files.
  }
}

\opt{h300}{
  \note{
    There's no need to remove the installed boot loader, although you if you
    retain the Rockbox boot loader, you will need to hold the \ButtonRec{}
    button each time you want to start the original firmware. If you want to
    remove it simply flash an unpatched \playerman{} firmware. Be aware that
    doing so will also remove the boot loader USB mode. As that mode can come in
    quite handy (especially when having disk errors), you may wish to keep the
    boot loader. It also gives you the possibility of trying Rockbox anytime
    later by simply installing a new build.
  }
}
