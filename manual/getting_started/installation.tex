% $Id$ %
\chapter{Installation}\label{sec:installation}

\opt{ipodnano}{
  \note{Rockbox presently runs only on the original Ipod Nano. Rockbox does
  \emph{not} run on the second, third, or fourth generation Ipod Nano.
  For information on identifying which Ipod you own, see this page on
  Apple's website: \url{http://www.info.apple.com/kbnum/n61688}.
  }
}
\opt{ipodvideo}{
  \note{Rockbox presently runs only on the 5th and 5.5th generation Ipod videos.
  Rockbox does \emph{not} run on the newer, 6th/Classic generation Ipod. 
  For information on identifying which Ipod you own, see this page on Apple's 
  website: \url{http://www.info.apple.com/kbnum/n61688}.
  }
}
\opt{e200,c200}{
  \note{Rockbox doesn't function on the newer v2 models.  They can be identified
  by checking the Sandisk firmware version number under 
  Settings $\rightarrow$ Info.  The v1
  firmware is named 01.xx.xx, while the v2 firmware begins with 03.}
}

\section{Overview}
There are two ways of installing Rockbox: automated and manual. While the
manual way is older, more tested and proven to work correctly, the
automated installation is based on a nice graphical application that does
almost everything for you. It is still important that you have
an overview of the installation process to be able to select the correct
installation options.

\opt{MASCODEC}{Rockbox itself comes as a single package. There is no need
  to install additional software for running Rockbox.}
\opt{swcodec} {
  \opt{HAVE_RB_BL_ON_DISK}{There are three separate components,
    two of which need to be installed in order to run Rockbox:}
  \opt{HAVE_RB_BL_IN_FLASH}{There are two separate components
    which need to be installed in order to run Rockbox:}

\begin{description}
\opt{HAVE_RB_BL_ON_DISK}{
\item[The \playerman{} bootloader.]
  The \playerman{} bootloader is the program that tells your \dap{} how to load
  and start the original firmware. It is also responsible for any emergency,
  recovery, or disk modes on your \dap{}.  This bootloader is stored in special flash
  memory in your \playerman{} and comes factory-installed. It is not necessary
  to modify this in order to install Rockbox.}

\item[The Rockbox bootloader.] \index{Bootloader}
  \opt{HAVE_RB_BL_ON_DISK}{The Rockbox bootloader is loaded from disk by
  the \playerman{} bootloader. It is responsible for loading the Rockbox
  firmware and for providing the dual boot function. It directly replaces the
  \playerman{} firmware in the \daps{} boot sequence.
  \opt{gigabeatf}{\note{Dual boot does not currently work on the Gigabeat.}}}

  \opt{HAVE_RB_BL_IN_FLASH}{
  The bootloader is the program that tells your
  \dap{} how to load and start other components of Rockbox. This is the
  component of Rockbox that is installed to the flash memory of your
  \playerman.}

\item[The Rockbox firmware.]
  \opt{HAVE_RB_BL_IN_FLASH}{Unlike the \playerman{} firmware, which runs
  entirely from flash memory,}
  \opt{HAVE_RB_BL_ON_DISK}{Similar to the \playerman{} firmware,}
  most of the Rockbox code is contained in a
  ``build'' that resides on your \daps{} drive. This makes it easy to
  update Rockbox. The build consists of a directory called
  \fname{.rockbox} which contains all of the Rockbox files, and is 
  located in the root of your \daps{} drive.

\end{description}
}

\nopt{player} {
    Apart from the required parts there are some addons you might be interested
    in installing.
    \begin{description}
    \item[Fonts.] Rockbox can load custom fonts. The fonts are
        distributed as a separate package and thus need to be installed
        separately. They are not required to run Rockbox itself but
        a lot of themes require the fonts package to be installed.

    \item[Themes.] The view of Rockbox can be customized by themes. Depending
        on your taste you might want to install additional themes to change
        the look of Rockbox.
    \end{description}
}

\section{Prerequisites}\label{sec:prerequisites}
\index{Installation!Prerequisites}
Before installing Rockbox you should make sure you meet the prerequisites.
You may need some additional tools for installation. In most cases these will already be available on your computer, but if not, installing some additional
software might be necessary.

\begin{description}
\item[USB connection.] To transfer Rockbox to your \dap{} you need to
  connect it to your computer. For manual installation/uninstallation, or 
  should autodetection fail during automatic installation, you need to know 
  where to access the \dap{}. On Windows this means you need to figure out 
  the drive letter associated with the \dap{}. On Linux you need to know the 
  mount point of your \dap{}.

  \opt{ipod}{
    Your \dap{} should enter disk mode automatically when connected to a
    computer via USB. If your computer does not recognise your \dap{}, you may
    need to enter the disk mode manually. Disconnect your \dap{} from the
    computer. Hard reset the \dap{} by pressing and holding the \ButtonMenu{} and
    \ButtonSelect{} buttons simultaneously. As soon as the \dap{} resets, press
    and hold the \ButtonSelect{} and \ButtonPlay{} buttons simultaneously. Your
    \dap{} should enter disk mode and you can try reconnecting to the computer.
  }

  \opt{ipod3g,ipod4g,ipodcolor,ipodmini}{
    \note{\index{Firewire}Firewire detection is not supported in Rockbox at 
    the moment. Please use USB only.} 
  }
  \opt{x5}{
    \note{When instructed to connect/disconnect the USB cable, always use
    the USB port through the subpack, not the side 'USB Host' port! The side port
    is intended to be used for USB OTG connections only (digital cameras, memory
    sticks, etc)!}
  }    

  \opt{sansa}{\warn{The following steps require you to change the setting in
    \setting{Settings $\rightarrow$ USB Mode} to \setting{MSC} from within the
    original firmware. Never extract files to your \dap{} while it is in
    recovery mode.}}
  \opt{h10,h10_5gb}{\warn{The following steps require you to use UMS mode and so
    may require use of the UMS trick as described in the bootloader installation
    section.}}
  \opt{gigabeatf}{\warn{During installation, do not connect your \dap{}
    using the cradle but plug the USB cable directly to the \dap{}.}}  
\end{description}
For manual installation and customization additional software is required.
\begin{description}
\item[ZIP utility.]\index{zip}
  Rockbox is distributed as a compressed archive using the
  \fname{.zip} format. Your computer will normally already have a means of
  handling such archive files. Windows XP has built-in support for
  \fname{.zip} files and presents them to you as directories unless you
  have installed a third party program that handles compressed files. For
  other operating systems this may vary. If the \fname{.zip} file format
  is not recognised on your computer you can find a program to handle them
  at \url{http://www.info-zip.org/} or \url{http://sevenzip.sf.net/}, both of
  which can be downloaded and used free of charge.
\item[Text editor.] As you will see in the following chapters, Rockbox is
  highly configurable. In addition to saving configurations,
  Rockbox also allows you to create customised configuration files. If you
  would like to edit custom configuration files on your computer, you will
  need a text editor like Windows' ``Wordpad''.
\end{description}

\opt{ipod}{
  \note{In addition to the requirements described above, Rockbox only works on
  Ipods formatted with the FAT32 filesystem (i.e. Ipods initialized by Itunes
  for Windows). It does not work with the HFS+ filesystem (i.e. Ipods
  initialized by Itunes for the Mac). More information and instructions for
  converting an Ipod to FAT32 can be found on the
  \wikilink{IpodConversionToFAT32} wiki
  page on the Rockbox website.  Note that after conversion, you can still use
  a FAT32 Ipod on a Mac.
  }
}

\section{Installing Rockbox}\label{sec:installing_rockbox}\index{Installation}

\opt{ipodvideo}{\warn{There are separate versions of Rockbox for the 30GB and 
  60GB/80GB models.  You must ensure you download the correct version for your 
  \dap{}.}}

\subsection{Automated Installation}

\opt{mrobe100}

To automatically install Rockbox, download the official installer and
housekeeping tool \textsc{Rockbox Utility}. It allows you to:
\begin{itemize}
\item Automatically install all needed components for using Rockbox
        (``Small Installation'')
\item Automatically install all suggested components (``Full Installation'')
\item Selectively install optional components
\item Install additional themes
\item Install voice files and generate talk clips
\item Uninstall all components you installed using Rockbox Utility
\end{itemize}
Prebuilt binaries for Windows, Linux and MacOS~X are
available at the \wikilink{RockboxUtility} wiki page.
\\*
\warn{When first starting \textsc{Rockbox Utility} run ``Autodetect'', 
found in the configuration dialog (File $\rightarrow$ Configure). Autodetection
can detect most player types. If autodetection fails or is unable to detect 
the mountpoint, make sure to enter the correct values. The mountpoint indicates
the location of the \dap{} in your filesystem. On Windows, this is the drive
letter the \dap{} gets assigned, on other systems this is a path in the
filesystem.}

\opt{ipodvideo}
    {\warn{Autodetection is unable to distinguish between the
    \playerman{} 30~GB and 60~GB / 80~GB models and defaults to the
    30~GB model. This will usually work but you might want to check the
    detected value, especially if you experience problems with Rockbox.}
}

\note{Rockbox Utility currently lacks some guiding messages. Please have a
        look at the manual installation instructions if you are stuck
        during installation.}

\subsection{Manual Installation}

\subsubsection{Choosing a Rockbox version}\label{sec:choosing_version}

There are three different types of firmware binaries available from the
Rockbox website:
\label{Version}
Release version, current build and daily build. You need to decide which one
you want to install and get the appropriate version for your \dap{}.

\begin{description}

\item[Release.]
  \opt{archos}{The release version is the latest stable release, free
      of known critical bugs. The current stable release of Rockbox, version
      2.5, is available at \url{http://www.rockbox.org/download/}.
  }
  \opt{swcodec}{
      There has not yet been a stable release for the \playername{}. Until
      there is one, use a current build.
  }

\item[Current Build.] The current build is built at each source code change to
  the Rockbox SVN repository and represents the current state of Rockbox
  development. This means that the build could contain bugs but most of
  the time is safe to use. You can download the current build from  
  \url{http://build.rockbox.org/}.

\item[Archived Build.] In addition to the release version and the current build,
  there is also an archive of daily builds available for download. These are
  built once a day from the latest source code in the SVN repository. You can
  download archived builds from \url{http://www.rockbox.org/daily.shtml}.

\end{description}

\warn{Because current builds and daily builds are development versions which 
      change frequently, they may behave differently than described in this manual, 
      or they may introduce new (and maybe annoying) bugs. If you do not want to 
      get undefined behaviour from your \dap{}, you should stick to the current 
      stable release if there is one for your \dap{}. If you want to help with 
      project development, you can try development builds and help by reporting bugs. 
      Just be aware that these are development builds that are highly functional 
      but not perfect!}

\subsubsection{Installing the firmware}\label{sec:installing_firmware}

\begin{enumerate}

\item Download your chosen version of Rockbox from the links in the
  previous section.

\item Connect your \dap{} to the computer via USB
  \opt{ipod3g,ipod4g,ipodmini,ipodcolor}{ or Firewire} as described in
  the manual that came with your \dap{}.

\item Take the \fname{.zip} file that you downloaded and use
 the ``Extract all'' command of your unzip program to extract 
 the files onto your \dap{}.

\note{The entire contents of the \fname{.zip} file should be extracted 
directly to the root of your \daps{} drive. Do not try to
create a separate directory on your \dap{} for the Rockbox
files! The \fname{.zip} file already contains the internal
structure that Rockbox needs.}

\end{enumerate}

\opt{archos}{
  \note{
    If the contents of the \fname{.zip} file are extracted correctly, you will
    have a file called \fname{\firmwarefilename} in the main directory of your
    \daps{} drive, and also a directory called \fname{.rockbox}, which contains a
    number of other directories and system files needed by Rockbox. If you receive a
     ``-1'' error when you start Rockbox, you have not extracted the contents of
    the \fname{.zip} file to the proper location.
  }
}

% This has nothing to do with swcodec, just that these players need our own
% bootloader so we can decide where we want the main binary.
\opt{swcodec}{
  \note{
    If the contents of the \fname{.zip} file are extracted correctly, you will
    have a directory called \fname{.rockbox}, which contains all the files needed
    by Rockbox, in the main directory of your \daps{} drive. If you receive a
    ``-1'' error when you start Rockbox, you have not extracted the contents of
    the \fname{.zip} file to the proper location.
  }
}

\nopt{player}{
  \subsubsection{Installing the fonts package}{\index{Installation!Fonts}\label{sec:installing_fonts}
    Rockbox has a fonts package that is available at
    \url{http://www.rockbox.org/daily.shtml} or from the \emph{extras} link in
    the menu on the Rockbox website. While the current builds and
    daily builds change frequently, the fonts package rarely changes and is
	therefore not included in these builds.  When installing Rockbox for the first
	time, you should install the fonts package. The release version, on the other hand,
	does not change, so fonts are included with it.

  \begin{enumerate}

  \item Download the fonts package from the link above.

  \item Take the file that you downloaded above, and use the ``Extract
    all'' command of your unzip program to extract the files in the
    \fname{.zip} file onto your \dap{}. As with the firmware installation, the
    entire contents of the fonts \fname{.zip} should be extracted directly to the
    root of your \daps{} drive. Do not try to create a separate directory
    on your \dap{} for the fonts! The \fname{.zip} already contains the
    correct internal structure.

  \end{enumerate}
  }
}

\opt{swcodec}{
  \subsubsection{Installing the bootloader}
  \opt{h1xx,h300}{% $Id$ %

  Installing the boot loader is the trickiest part of the installation.
  The Rockbox boot loader allows users to boot into either the Rockbox 
  firmware or the \playerman{} firmware. For legal reasons, we cannot distribute 
  the boot loader. Instead, we have developed a program that will patch the 
  Iriver firmware with the Rockbox boot loader. These instructions will explain 
  how to download and patch the Iriver firmware with the Rockbox boot loader 
  and install it on your jukebox.


\begin{enumerate}
  \item Download a supported version of the Iriver firmware for your 
  \playername{} from the Iriver website or from 
  \wikilink{ManualRockboxInstall}.
  Supported Iriver firmware versions currently include 
  \opt{IRIVER_H100_PAD}{1.63US, 1.63EU, 1.63K, 1.65US, 1.65EU, 1.65K, 1.66US, 
    1.66EU and 1.66K.  Note that the H140 uses the same firmware as the H120;
    H120 and H140 owners should use the	firmware called \fname{ihp\_120.hex}.
    Likewise, the iHP110 and iHP115 use the same firmware, called 
    \fname{ihp\_100.hex}.   Be sure to use the correct firmware file for 
    your player.}
  \opt{IRIVER_H300_PAD}{1.28K, 1.28EU, 1.28J, 1.29K, 1.29J and 1.30EU.
    \note{The US \playername{} firmware is not supported and cannot be
    patched to be used with the boot loader. If you wish to install Rockbox
    on a US \playername, you must first install a non-US version of the
    original firmware and then install one of the supported versions patched
    with the Rockbox bootloader. 
    \warn{Installing non-US firmware on a US \playername{} will
    permanently remove DRM support from the player.}}
  }%
  If the file that you downloaded is a \fname{.zip} file, use an unzip 
  utility like mentioned in the prerequisites section to extract
  the \fname{.hex} from the \fname{.zip} file
  to your desktop. Likewise, if the file that you downloaded is an 
  \fname{.exe} file, double-click on the \fname{.exe} file to extract 
  the \fname{.hex} file to your desktop.
  When running Linux you should be able extracting \fname{.exe}
  files using \fname{unzip}.
  %
  \item Download the firmware patcher \fname{fwpatcher.exe} from 
  \url{http://download.rockbox.org/bootloader/iriver/} and save it to your desktop.
    \warn{The firmware patcher contains Unicode support, which is not supported by 
    all versions of Windows. If you have difficulty with the firmware patcher, try 
    downloading the alternate firmware patcher \fname{fwpatchernu.exe}, which is 
    built without Unicode support.}
  %
  \item Go to your desktop and double-click on whichever version of the firmware 
  patcher you downloaded in the prior step.
  %
  \item In the firmware patcher dialog box, click on the \setting{Browse}
  button and navigate
  to the \fname{.hex} file that you previously downloaded to your desktop.
  %
  \item Click \setting{Patch}. The firmware patcher will patch the
    original firmware to include the Rockbox boot loader. The \fname{.hex}
    file on your desktop is now a modified version of the original
    \fname{.hex} file.
  %
  \item Turn on your \playerman{} and connect it to your computer via USB.
  %
  \item Copy or move the modified \fname{.hex} file directly to the root of
    your \daps{} drive. Do not put it inside a directory on your \dap.
  %
  \item Disconnect the jukebox from USB. (Be sure to use Windows' ``safely remove
  hardware'' option.)
  \warn{Before proceeding further, make sure that your player has a full charge
    or that it is connected to the power adaptor. Interrupting the next step
    due to a power failure most likely will brick your \dap{}.}
  %
  \item Update your \daps{} firmware with the patched boot loader. To do this, turn
    the jukebox on. Press and hold the
    \opt{IRIVER_H100_PAD,IRIVER_H300_PAD}{\ButtonSelect{} button }%
    to enter the main menu, and navigate to \setting{General $\rightarrow$ Firmware 
    Upgrade}. Select \setting{Yes} when asked to confirm if you want to upgrade the 
    firmware. The \playerman{} will display a message indicating that the
    firmware update 
    is in progress. Do \emph{not} interrupt this process. When the
    firmware update is complete the player will turn itself off. (The update
    firmware process usually takes a minute or so.)
    
    You have now installed the Rockbox boot loader.
\end{enumerate}

\note{If you install the Rockbox boot loader but do not install the
  Rockbox firmware the Rockbox boot loader will load the Iriver firmware when the
  jukebox is turned on.
  To load the \playerman{} firmware press and hold \ButtonRec{} before
  powering up the \dap{} until the \playerman{} logo appears.
  }

\note{The boot loader has a built-in ``boot loader USB mode''. This function
  switches to USB mode when the \dap{} is connected to a computer upon
  power-up. This way you can access the \daps{} hard disk without the need
  to boot any firmware (which is also useful when your hard disk is 
  damaged). The screen will simply display the text ``boot loader USB mode''.
  After you disconnect the \dap{} from USB the boot loader will
  continue booting Rockbox. As in boot loader USB mode the firmware
  itself hasn't been loaded this is also a simple way of updating Rockbox.
  After the disconnect the boot loader will load the updated version of
  Rockbox.
}
}
  \opt{ipod}{% $Id$ %

\opt{ipodvideo}{\newcommand{\bootloaderfile}{bootloader-ipodvideo.ipod}}%
\opt{ipodmini}{\newcommand{\bootloaderfile}{bootloader-ipodmini.ipod}}%
\opt{ipodnano}{\newcommand{\bootloaderfile}{bootloader-ipodnano.ipod}}%
\opt{ipodcolor}{\newcommand{\bootloaderfile}{bootloader-ipodcolor.ipod}}%
\opt{ipod4g}{\newcommand{\bootloaderfile}{bootloader-ipod4g.ipod}}%
\opt{ipod3g}{\newcommand{\bootloaderfile}{bootloader-ipod3g.ipod}}%
%
\opt{ipodnano}{\warn{If your Nano has a stainless steel back and plastic front 
it is a 1st generation and is compatible with Rockbox.  If, on the other hand, 
your Nano has a one-piece aluminum body it is a 2nd generation Nano and there 
is currently no Rockbox port available.  Do not attempt to install the 
bootloader on a 2nd generation Nano}}

In order to make your Ipod load and execute the Rockbox firmware you have just 
installed, you will need to install the Rockbox bootloader. Unless bugs are 
found in the bootloader code, or significant new feature are added, you will 
only have to perform this step once.

The following instructions refer to the ``installation folder.''  For Windows 
users, the ``installation folder'' is a folder in the root (top-level) of the C: 
drive called \fname{\textbackslash{}rockbox} (you will obviously need to create 
this folder yourself).  For Mac OS X and Linux users, the ``installation 
folder'' is assumed to be the Desktop folder.  Note that the bootloader 
installation files should be saved onto your computer's hard disk, \emph{not} on 
your Ipod. 

\begin{enumerate} 

  \item First, download the \fname{ipodpatcher} tool to your installation 
  folder.  You can download the \fname{ipodpatcher} tool for your operating 
  system at \download{bootloader/ipod/ipodpatcher/}.
  
  \item Next, download the following file to the installation folder: 

    \download{bootloader/ipod/\bootloaderfile}
    \opt{ipodmini}{%
        or \download{bootloader/ipod/bootloader-ipodmini2g.ipod}
        depending on which generation your \dap{} is.
        The following page describes the differences between the two
        generations of the \dap{}: 
        \url{http://docs.info.apple.com/article.html?artnum=300850}.
    }

  \item Next, open a command prompt (Windows) or terminal window (Mac OSX and Linux).
  
    Windows users will perform this and the following steps from the Windows 
    command prompt.  To start a command prompt, click \fname{start}, and then 
    click \fname{Run...}.  Type ``cmd'' and press \fname{Enter}.  Navigate 
    to the installation directory by typing the following command:

    \begin{code} 
        cd \textbackslash{}rockbox
      \end{code}
      
    Mac OS X and Linux/Unix users will perform these steps from the Terminal. 
    Start a new terminal window and navigate to the Desktop folder (type cd 
    Desktop into the terminal and press enter). You then need to ensure that the 
    ipodpatcher program is ``executable'' by typing the command chmod +x 
    ipodpatcher and then pressing \fname{Enter}.
  
  \item Connect your Ipod to your computer.

    If you haven't already done so, you should now plug your Ipod into your 
    computer (via either the USB or Firewire cable).

    \fixme{Notes about closing itunes, enabling the ``show ipod as disk'' option 
    in ipod, anything else?}

  \item Find your Ipod with ipodpatcher (Windows users only)

    Windows users:  Type the following command to search for Ipods attached to 
    your computer: 
      \begin{code} 
        ipodpatcher --scan 
      \end{code}
    
    When ipodpatcher finds your Ipod, remember the number it displays after the 
    words ``disk device''- this will  be the number you use to access your Ipod 
    in the following steps.  So, for example, if ipodpatcher displays ``disk 
    device 1'' you will use the number 1 in the commands described below.

    \note{Windows users require administrator rights for running ipodpatcher. 
    Either re-login as administrator, or open a command prompt running under an 
    administrator account by using one of the "Run as" features of Windows XP.}

  \item Find your Ipod (Mac OS X users only)

    Attach your Ipod to your Mac (using either USB or Firewire) and wait for 
    iTunes to open. When iTunes opens, close it down.  In your Terminal window, 
    type the command mount and press enter. This will list all the disks (and 
    other devices) that are "mounted" on your computer. The last  drive in the 
    list should be your Ipod. For example: 
    \begin{code}
       /dev/disk1s2 on /Volumes/DAVE_S IPOD 1 (local, nodev, nosuid) 
    \end{code}

    In order to install the ipod bootloader, you need to ``unmount'' this disk 
    using the following command: 
      \begin{code} 
        diskutil unmount /dev/disk1s2 
      \end{code}
      
    replacing ``/dev/disk1s2'' with the device name Mac OS has assigned to your
    Ipod. This may take a few seconds, after which Mac OS will say ``Volume 
    /dev/disk1s2 unmounted.'' ``/dev/disk1s2'' refers to the second partition on 
    /dev/disk1 - remember   ``/dev/disk1'' for the next step.

    It's possible that itunes will try to be ``helpful'' and remount your Ipod 
    after you modify it with ipodpatcher. If this happens, you need to unmount 
    it again using the above command. 
  
  \item Create a backup of your Ipod's firmware partition

    Type the following command, replacing ``N'' with the number (for 
    Windows users) or the device name (Mac OS X and Unix users) assigned to 
    your Ipod that you identified in the previous step: 
      \begin{code} 
        ipodpatcher N -r bootpartition.bin (Windows) 
      \end{code}
      or
      \begin{code}
        ./ipodpatcher N -r bootpartition.bin (Mac OS X/Unix)
      \end{code}
  
    This should create a file in the current folder called 
    \fname{bootpartition.bin} (approximately 40MB for the iPod 3G, 4G and 
    Color/Photo, 80MB for the Nano 1st gen and 30GB Video, and 112MB for the 
    60GB Video) containing a copy of the ``firmware partition'' from your Ipod.

    If it ever becomes necessary (for example, if your Ipod refuses to start), 
    you can restore this backup to your Ipod using the command ipodpatcher N -w 
    bootpartition.bin (Windows) or ./ipodpatcher N -w bootpartition.bin (Mac OS 
    X/Unix).   

    \opt{ipodmini}{
      \note{Ipod Mini 2g users need to replace ``1g'' with ``2g'' in the
      following commands.} 
    }
 
  \item Install the bootloader.
    Windows users should now type:
    \begin{code}
      ipodpatcher N -a \bootloaderfile
    \end{code}
    %

    and Mac OS X/Unix users should type:

    \begin{code}
      ./ipodpatcher N -a \bootloaderfile
    \end{code}

  Replace N with the number (Windows users) or device name (Mac OS X/Unix 
  users) you've been using to access your Ipod.  
  
  You can now disconnect your Ipod from your computer in the normal way. This 
  should cause your Ipod to reboot and start Rockbox.
  
  \note{If your Ipod displays the message ``Error: -1,'' you have either 
  neglected to install a Rockbox build as described in the preceding section, 
  or you have extracted the contents of the \fname{.zip} file to some 
  directory other than the the root directory of your Ipod.  To fix this 
  error, following the directions in the preceding section for downloading and 
  installing a Daily Build.}
  
\end{enumerate} 
}
  \opt{m5,x5}{The \playername{} has a built-in bootloader which performs the
firmware update and can also access the hard drive via USB.  The
Rockbox bootloader can therefore be very minimalistic, as it does not require
 it's own USB mode.  This makes it less dangerous to install the Rockbox bootloader
 as you can always restore it using the \playerman{} bootloader.

\note{The Rockbox bootloader overwrites the original firmware, making it
   impossible to dual-boot.}

\subsubsection{Installation}
\begin{itemize}
\item Download the Rockbox bootloader binary from 
\url{http://download.rockbox.org/bootloader/iaudio/}.
  \opt{x5}{Use the \fname{x5v\_fw.bin} file if your \dap{} is a X5V. If it is a X5
    or X5L, use the \fname{x5\_fw.bin} file.}
  \opt{m5}{Use the \fname{m5\_fw.bin} file.}
\item Copy it to the \fname{FIRMWARE} directory on your \dap{}.
\item Turn the \dap{} off, remove the USB cable and insert the charger. The
Rockbox bootloader will automatically be flashed.
\end{itemize}
}
  \opt{h10,h10_5gb}{\subsubsection{Installation}
\begin{enumerate}
  \item Download 
  \opt{iriverh10}{\url{http://download.rockbox.org/bootloader/iriver/H10_20GC.mi4}}
  \opt{iriverh10_5gb}{
    \begin{itemize}
      \item \url{http://download.rockbox.org/bootloader/iriver/H10.mi4} if your \dap{} is UMS or
      \item \url{http://download.rockbox.org/bootloader/iriver/H10_5GB-MTP/H10.mi4} if it is MTP.
    \end{itemize}}
  \item Connect your \archosplayertype{} to the computer using UMS mode and the UMS trick\opt{iriverh10_5gb}{ if necessary}.
  \item Rename the \opt{iriverh10}{\fname{H10\_20GC.mi4}}\opt{iriverh10_5gb}{\fname{H10.mi4}} file to \fname{OF.mi4} in the \fname{System} directory on your \archosplayertype{}.
    \opt{iriverh10_5gb}{\note{If you have a Pure model \archosplayertype{} (which does not have a FM radio) it is possible that this file will be called \fname{H10EMP.mi4} instead. If so, rename the \fname{H10.mi4} you downloaded in step 1 to \fname{H10EMP.mi4}.}}
    \note{You should keep a safe backup of this file for use if you ever wish to switch back to the \archosplayerman{} firmware.}
    \note{If you cannot see the \fname{System} directory, you will need to make sure your operating system is configured to show hidden files and directories.}

  \item Copy the \opt{iriverh10}{\fname{H10\_20GC.mi4}}\opt{iriverh10_5gb}{\fname{H10.mi4} (or \fname{H10EMP.mi4} if you have a \archosplayertype{} Pure)} file you downloaded to the System directory on your \dap{}.
\end{enumerate}
}
  \opt{gigabeatf}{% $Id$

\begin{itemize}
\item Download the Rockbox bootloader from
  \url{http://download.rockbox.org/bootloader/gigabeat/}
\item Starting at the root directory of your player browse into the directory
  \fname{GBSYSTEM} and from that into the subdirectory \fname{FWIMG}.
  These directories are hidden. Make sure that you have configured your browser
  to show hidden files or you may be unable to see \fname{FWIMG}.
\item In that directory you'll find a file called \fname{FWIMG01.DAT}. This too
  may be hidden. Rename the file to \fname{FWIMG01.DAT.ORIG}. Make sure you
  spelled that name  correctly as it is needed for booting the \playerman{} firmware.
  \warn{If you do not complete this step then you will be unable to uninstall Rockbox
  without a copy of the original firmware from the original install CD.}
\item Now copy the file \fname{FWIMG01.DAT} you downloaded to that directory.
  Make sure the spelling is correct.
\end{itemize}}
  \opt{sansa}{% $Id:$ %
\fixme{This information is new and might contain errors. Please
  \emph{always} check out the installation page in the wiki at
  \wikilink{SansaE200Install} and the troubleshooting page at
  \wikilink{SansaE200TroubleShooting} first. If you have any doubts
  about installation, \emph{stop now!} Errors during the installation might
  render your player useless!}

The installation of the bootloader is the most critical part of the
installation. Please make sure to read the instructions completely
first before doing any installation step. You \emph{need} to
prepare your \dap{} for dual boot as explained below as you
otherwise \emph{will not} have USB access to your player.

\begin{enumerate}
\item Save a copy of your original firmware onto your computer's hard drive.
  You can obtain a firmware file from \url{http://daniel.haxx.se/sansa/mi4.html},
  another way to obtain the file is to intercept the file during the firmware
  upgrade.
\item Install a version of the mi4code program on your system.
  You can get a windows binary from
  \url{http://daniel.haxx.se/sansa/mi4code.html}.
\item Decrypt the mi4 firmware file with the following command
  \begin{code}
    mi4code decrypt -s SKU_E-PP5022.mi4 OF.bin
  \end{code}
\item Copy the decrypted original firmware file \fname{OF.bin} to the 
  \fname{/SYSTEM} folder on your \dap{}. (This folder might be hidden).
\item Download the Rockbox bootloader from 
  \url{http://download.rockbox.org/bootloader/sandisk-sansa/}.
\item Copy the bootloader you just downloaded to the root directory
  of your \dap{}.
\item Safely remove your \dap{} from the computer and then disconnect the
  USB cable. This will reboot the \dap{}, then it will install the Rockbox
  bootloader and reboot again into Rockbox.
\end{enumerate}

}
  \opt{mrobe100}{\subsubsection{Installation}
\begin{enumerate}
  \item Download 
  \opt{mrobe100}{\url{http://download.rockbox.org/bootloader/olympus/mrobe100/pp5020.mi4}}
  \item Connect your \playertype{} to the computer.
  \item Rename the original \fname{pp5020.mi4} file to \fname{OF.mi4} in the \fname{System} directory on your \playertype{}.
    \note{You should keep a safe backup of this file for use if you ever wish to switch back to the \playerman{} firmware.}
    \note{If you cannot see the \fname{System} directory, you will need to make sure your operating system is configured to show hidden files and directories.}

  \item Copy the \fname{pp5020.mi4} file you downloaded to the System directory on your \dap{}.
\end{enumerate}
}
  \opt{gigabeats}{% $Id$

\warn{Before starting this procedure, ensure that you have a copy
of the original \playerman{} firmware. Without this, it is
\emph{not} possible to uninstall Rockbox. The \playerman{}
firmware can be downloaded from
\url{http://www.tacp.toshiba.com/tacpassets-images/firmware/MESV12US.zip}.\\}

Installing the bootloader is only needed once. It involves replacing the
existing firmware file on your \dap{} with another version.
When running the original \playerman{} firmware (a version of Windows CE), it is
only possible to connect the \dap{} to a PC in ``MTP mode'', which hides
the actual content of your \daps{} disk and provides restricted access
to its contents.
In reality, the \daps{} hard disk contains two partitions, a small
(150MB) ``firmware partition'' containing the \daps{} firmware (operating
system), and a second ``data partition'' containing your media files. The main
firmware file in the bootloader partition is called \fname{nk.bin}, and
this is the file that is loaded into RAM (by the \daps{} ROM-based
bootloader) and executed when your \dap{} is powered on.

\subsubsection{Bootloader installation from Windows}

\begin{enumerate}

\item Attach your \dap{} to your computer.

\item Download \fname{beastpatcher.exe} from
\fixme{add download location}
and run it.

\item You should see some information displayed about
your \dap{} and a message asking you if you wish to install the Rockbox
bootloader. Press i followed by ENTER, and beastpatcher will
install the bootloader. After a short time you should see the message
``[INFO] Bootloader installed successfully'' followed by some error
messages that you can safely ignore. Press ENTER again to exit
beastpatcher.

\item After a successful installation, your \dap{} will immediately turn off.
Turn it on again, and (because it is still connected to your PC)
it will enter the Rockbox bootloader's
``USB Mass Storage'' mode, which exposes your \daps{} disk to your computer
as a standard USB Mass Storage device.
\end{enumerate}

\subsubsection{Bootloader installation from Mac OS X}
\begin{enumerate}
\item Attach your \dap{} to your computer.

\item Download and open beastpatcher.dmg from 
\fixme{add download location}
and then double-click on the beastpatcher icon inside. You can also
drag the beastpatcher icon to a location on your hard drive and launch
it from the Terminal.

\item If all has gone well, you should see some 
information displayed about your \dap{} and a message asking you if you 
wish to install the Rockbox bootloader. Press i followed by ENTER, and 
beastpatcher will now install the bootloader. After a short time you 
should see the message ``[INFO] Bootloader installed successfully''
followed by some error messages that you can safely ignore. Press 
ENTER again to exit beastpatcher and then quit the Terminal application.

\item After a successful installation, your \dap{} will immediately turn off.
Turn it on again, and (because it is still connected to your Mac)
it will enter the Rockbox bootloader's
``USB Mass Storage'' mode, which exposes your \daps{} disk to your computer
as a standard USB Mass Storage device.
\end{enumerate}

\subsubsection{Bootloader installation from Linux}

\begin{enumerate}

\item Download beastpatcher from
\fixme{add download location} (32-bit x86 
binary) or \fixme{add download location} 
(64-bit amd64 binary). You can save this anywhere you wish, but the next 
steps will assume you have saved it in your home directory.

\item Attach your \dap{} to your computer.

\item Open up a terminal window and type the following commands:

\begin{code} 
    cd $HOME
    chmod +x beastpatcher
    ./beastpatcher
\end{code}

\item If all has gone well, you should see some information displayed about
your \dap{} and a message asking you if you wish to install the Rockbox
bootloader. Press i followed by ENTER, and beastpatcher will now install the
bootloader. After a short time you should see the message ``[INFO] Bootloader
installed successfully'' followed by some error
messages that you can safely ignore. Press ENTER again to exit beastpatcher.

\item After a successful installation, your \dap{} will immediately turn off.
Turn it on again, and (because it is still connected to your PC)
it will enter the Rockbox bootloader's
``USB Mass Storage'' mode, which exposes your \daps{} disk to your computer
as a standard USB Mass Storage device.

\end{enumerate}}
}

\subsection{Enabling Speech Support (optional)}\label{sec:enabling_speech_support}
\index{Speech}\index{Installation!Optional Steps}
If you wish to use speech support you will also need a voice file, English ones
are available from \url{http://www.rockbox.org/daily.shtml}. Download the
``voice'' package for your player and unzip it directly to the root of your \dap.
You should now find an \fname{english.voice} in the \fname{/.rockbox/langs}
directory on your \dap{}. Voice menus are enabled by default and will come
into effect after a reboot. See \reference{ref:Voiceconfiguration} for details
on voice settings.

\section{Running Rockbox}
Remove your \dap{} from the computer's USB port. %
\nopt{ipod,e200}{Unplug any connected power supply and turn the unit off. When
you next turn the unit on, Rockbox should load. }%
\opt{ipod}{Hard resetting the Ipod by holding
  \opt{IPOD_4G_PAD}{\ButtonMenu{}+\ButtonSelect{}}%
  \opt{IPOD_3G_PAD}{\ButtonMenu{}+\ButtonPlay{}}
  for a couple of seconds until the \dap{} resets. Now Rockbox should load.
} %
\opt{e200}{Your e200 will automatically reboot and Rockbox should load. }%
When you see the Rockbox splash screen, Rockbox is loaded and ready for
use.

\opt{ipod}{
  \note{
    If you have loaded music onto your \dap{} using Itunes, 
    you will not be able to see your music properly in the \setting{File Browser}. 
	This is because Itunes changes your files' names and hides them in 
	directories in the \fname{Ipod\_Control} directory. Files placed on your 
	\dap{} using Itunes can be viewed by initializing and using Rockbox's database.
	See \reference{ref:database} for more information.
  }
}

\section{Updating Rockbox}
Updating Rockbox is easy even if you do not use the Rockbox Utility.
Download a Rockbox build.
(The latest release of the Rockbox software will always be available from
\url{http://www.rockbox.org/download/}). Unzip the build to the root directory
of your \dap{} like you did in the installation stage. If your unzip
program asks you whether to overwrite files, choose the ``Yes to all'' option.
The new build will be installed over your current build.

\note{If you use Rockbox Utility be aware that it cannot detect manually
        installed components.}

\section{Uninstalling Rockbox}\index{Installation!uninstall}

\nopt{gigabeatf,m5,x5,archos,mrobe100}{
  \note{The Rockbox bootloader allows you to choose between Rockbox and 
  the original firmware. (See \reference{ref:Dualboot} for more information.)}
}

\subsection{Automatic Uninstallation}
You can uninstall Rockbox automatically by using Rockbox Utility. If you
installed Rockbox manually you can still use Rockbox Utility for uninstallation
but will not be able to do this selectively.

\opt{h1xx,h300}{\note{Rockbox Utility cannot uninstall the bootloader due to
the fact that it requires a flashing procedure. To uninstall the bootloader
completely follow the manual uninstallation instructions below.}}

\subsection{Manual Uninstallation}

\opt{archos}{
  If you would like to go back to using the original \playerman{} software,
  connect the \dap{} to your computer, and delete the
  \fname{\firmwarefilename} file.
}

\opt{h10,h10_5gb}{
  If you would like to go back to using the original \playerman{} software,
  connect the \dap{} to your computer, and delete the
  \opt{h10}{\fname{H10\_20GC.mi4}}\opt{h10_5gb}{\fname{H10.mi4}} file and rename
  \fname{OF.mi4} to \opt{h10}{\fname{H10\_20GC.mi4}}\opt{h10_5gb}{\fname{H10.mi4}}
  in the \fname{System} directory on your \playertype{}. As in the installation,
  it may be necessary to first put your device into UMS mode.
}

\opt{mrobe100}{
  If you would like to go back to using the original \playerman{} software,
  connect the \dap{} to your computer, and delete the
  \fname{pp5020.mi4} file and rename
  \fname{OF.mi4} to \fname{pp5020.mi4}
  in the \fname{System} directory on your \playertype{}.
}

\opt{e200}{
  If you would like to go back to using the original \playerman{} software,
  connect the \dap{} to your computer, and follow the instructions to install
  the bootloader, but when prompted by sansapatcher, enter \texttt{u} for uninstall,
  instead of \texttt{i} for install. As in the installation, it may be necessary to
  first put your device into MSC mode.
}

\optv{ipod}{
  To uninstall Rockbox and go back to using just the original Ipod software, connect
  the \dap{} to your computer and follow the instructions to install 
  the bootloader but, when prompted by ipodpatcher, enter \texttt{u} for uninstall 
  instead of \texttt{i} for install.
}

\opt{m5,x5}{
  If you would like to go back to using the original \playerman{} software,
  connect the \dap{} to your computer, download the original \playername{}
  firmware from the \playerman{} website, and copy it to the \fname{FIRMWARE}
  directory on your \playername{}. Turn off the \dap{}, remove the USB cable
  and insert the charger. The original firmware will automatically be flashed.
}

\opt{h1xx,h300}{
  \note{
    If you want to remove the Rockbox bootloader, simply flash an unpatched
    \playerman{} firmware. Be aware that doing so will also remove the bootloader
    USB mode. As that mode can come in quite handy (especially when
    having disk errors) it is recommended to keep the bootloader. It also
    gives you the possibility of trying Rockbox anytime later by simply
    installing the distribution files.
    \opt{h1xx}{
      The Rockbox bootloader will automatically start the original firmware if
      the \fname{.rockbox} directory has been deleted.
    }%
    \opt{h300}{%
      Although if you retain the Rockbox bootloader, you will need to hold the
      \ButtonRec{} button each time you want to start the original firmware.
    }
  }
}

If you wish to clean up your disk, you may also wish to delete the
\fname{.rockbox} directory and its contents.
\nopt{m5,x5}{Turn the \playerman{} off.
  Turn the \dap{} back on and the original \playerman{} software will load.
}

